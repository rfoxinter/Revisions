\documentclass[14pt,usepdftitle=false,aspectratio=169]{beamer}
\usepackage{preambule}
\setbeamercolor{structure}{fg=black}
\usepackage{analyse,equivalents}
\hypersetup{pdftitle=Analyse -- Séries numériques}
\title{Analyse\\\emph{Séries numériques}}
\author{}
\date{}
\begin{document}
\begin{frame}
    \titlepage
\end{frame}
\slideq{Sommation des relations de comparaison dans le cas divergeant\linebreak$\serie{v_n}$ diverge}{1/2}
\slider{$u_n=\O{v_n}\Rightarrow\sum{k=0}{n}{u_k}=\O{\sum{k=0}{n}{v_k}}$\linebreak$u_n=\o{v_n}\Rightarrow\sum{k=0}{n}{u_k}=\o{\sum{k=0}{n}{v_k}}$\linebreak$\eq{u_n}{v_n}\Rightarrow\eq{\sum{k=0}{n}{u_k}}{\sum{k=0}{n}{v_k}}$\linebreak Dans ce dernier cas, $\serie{u_n}$ diverge}{1/2}
\slideq{Sommation des relations de comparaison dans le cas convergeant\linebreak$\serie{v_n}$ converge}{2/2}
\slider{$u_n=\O{v_n}\Rightarrow\sum{k=n+1}{+\infty}{u_k}=\O{\sum{k=n+1}{+\infty}{v_k}}$\linebreak$u_n=\o{v_n}\Rightarrow\sum{k=n+1}{+\infty}{u_k}=\o{\sum{k=n+1}{+\infty}{v_k}}$\linebreak$\eq{u_n}{v_n}\Rightarrow\eq{\sum{k=n+1}{+\infty}{u_k}}{\sum{k=n+1}{+\infty}{v_k}}$\linebreak Dans ces trois cas, $\serie{u_n}$ converge}{2/2}
\end{document}
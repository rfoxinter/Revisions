\documentclass[14pt,usepdftitle=false,aspectratio=169]{beamer}
\usepackage{preambule}
\setbeamercolor{structure}{fg=black}
\usepackage{topologie}\usepackage{analyse}
\hypersetup{pdftitle=Calcul différentiel -- Théorème d’inversion locale}
\title{Calcul différentiel\\\emph{Théorème d'inversion locale}}
\author{}
\date{}
\begin{document}
\begin{frame}
    \titlepage
\end{frame}
\slideq{Théorème d'inversion locale}{1/2}
\slider{Si $f$ est $\mathcal C^1$ sur $U$ et si $x_0\in U$ tel que $\dd f_{x_0}$ est un homéomorphisme alors il existe $V\in\mathcal V\l x_0\r$ et $W\in\mathcal V\l f\l x_0\r\r$ tels que $f$ induit un difféomorphisme de $V$ dans $W$}{1/2}
\slideq{Théorème du point fixe de Banach}{2/2}
\slider{Si $\l E,d\r$ est un espace métrique complet et $f\!:\!E\to E$ est une fonction $\kappa$-contractante, $\kappa<1$ alors $f$ admet un unique point fixe qui est la limite des suites définies par $x_0\in E$ et $x_{n+1}=f\l x_n\r$\linebreak De plus, $\nrm{x-x_n}\leqslant\frac{k^n}{1-k}\nrm{x_1-x_0}$}{2/2}
\end{document}
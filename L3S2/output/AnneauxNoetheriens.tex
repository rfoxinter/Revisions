\documentclass[14pt,usepdftitle=false,aspectratio=169]{beamer}
\usepackage{preambule}
\setbeamercolor{structure}{fg=black}
\usepackage{bigoperators}\usepackage{usuelles,polynomes}
\hypersetup{pdftitle=Algèbre 2 -- Anneaux noethériens}
\title{Algèbre 2\\\emph{Anneaux noethériens}}
\author{}
\date{}
\begin{document}
\begin{frame}
    \titlepage
\end{frame}
\slideq{Théorème de Krull}{1/6}
\slider{Si $A$ est commutatif alors $A$ admet un idéal maximal propre}{1/6}
\slideq{Propriétés des algèbres de type fini pour $A$ noethérien}{2/6}
\slider{Une sous-algèbre de type fini d'un anneau noethérien est noethérien}{2/6}
\slideq{Algèbre de type fini}{3/6}
\slider{$B$ est une $A$-algèbre de type fini s'il existe $\l b_1,\cdots,b_n\r\in B^n$ tel que pour tout $b\in B$, il existe $P\in A\left[X_1,\cdots,X_n\right]$ tel que $b=P\l b_1,\cdots,b_n\r$}{3/6}
\slideq{Théorème de la base de Hilbert}{4/6}
\slider{Si $A$ est noethérien alors $A\left[X_1,\cdots,X_n\right]$ est noethérien}{4/6}
\slideq{Propriété de $A/I$}{5/6}
\slider{Si $A$ est noethérien et $I\vartriangleleft A$ alors $A/I$ est noethérien}{5/6}
\slideq{Anneau noethérien}{6/6}
\slider{$A$ est noethérien si toute soute croissante d'idéaux de $A$ est stationnaire\linebreak$A$ est noethérien si tout idéal de $A$ est de type fini}{6/6}
\end{document}
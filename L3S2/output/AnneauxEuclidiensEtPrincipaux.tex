\documentclass[14pt,usepdftitle=false,aspectratio=169]{beamer}
\usepackage{preambule}
\setbeamercolor{structure}{fg=black}

\hypersetup{pdftitle=Algèbre 2 -- Anneaux euclidiens et principaux}
\title{Algèbre 2\\\emph{Anneaux euclidiens et principaux}}
\author{}
\date{}
\begin{document}
\begin{frame}
    \titlepage
\end{frame}
\slideq{CNS sur les idéaux pour $a\mid b\wedge b\mid a$}{1/10}
\slider{Pour $A$ intègre, $\l a\r=\l b\r$\linebreak$\exists u\in A^\times$, $a=bu$}{1/10}
\slideq{CNS sur les idéaux pour $a\mid b$}{2/10}
\slider{Pour $A$ intègre, $\l b\r\subset\l a\r$}{2/10}
\slideq{CNS sur les idéaux pour $a$ et $b$ premiers entre eux}{3/10}
\slider{Pour $A$ intègre, $\l b\r+\l a\r=\l a,b\r=A$}{3/10}
\slideq{Identité de Bézout}{4/10}
\slider{Si $\l a,b\r\in A^2\setminus\left\{0,0\right\}$, il existe $\l u,b\r\in A^2$ tel que $au+bv=a\wedge b$}{4/10}
\slideq{Anneau principal}{5/10}
\slider{Un anneau est principal s'il est intègre et si tous ses idéaux sont principaux}{5/10}
\slideq{Lien entre anneau euclidien et principal}{6/10}
\slider{Tout anneau euclidien est principal}{6/10}
\slideq{$a\wedge b$}{7/10}
\slider{Pour $A$ intègre, $a\wedge b$ est tel que $\l a\wedge b\r=\l a,b\r$}{7/10}
\slideq{CNS sur les idéaux pour $a\in A^\times$}{8/10}
\slider{Pour $A$ intègre, $\l a\r=\l1\r=A$}{8/10}
\slideq{CNS pour $a\neq0$ irréductible dans $A$ principal}{9/10}
\slider{$\l a\r$ est maximal\linebreak De manière équivalente, $a$ est premier}{9/10}
\slideq{Anneau euclidien}{10/10}
\slider{Un anneau $A$ est euclidien s'il est intègre, avec un stathme $N\!:\!A^*\to\mathbb Z$ tel que pour tout $a\in A$ et tout $b\in A^*$, il existe $\l q,r\r\in A^2$ tel que $a=bq+r$ et $r=0$ ou $N\l r\r<N\l b\r$}{10/10}
\end{document}
\documentclass[14pt,usepdftitle=false,aspectratio=169]{beamer}
\usepackage{preambule}
\setbeamercolor{structure}{fg=black}
\usepackage{analyse}\toggleanalysepar\usepackage{complexes}\def\ff{\mathop{}\!\mathcal F\mkern-2.5mu}\usepackage{matrices,al}\newcommand{\multider}[3]{\frac{\partial^{\left|#3\right|#1}}{\partial#2^{#3}}}\usepackage{equivalents}\usepackage{usuelles}\usepackage{dsft}
\hypersetup{pdftitle=Intégration et probabilités -- Transformée de Fourier}
\title{Intégration et probabilités\\\emph{Transformée de Fourier}}
\author{}
\date{}
\begin{document}
\begin{frame}
    \titlepage
\end{frame}
\slideq{Limite de $\ff f$}{1/19}
\slider{$\ff f$ est continue et $\ff f\l\xi\r\xrightarrow[\left|\xi\right|\to+\infty]{}0$}{1/19}
\slideq{Proporété de $\mathcal F\!:\!\mathcal M_s\l\mathbb R^d\r\to\mathcal C\l\mathbb R^d,\mathbb C\r$}{2/19}
\slider{$\mathcal F$ est injective}{2/19}
\slideq{Théorème de Hahn-Jordan}{3/19}
\slider{Si $\mu$ est une mesure signée sur $\l\mathbb R^d,\mathcal B\l\mathbb R^d\r\r$ alors $\mu$ se décompose de manière unique en deux mesures positives $\mu_+$ et $\mu_-$ étrangères telles que $\mu=\mu_+-\mu_-$}{3/19}
\slideq{Lemme de réciprocité pour des mesures}{4/19}
\slider{Si $\mu$ et $\nu$ sont deux mesures signées, $\int[x][\mathbb R^d][][\nu]{\ff \mu\l x\r}=\int[x][\mathbb R^d][][\mu]{\ff \nu\l x\r}$}{4/19}
\slideq{Lemme de réciprocité}{5/19}
\slider{Si $f$ et $g$ sont deux fonctions dans $\mathcal L^1$, $\int[x][\mathbb R^d][][\lambda_d]{\ff f\l x\r g\l x\r}=\int[x][\mathbb R^d][][\lambda_d]{f\l x\r\ff g\l x\r}$}{5/19}
\slideq{$\ff\l\e_yf\r$}{6/19}
\slider{$\tau_y\ff f$}{6/19}
\slideq{$\lambda_d\l\dd x\r$}{7/19}
\slider{$\frac{\dd x}{\l2\pi\r^{d/2}}$}{7/19}
\slideq{$\ff\l f*g\r$}{8/19}
\slider{$\ff f\times\ff g$}{8/19}
\slideq{$\ff\l\tau_yf\r$}{9/19}
\slider{$\e_{-y}\ff f$}{9/19}
\slideq{$\ff g\l\xi\r$\linebreak$f\in\mathcal L^1$, $M\in\matgl d{\mathbb R}$, $g\l x\r=f\l M^{-1}x\r$}{10/19}
\slider{$\left|\det M\right|\ff f\l M^\top\xi\r$}{10/19}
\slideq{$\ff f\l\xi\r$}{11/19}
\slider{$\int[x][\mathbb R^d][][\lambda_d]{f\l x\r\e^{-\i\xi\cdot x}}$}{11/19}
\slideq{Régularité de $\appl{\mathcal F}{\mathcal L^1}{\mathcal C_0\l\mathbb R^d,\mathbb C\r}{f}{\ff f}$}{12/19}
\slider{$\mathcal F$ est $1$-lipschitzienne}{12/19}
\slideq{Intégrale de Gauss}{13/19}
\slider{$g_\sigma\l x\r=\frac{1}{\sigma^d}\exp{\frac{\left|x\right|^2}{2\sigma^2}}$\linebreak$\int[x][\mathbb R^d][][\lambda_d]{g_\sigma\l x\r}=1$}{13/19}
\slideq{$\ff\mu$}{14/19}
\slider{$\int[x][\mathbb R^d][][\mu]{\e^{-\i\xi\cdot x}}$}{14/19}
\slideq{Formule de Fourier-Plancherel}{15/19}
\slider{L'unique application $\varPhi\!:\!L^2\to L^2$ continue telle que $\varPhi_{|L^1\cap L^2}=\mathcal F_{|L^1\cap L^2}$\linebreak$\varPhi$ est une isométrie}{15/19}
\slideq{Régulatité de $\ff f$}{16/19}
\slider{Si $\left|x\right|^kf\in\mathcal L^1$ alors $\ff f\in\mathcal C^k\l\mathbb R^d,\mathbb C\r$ et pour tout $\alpha\in\mathbb N^d$, $\left|\alpha\right|\leqslant k$, $\multider{\ff f}{x}{\alpha}\l\xi\r=\int[x][\mathbb R^d][][\lambda_d]{\l-\i x\r^\alpha f\l x\r\e^{-\i\xi\cdot x}}$\linebreak En particulier, $\ff f\l\xi\r=\o[\left|\xi\right|\to+\infty]{\frac{1}{\left|\xi\right|^k}}$}{16/19}
\slideq{Théorème d'inversion de Fourier}{17/19}
\slider{Si $f\in\mathcal L^1$ telle que $\ff f\in\mathcal L^1$ $\lambda_d$ presque partout alors $f\l x\r=\int[\xi][\mathbb R^d][][\lambda_d]{\ff f\l\xi\r\e^{\i\xi\cdot x}}$}{17/19}
\slideq{$\mu*\nu$}{18/19}
\slider{$\mu*\nu\l A\r=\altint{\oldint}[x,y][\mathbb R^d\times\mathbb R^d][][\mu\otimes\nu]{\1A\l x+y\r}$}{18/19}
\slideq{Théorème d'inversion de Fourier pour les mesures}{19/19}
\slider{Soit $\mu$ une mesure signée sur $\mathbb R^d$ telle que $\ff\mu\in\mathcal L^1\l\mathbb R^d\r$ alors $\mu$ admet une densité par rapport à $\mathcal{RF}{\ff}\mu$ $\lambda_d$ presque partout}{19/19}
\end{document}
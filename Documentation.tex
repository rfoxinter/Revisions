\documentclass[a4paper,12pt]{article}
\usepackage[utf8]{inputenc}
\usepackage[french]{babel}
\usepackage[T1]{fontenc}
\usepackage[margin=2.5cm]{geometry}
\usepackage{hyperref}
\hypersetup{pdftitle=Documentation}
\let\oldoplus\oplus
\usepackage{output/htmlpreambule,output/matrices,output/al,output/bigoperators,output/analyse,output/arithmetique,output/complexes,output/dsfont,output/equivalents,output/polynomes,output/probas,output/structures,output/trigo,output/tables}
\usepackage{fvextra}
\setlength{\parindent}{0pt}
\hypersetup{hidelinks,colorlinks=true,linkcolor=black,linktoc=all,urlcolor=blue}
\setcounter{secnumdepth}{5}
\setcounter{tocdepth}{5}
\let\oldsection\section
\renewcommand\section{\clearpage\oldsection}
\title{Documentation}
\author{}
\date{}

\newcommand{\mhd}{
    \hrule\vline\hfill
    \begin{minipage}{0.475\linewidth}
        \begin{center}
            \vspace{2pt}
            \textbf{Commande}

            \vspace{2pt}
        \end{center}
    \end{minipage}
    \hfill\vline\hfill
    \begin{minipage}{0.475\linewidth}
        \begin{center}
            \vspace{2pt}
            \textbf{Résultat}
            
            \vspace{2pt}
        \end{center}
    \end{minipage}
    \hfill\vline\hrule
}
\newcommand{\mcmd}[2][]{
    \vline\hfill
    \begin{minipage}{0.475\linewidth}
        \begin{center}
            \vspace{2pt}
            \Verb?#2?#1

            \vspace{2pt}
        \end{center}
    \end{minipage}
    \hfill\vline\hfill
    \begin{minipage}{0.475\linewidth}
        \begin{center}
            \vspace{2pt}
            $#2$
            
            \vspace{2pt}
        \end{center}
    \end{minipage}
    \hfill\vline\hrule
}
\newcommand{\mmcmd}[3][]{
    \vline\hfill
    \begin{minipage}{0.475\linewidth}
        \begin{center}
            \vspace{2pt}
            \Verb?#2?#1

            \vspace{2pt}
        \end{center}
    \end{minipage}
    \hfill\vline\hfill
    \begin{minipage}{0.475\linewidth}
        \begin{center}
            \vspace{2pt}
            #3
            
            \vspace{2pt}
        \end{center}
    \end{minipage}
    \hfill\vline\hrule
}

\begin{document}
\pagenumbering{gobble}
\maketitle
Le package \texttt{Preambule.sty}\footnote{Pour utiliser avec \Verb?Beamer?} ou \texttt{HTMLPreambule.sty}\footnote{Pour les documents autres que \Verb?Beamer?} doit être chargé pour pouvoir utiliser les autres qui sont donnés ci-dessous.

Les fichiers \Verb?.sty? doivent être placés dans le même répertoire que le fichier \Verb?.tex? qui est utilisé.

Pour charger un package (par exemple \Verb?NomDuPackage.sty?), il faut utiliser la commande \Verb?\usepackage{nomdupackage}? avant \Verb?\begin{document}?.

\vspace{0.5cm}

En utilisant \texttt{Preambule.sty} ou \texttt{HTMLPreambule.sty}, les packages suivant seront chargés:
\begin{itemize}
    \renewcommand{\labelitemi}{$\to$}
    \item \Verb?\usepackage[utf8]{inputenc}?
    \item \Verb?\usepackage[french]{babel}?
    \item \Verb?\usepackage[T1]{fontenc}?
    \item \Verb?\usepackage{amsmath, amsfonts, amssymb}?
    \item \Verb?\usepackage{stmaryrd}?
    \item \Verb?\usepackage{adjustbox}? (pour \texttt{HTMLPreambule.sty})
    \item \Verb?\usepackage{xcolor}? (pour \texttt{Preambule.sty})
\end{itemize}

\vspace{0.5cm}
Il est nécessaire que \texttt{cm-super} soit installé (disponible sur \href{https://ctan.org/pkg/cm-super}{CTAN}) pour pouvoir utiliser \texttt{Preambule.sty}. Pour ne pas avoir à installer \texttt{cm-super}, il est possible de commenter les lignes \Verb?\usepackage{sffont}? et \Verb?\renewcommand{\sfdefault}{cmssp}? du fichier \texttt{Preambule.sty} en mettant \Verb?%? au début de chacune de ces lignes (numéro 10 et 11).
\tableofcontents
\null\newpage
\pagenumbering{arabic}
\section{Preambule.sty et HTMLPreambule.sty}
\mhd
\mmcmd[\footnotemark]{\l}{$\l\right.$}
\footnotetext{Correspond à la commande usuelle \Verb?\left(?}
\mmcmd[\footnotemark]{\r}{$\left.\r$}
\footnotetext{Correspond à la commande usuelle \Verb?\right)?}
\mmcmd[\footnotemark]{\llb}{$\llb\right.$}
\footnotetext{Correspond à la commande usuelle \Verb?\left\llbracket?}
\mcmd[\footnotemark]{\oldfrac{a}{b}}
\footnotetext{Correspond à la commande usuelle \Verb?\frac?}
\mcmd[\footnotemark]{\frac{a}{b}}
\footnotetext{Correspond à la commande usuelle \Verb?\dfrac?}
\mcmd[\footnotemark]{l}
\footnotetext{Correspond à la commande usuelle \Verb?\ell?}
\section{AL.sty}
Le package \hyperlink{section.9}{\Verb?Matrices.sty?} sera importé automatiquement avec \Verb?AL.sty?.

\vspace{0.5cm}

\mhd
\mcmd{\oldvect}
\mcmd{\vect{E}}
\mcmd{\al{E}{}}
\mcmd{\al{E}{F}}
\mcmd[\footnotemark]{\oplus}
\footnotetext{Le \Verb?\oplus? utilisé est celui de \Verb?stmaryrd?\\Pour récupérer celui de \LaTeX, il est possible d'utiliser la commande \Verb?\let\oldoplus\oplus? avant \Verb?\usepackage{al}? puis de faire \Verb?\let\oplus\oldoplus? après importation\\Comparaison \LaTeX~- \Verb?stmaryrd? avec le plus normal $\oldoplus\oplus+$}
\mcmd[\footnotemark]{\matgl{n}{\mathbb{K}}}
\footnotetext{Le \Verb?\matgl? de \Verb?AL.sty? correspond à la commande \Verb?\gl? de \Verb?Matrices.sty? qui a été renommé}
\mcmd{\gl{E}}
\mcmd[\footnotemark]{\olddim}
\footnotetext{Correspond à la commande usuelle \Verb?\dim?}
\mcmd{\dim{E}}
\mcmd{\oldrg}
\mcmd{\rg{u}}
\mcmd{\oldtr}
\mcmd{\tr{u}}
\mcmd{\oldmat}
\mcmd{\mat{\mathcal{B}}{}{u}}
\mcmd{\mat{\mathcal{B}}{\mathcal{C}}{u}}
\mmcmd{\lc}{$\lc\right.$}
\mmcmd{\rc}{$\left.\rc$}
\section{Analyse.sty}
Le package \hyperlink{section.4}{\Verb?BigOperators.sty?} sera importé automatiquement avec \Verb?Analyse.sty?.

\vspace{0.5cm}

\mhd
\mcmd[\footnotemark]{\oldd}
\footnotetext{$\oldd$ de dérivation}
\mcmd{\der{f(x)}}
\mcmd{\der[n]{f(x)}}
\mcmd{\der[][t]{f(t)}}
\mcmd[\footnotemark]{\oldint}
\footnotetext{Correspond à la commande usuelle \Verb?\int?}
\mcmd{\int{f}}
\mcmd{\int[t]{f(t)}}
\mcmd[\footnotemark]{\int[t][{[a,b]}]{f(t)}}
\footnotetext{L'argument \Verb?[a,b]? doit être mis entre accolades pour être traîté correctement par \LaTeX}
\mcmd{\int[t][a][b]{f(t)}}
\mcmd{\eval[{[a,b]}]{f(t)}}
\mcmd{\eval[a][b]{f(t)}}
\mcmd{\serie{a_n}}
\section{Arithmetique.sty}
\mhd
\mcmd[\footnotemark]{\olddiv}
\footnotetext{Correspond à la commande usuelle \Verb?\div?}
\mcmd[\footnotemark]{\div}
\footnotetext{Correspond à la commande usuelle \Verb?\mid?}
\mcmd{\cgr{a}{b}{n}}
\mcmd[\footnotemark]{\oldphi}
\footnotetext{Correspond à la commande usuelle \Verb?\phi?}
\mcmd[\footnotemark]{\phi}
\footnotetext{Correspond à la commande usuelle \Verb?\varphi?}
\section{BigOperators.sty}
\mhd
\mcmd[\footnotemark]{\oldsum}
\footnotetext{Correspond à la commande usuelle \Verb?\sum?}
\mcmd{\sum{n=0}{+\infty}{u_n}}
\mcmd[\footnotemark]{\oldprod}
\footnotetext{Correspond à la commande usuelle \Verb?\prod?}
\mcmd{\prod{n=0}{+\infty}{u_n}}
\mcmd[\footnotemark]{\oldcap}
\footnotetext{Correspond à la commande usuelle \Verb?\bigcap?}
\mcmd{\bigcap{n=0}{+\infty}{A_n}}
\mcmd[\footnotemark]{\oldcup}
\footnotetext{Correspond à la commande usuelle \Verb?\bigcup?}
\mcmd{\bigcup{n=0}{+\infty}{A_n}}
\mcmd[\footnotemark]{\olduplus}
\footnotetext{Correspond à la commande usuelle \Verb?\biguplus?}
\mcmd{\biguplus{n=0}{+\infty}{A_n}}
\mcmd{\bigop{n=0}{+\infty}{E_n}}
\section{Complexes.sty}
\mhd
\mcmd[\footnotemark]{\oldbar{z}}
\footnotetext{Correspond à la commande usuelle \Verb?\bar?}
\mcmd[\footnotemark]{\bar{z}}
\footnotetext{Se comporte comme \Verb?\overline?}
\mcmd[\footnotemark]{\e}
\footnotetext{$\e$ de la fonction exponentielle}
\mcmd[\footnotemark]{\i}
\footnotetext{$\i$ complexe\\L'ancienne commande \Verb?\i? s'obtient avec \Verb?\ii?}
\mcmd[\footnotemark]{\j}
\footnotetext{$\j=\e^{\oldfrac{2\i\pi}{3}}$\\L'ancienne commande \Verb?\j? s'obtient avec \Verb?\jj?}
\section{Dsfont.sty}
Ce package remplace le $\mathds 1$ du package \Verb?Dsfonts.sty? disponible sur \href{https://www.ctan.org/pkg/doublestroke}{CTAN}.

Pour l'utiliser, il faut copier les fichiers \Verb?dsrom12.pfb? et \Verb?dsrom12.tfm? dans les dossiers où ils sont actuellement avec \Verb?dsfonts? (et éventuellement créer une copie des anciens fichiers).

\vspace{0.5cm}

\mhd
\mcmd{\mathds{1}}
\mcmd{\1{E}(x)}
\mcmd{\square}
\mcmd{\star}
\mcmd{\triangle}
\section{Equivalents.sty}
\mhd
\mcmd{\o{x}}
\mcmd{\o[x\to0]{x}}
\mcmd{\O{x}}
\mcmd{\O[x\to0]{x}}
\mcmd{\Th{x}}
\mcmd{\Th[x\to0]{x}}
\mcmd{\Om{x}}
\mcmd{\Om[x\to0]{x}}
\mcmd{\eq{u_n}{v_n}}
\mcmd{\eq[n\to+\infty]{u_n}{v_n}}
\mcmd{\eg{u_n}{v_n+\o{v_n}}}
\mcmd{\eg[n\to+\infty]{u_n}{v_n+\o{v_n}}}
\section{Matrices.sty}
\mhd
\mcmd{\mat{n}{p}{\mathbb{K}}}
\mcmd{\mat{n}{}{\mathbb{K}}}
\mcmd{\sym{n}{\mathbb{K}}}
\mcmd{\ant{n}{\mathbb{K}}}
\mcmd{\diag{n}{\mathbb{K}}}
\mcmd{\ts{n}{\mathbb{K}}}
\mcmd{\ti{n}{\mathbb{K}}}
\mcmd[\footnotemark]{\olddet}
\footnotetext{Correspond à la commande usuelle \Verb?\det?}
\mcmd{\det{M}}
\mmcmd[\footnotemark]{\gl{n}{\mathbb{K}}}{$\matgl{n}{\mathbb{K}}$}
\footnotetext{Si \Verb?AL.sty? est chargé, cette commande est remplacée et il faut utiliser \Verb?\matgl{n}{\mathbb{K}}? pour obtenir ce résultat}
\mcmd{\mdots}
\mcmd{\ddots}
\mcmd{\idots}
\mcmd{\vdots}
\mcmd{\xdots}
\mcmd{\tmatrix({1\&0\\0\&1\\})}

\vspace{0.5cm}

\textbf{\Large La commande \Verb?\tmatrix?}


\Verb?\tmatrix? est composé de deux arguments optionnels (les éléments à ajouter à la matrice tikz et les éléments de mise en page de la matrice) ainsi que de trois arguments (le délimiteur d'ouverture, le contenu de la matrice et le délimiteur de fermeture).

Les commandes sont:

\vspace{0.5cm}

\mhd
\mmcmd{\mtxvline{params}{n}}{Crée une ligne verticale après la colonne \texttt{n} (ou left/right pour les extrémités) avec les paramètres tikz \texttt{params}}
\mmcmd{\mtxhline{params}{n}}{Crée une ligne horizontale après la ligne \texttt{n} (ou top/bottom pour les extrémités) avec les paramètres tikz \texttt{params}}
\mmcmd{\mtxvpartial{params}{n}{a}{b}}{Crée une ligne verticale après la colonne \texttt{n} (ou left/right pour les extrémités), la ligne ayant pour extrémités la fin de la ligne \texttt{a} et \texttt{b} (ou top/bottom) avec les paramètres tikz \texttt{params}}
\pagebreak
\hrule
\mmcmd{\mtxhpartial{params}{n}{a}{b}}{Crée une ligne horizontale après la ligne \texttt{n} (ou top/bottom pour les extrémités), la ligne ayant pour extrémités la fin de la ligne \texttt{a} et \texttt{b} (ou left/right) avec les paramètres tikz \texttt{params}}
\mmcmd{\mtxbox{params}{x}{y}}{Crée une boîte autour de la case de coordonnées \texttt{x} et \texttt{y} (l'indexation commence à 1) avec les paramètres tikz \texttt{params}}
\section{Polynomes.sty}
\mhd
\mcmd{\pol{K}{X}}
\mcmd{\fr{K}{X}}
\mcmd[\footnotemark]{\olddeg}
\footnotetext{Correspond à la commande usuelle \Verb?\deg?}
\mcmd{\deg{P}}
\mcmd{\oldval}
\mcmd{\val{P}}
\mcmd{\oldcar}
\mcmd{\car{\mathbb{K}}}
\section{Probas.sty}
\mhd
\mcmd{\p{A}}
\mcmd{\p[B]{A}}
\mcmd[\footnotemark]{\oldOmega}
\footnotetext{Correspond à la commande usuelle \Verb?\Omega?}
\mcmd[\footnotemark]{\Omega}
\footnotetext{Correspond à la commande usuelle \Verb?\varOmega?}
\mmcmd[\footnotemark]{\sq}{$\left.\sq\right.$}
\footnotetext{Doit être utilisé entre \Verb?\left? et \Verb?\right?, ou dans la commande \Verb?\p?: $\p{A\sq B}$}
\mcmd[\footnotemark]{\bor}
\footnotetext{Correspond à la commande usuelle \Verb?\mathcal{B}?}
\section{Structures.sty}
\mhd
\mcmd{\oldhom}
\mcmd{\hom{E}}
\mcmd{\oldaut}
\mcmd{\aut{E}}
\mcmd[\footnotemark]{\oldker}
\footnotetext{Correspond à la commande usuelle \Verb?\ker?}
\mcmd{\ker{f}}
\mmcmd[\footnotemark]{\la}{$\la\right.$}
\footnotetext{Correspond à la commande usuelle \Verb?\left\langle?}
\mmcmd[\footnotemark]{\ra}{$\left.\ra$}
\footnotetext{Correspond à la commande usuelle \Verb?\right\rangle?}
\mcmd{\oldord}
\mcmd{\ord{x}}
\section{Tables.sty}
Ce package sert à mettre en forme des tables an latex grâce à tikz.

Pour insérer une table, il faut appeler \Verb?\setrowcol{ncols}{nrows}? avec le nombre de colonnes et de lignes de la table, puis de rentrer la table tikz.

Une table a une largeur de 10cm et une hauteur de 6,5cm.

\vspace{0.5cm}

Par exemple, la table
\begin{center}
\setcolrow{6}{5}\begin{tikzpicture}[ampersand replacement=\&] \matrix[table] { \&$0$\&$\oldfrac{\pi}{6}$\&$\oldfrac{\pi}{4}$\&$\oldfrac{\pi}{3}$\&$\oldfrac{\pi}{2}$ \\ $\oldsin$\&$0$\&$\oldfrac{1}{2}$\&$\oldfrac{\sqrt{2}}{2}$\&$\oldfrac{\sqrt{3}}{2}$\&$1$\\ $\oldcos$\&$1$\&$\oldfrac{\sqrt{3}}{2}$\&$\oldfrac{\sqrt{2}}{2}$\&$\oldfrac{1}{2}$\&$0$\\ $\oldtan$\&$0$\&$\oldfrac{1}{\sqrt{3}}$\&$1$\&$\sqrt{3}$\&--\\ $\oldcot$\&--\&$\sqrt{3}$\&$1$\&$\oldfrac{1}{\sqrt{3}}$\&$0$\\ }; \draw [line width=0.5mm] (-10cm/3,-6.5cm/2) -- (-10cm/3,6.5cm/2); \draw [line width=0.5mm] (-10cm/2,3*6.5cm/10) -- (10cm/2,3*6.5cm/10); \draw [line width=0.5mm] (-5cm,-3.25cm) rectangle (5cm,3.25cm);\end{tikzpicture}
\end{center}
est produite avec le code suivant
\begin{Verbatim}[breaklines,breakafter=\&,breakaftersymbolpre=,breaksymbol=~~~~]
\setcolrow{6}{5}
\begin{tikzpicture}[ampersand replacement=\&]
    \matrix[table] {
        \&$0$\&$\oldfrac{\pi}{6}$\&$\oldfrac{\pi}{4}$\&$\oldfrac{\pi}{3}$\&$\oldfrac{\pi}{2}$\\
        $\oldsin$\&$0$\&$\oldfrac{1}{2}$\&$\oldfrac{\sqrt{2}}{2}$\&$\oldfrac{\sqrt{3}}{2}$\&$1$\\
        $\oldcos$\&$1$\&$\oldfrac{\sqrt{3}}{2}$\&$\oldfrac{\sqrt{2}}{2}$\&$\oldfrac{1}{2}$\&$0$\\
        $\oldtan$\&$0$\&$\oldfrac{1}{\sqrt{3}}$\&$1$\&$\sqrt{3}$\&--\\
        $\oldcot$\&--\&$\sqrt{3}$\&$1$\&$\oldfrac{1}{\sqrt{3}}$\&$0$\\
    };
    \draw [line width=0.5mm] (-10cm/3,-6.5cm/2) -- (-10cm/3,6.5cm/2);
    \draw [line width=0.5mm] (-10cm/2,3*6.5cm/10) -- (10cm/2,3*6.5cm/10);
    \draw [line width=0.5mm] (-5cm,-3.25cm) rectangle (5cm,3.25cm);
\end{tikzpicture}
\end{Verbatim}
\section{Trigo.sty}
\mhd
\mcmd[\footnotemark]{\oldcos}
\footnotetext{Correspond à la commande usuelle \Verb?\cos?}
\mcmd{\cos{x}}
\mcmd{\cos[n]{x}}
\mcmd[\footnotemark]{\oldsin}
\footnotetext{Correspond à la commande usuelle \Verb?\sin?}
\mcmd{\sin{x}}
\mcmd{\sin[n]{x}}
\mcmd[\footnotemark]{\oldtan}
\footnotetext{Correspond à la commande usuelle \Verb?\tan?}
\mcmd{\tan{x}}
\mcmd{\tan[n]{x}}
\mcmd[\footnotemark]{\oldcot}
\footnotetext{Correspond à la commande usuelle \Verb?\cot?}
\mcmd{\cot{x}}
\mcmd{\cot[n]{x}}
\mcmd{\acos{x}}
\mcmd{\acos[n]{x}}
\mcmd{\asin{x}}
\mcmd{\asin[n]{x}}
\mcmd{\atan{x}}
\mcmd{\atan[n]{x}}
\mcmd{\oldch}
\mcmd{\ch{x}}
\mcmd{\ch[n]{x}}
\mcmd{\oldsh}
\mcmd{\sh{x}}
\mcmd{\sh[n]{x}}
\mcmd{\oldth}
\mcmd{\th{x}}
\mcmd{\th[n]{x}}
\mcmd{\oldach}
\mcmd{\ach{x}}
\mcmd{\ach[n]{x}}
\mcmd{\oldash}
\mcmd{\ash{x}}
\mcmd{\ash[n]{x}}
\mcmd{\oldath}
\mcmd{\ath{x}}
\mcmd{\ath[n]{x}}
\mcmd[\footnotemark]{\oldIm}
\footnotetext{Correspond à la commande usuelle \Verb?\Im?}
\pagebreak
\hrule
\mcmd{\Im}
\mcmd{\pIm{x}}
\mcmd[\footnotemark]{\oldRe}
\footnotetext{Correspond à la commande usuelle \Verb?\Re?}
\mcmd{\Re}
\mcmd{\pRe{x}}
\end{document}
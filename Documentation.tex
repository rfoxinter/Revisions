\documentclass[a4paper,12pt]{article}
\usepackage[utf8]{inputenc}
\usepackage[french]{babel}
\usepackage[T1]{fontenc}
\usepackage[margin=2.5cm]{geometry}
\usepackage{hyperref}
\hypersetup{pdftitle=Documentation}
\let\oldoplus\oplus
\usepackage{output/htmlpreambule,output/matrices,output/al,output/bigoperators,output/analyse,output/arithmetique,output/complexes,output/dsft,output/equivalents,output/polynomes,output/probas,output/structures,output/trigo,output/tables}
\SetSymbolFont{stmry}{bold}{U}{stmry}{m}{n}
\usepackage{fvextra}
\setlength{\parindent}{0pt}
\hypersetup{hidelinks,colorlinks=true,linkcolor=black,linktoc=all,urlcolor=blue}
\setcounter{secnumdepth}{5}
\setcounter{tocdepth}{5}
\let\oldsection\section
\renewcommand{\section}{\clearpage\oldsection}
\title{Documentation}
\author{}
\date{}

\newcommand{\mhd}{
    \hrule\vline\hfill
    \begin{minipage}{0.475\linewidth}
        \begin{center}
            \vspace{2pt}
            \textbf{Commande}

            \vspace{2pt}
        \end{center}
    \end{minipage}
    \hfill\vline\hfill
    \begin{minipage}{0.475\linewidth}
        \begin{center}
            \vspace{2pt}
            \textbf{Résultat}
            
            \vspace{2pt}
        \end{center}
    \end{minipage}
    \hfill\vline\hrule
}
\newcommand{\mcmd}[2][]{
    \vline\hfill
    \begin{minipage}{0.475\linewidth}
        \begin{center}
            \vspace{2pt}
            \Verb?#2?#1

            \vspace{2pt}
        \end{center}
    \end{minipage}
    \hfill\vline\hfill
    \begin{minipage}{0.475\linewidth}
        \begin{center}
            \vspace{2pt}
            $#2$
            
            \vspace{2pt}
        \end{center}
    \end{minipage}
    \hfill\vline\hrule
}
\newcommand{\mmcmd}[3][]{
    \vline\hfill
    \begin{minipage}{0.475\linewidth}
        \begin{center}
            \vspace{2pt}
            \Verb?#2?#1

            \vspace{2pt}
        \end{center}
    \end{minipage}
    \hfill\vline\hfill
    \begin{minipage}{0.475\linewidth}
        \begin{center}
            \vspace{2pt}
            #3
            
            \vspace{2pt}
        \end{center}
    \end{minipage}
    \hfill\vline\hrule
}

\begin{document}
\pagenumbering{gobble}
\maketitle
Le package \texttt{Preambule.sty}\footnote{Pour utiliser avec \Verb?Beamer?} ou \texttt{HTMLPreambule.sty}\footnote{Pour les documents autres que \Verb?Beamer?} doit être chargé pour pouvoir utiliser les autres qui sont donnés ci-dessous.

Les fichiers \Verb?.sty? doivent être placés dans le même répertoire que le fichier \Verb?.tex? qui est utilisé.

Pour charger un package (par exemple \Verb?NomDuPackage.sty?), il faut utiliser la commande \Verb?\usepackage{nomdupackage}? avant \Verb?\begin{document}?.

\vspace{0.5cm}

En utilisant \texttt{Preambule.sty} ou \texttt{HTMLPreambule.sty}, les packages suivant seront chargés:
\begin{itemize}
    \renewcommand{\labelitemi}{$\to$}
    \item \Verb?\usepackage[utf8]{inputenc}?
    \item \Verb?\usepackage[french]{babel}?
    \item \Verb?\usepackage[T1]{fontenc}?
    \item \Verb?\usepackage{amsmath, amsfonts, amssymb}?
    \item \Verb?\usepackage{stmaryrd}?
    \item \Verb?\usepackage{adjustbox}? (pour \texttt{HTMLPreambule.sty})
    \item \Verb?\usepackage{xcolor}? (pour \texttt{Preambule.sty})
\end{itemize}

\vspace{0.5cm}
Il est nécessaire que \texttt{cm-super} soit installé (disponible sur \href{https://ctan.org/pkg/cm-super}{CTAN}) pour pouvoir utiliser \texttt{Preambule.sty}. Pour ne pas avoir à installer \texttt{cm-super}, il est possible de commenter les lignes \Verb?\usepackage{sffont}? et \Verb?\renewcommand{\sfdefault}{cmssp}? du fichier \texttt{Preambule.sty} en mettant \Verb?%? au début de chacune de ces lignes (numéro 10 et 11).

Lors de l'utilisation de \texttt{beamer} (avec une police sans-sérif), il est possible d'utiliser les commandes avec les polices sans-serif, sauf pour les lettres grecques ($\Omega$, $\oldphi$, $\phi$, \dots), la redéfinition du $l$ en mathématiques, les alphabets \Verb?\mathcal? et \Verb?\mathbb? ainsi que les symboles.

Il est possibles de changer les polices de caractères/symboles en important des packages après \Verb?\usepackage{preambule}?. Il peut être nécessaire de placer l'importation avant d'imorter les autres modules décrit ci-dessous.

Il n'est pas possible d'utiliser en simultané le package \Verb?Dsfont.sty? disponible sur \href{https://www.ctan.org/pkg/doublestroke}{CTAN} et \hyperlink{section.8}{\Verb?Dsft.sty? } décrit ci-dessous. De plus, la commande \Verb?\1? ne sera pas modifiée si un package définissant \Verb?\mathbb{1}? est importé. Il est alors possible de redéfinir la commande en utilisant \Verb?\newcommand\1[1]{\mathbb{1}_{#1}}? (si \texttt{Dsft.sty} n'est pas importé) ou \Verb?\renewcommand\1[1]{\mathbb{1}_{#1}}?.

\vspace{0.5cm}

Il est possible de redéfinir le $l$ à sa version d'origine avec:
\begin{Verbatim}
\mathcode`l="8000
\begingroup
\makeatletter
\lccode`\~=`\l
\DeclareMathSymbol{\lsb@l}{\mathalpha}{letters}{`l}
\lowercase{\gdef~{\lsb@l}}%
\endgroup
\makeatother
\end{Verbatim}

\vspace{0.5cm}

Pour utiliser des commandes avec des parenthèses automatiques (comme pour $\sup$), il est possible de faire :
\begin{Verbatim}
\let\oldsup\sup
\renewcommand{\sup}[1]{\oldsup\l#1\r}
\end{Verbatim}
\Verb?\l? et \Verb?\r? sont définis dans \hyperlink{section.2}{Preambule.sty et HTMLPreambule.sty}.
\tableofcontents
\null\newpage
\pagenumbering{arabic}
\section{Flashcards.py et Htmlcards.py}
Les fichiers \texttt{Flashcards.py} et \texttt{Htmlcards.py} permettent d'exporter facilement des flashcards en \texttt{.pdf} et \texttt{.svg} (pour affichage dans le navigateur).

Pour pouvoir créer des fiches de révision, il faut mettre un fichier \texttt{.txt} (décrit plus bas) dans un dossier \texttt{input} et mettre les fichiers \texttt{.sty} nécessaires dans un dossier \texttt{output}.
\subsection{Flashcards.py}
Pour exporter la fiche \texttt{fiche.txt}, il faut soit lancer le fichier python et entrer le nom du fichier (\texttt{fiche}), soit utiliser la commande \Verb?python Flashcards.py --file=fiche? (ou \texttt{python3}), à laquelle on peut rajouter les paramètres optionnels \Verb?--n=nombre? (avec le nombre d'exemplaires), \Verb?--dest=dossier? (avec le dossier où il faut mettre le \texttt{.pdf} produit) et \Verb?--open=True/False? (pour ouvrir le dossize où le \texttt{.pdf} est produit).
\subsection{Htmlcards.py}
Pour exporter la fiche \texttt{fiche.txt}, il faut soit lancer le fichier python et entrer le nom du fichier (\texttt{fiche}), soit utiliser la commande \Verb?python Flashcards.py --file=fiche? (ou \texttt{python3}), à laquelle on peut rajouter les paramètres optionnels \Verb?--dest=dossier? (avec le dossier où il faut mettre le \texttt{.pdf} produit) et \Verb?--open=True/False? (pour ouvrir le dossize où le \texttt{.pdf} est produit).
\subsection{Les fiches .txt}
Pour faire des fiches, il faut créer un fichier \texttt{.txt} de la forme
\begin{Verbatim}
TITRE
Shuffle questions : True/False
Q/R & R/Q : True/False
PACKAGES & COMMANDES SUPPLÉMENTAIRES
QUESTION;;RÉPONSE
...
QUESTION;;RÉPONSE
\end{Verbatim}

Le titre doit être de la forme \Verb?Thème -- Chapitre? ou \Verb?Chapitre?.
On peut aussi spécifier un titre racourci pour le nom du fichier avec \verb?Titre_raccourci!!ttleTitre classique? où le titre raccourci ne peut pas contenir d'espaces ou de caractères spéciaux, et titre classique étant de la forme des seux premiers.

La ligne 2 indique si on peut ou non mettre un ordre aléatoire pour les questions.

La ligne 3 indique si on peut échanger l'ordre des questions et des réponses pour les fiches. Avec cette option à \texttt{True}, il est possible de forcer une question à être avant la réponse en mettant \Verb?!!fst? devant la/les ligne(s) concernée(s).

Les packages et commandes supplémentaires doivent être placées sur une seule ligne (voir \href{https://www.overleaf.com/learn/latex/Commands}{CTAN}).

Il faut s'assurer qu'il n'y ait pas de ligne vide à la fin du fichier.

S'il y a une erreur lors de la compilation latex, le programme python affichera le message d'erreur affiché par latex.

\subsection{Visionner les flashcards en svg}
Pour pouvoir visionner les flashcards exportées en svg, il faut disposer d'un serveur web (comme \href{https://github.com/}{github} avec \href{https://pages.github.com/}{github pages}) sur lequel on met le dossier généré par \texttt{Htmlcards.py} (on suppose que l'url est \texttt{https://example.fr/dossier}).

Il faut alors convertir l'url du dossier en base 64 (cette conversion peut se faire sur le site \href{https://www.base64encode.org/}{\texttt{https://www.base64encode.org/}} ou avec la fonction \texttt{btoa} de JavaScript). Dans l'exemple, on obtient \Verb?aHR0cHM6Ly9leGFtcGxlLmZyL2Rvc3NpZXI=?.

Il faut alors aller sur le site \href{https://rfoxinter.github.io/revisions/flashcards/?file=}{\texttt{https://rfoxinter.github.io/revisions/flashcards/}} en rajoutant à la fin de l'url \texttt{?file=nom\_du\_dossier} où le nom du dossier correspond à celui en base 64.

Dans l'exemple, on obtient l'url suivante:\\\texttt{https://rfoxinter.github.io/revisions/flashcards/}

\texttt{~~~~?file=aHR0cHM6Ly9leGFtcGxlLmZyL2Rvc3NpZXI=}
\section{Preambule.sty et HTMLPreambule.sty}
\mhd
\mmcmd[\footnotemark]{\l}{$\l\right.$}
\footnotetext{Correspond à la commande usuelle \Verb?\left(?}
\mmcmd[\footnotemark]{\r}{$\left.\r$}
\footnotetext{Correspond à la commande usuelle \Verb?\right)?}
\mmcmd[\footnotemark]{\llb}{$\llb\right.$}
\footnotetext{Correspond à la commande usuelle \Verb?\left\llbracket?}
\mmcmd[\footnotemark]{\rrb}{$\left.\rrb$}
\footnotetext{Correspond à la commande usuelle \Verb?\right\rrbracket?}
\mcmd[\footnotemark]{\oldfrac{a}{b}}
\footnotetext{Correspond à la commande usuelle \Verb?\frac?}
\mcmd[\footnotemark]{\frac{a}{b}}
\footnotetext{Correspond à la commande usuelle \Verb?\dfrac?}
\mcmd[\footnotemark]{l}
\footnotetext{Correspond à la commande usuelle \Verb?\ell?}
\section{AL.sty}
Le package \hyperlink{section.10}{\Verb?Matrices.sty?} sera importé automatiquement avec \Verb?AL.sty?.

\vspace{0.5cm}

\mhd
\mcmd{\oldvect}
\mcmd{\vect{E}}
\mcmd{\al{E}{}}
\mcmd{\al{E}{F}}
\mcmd[\footnotemark]{\oplus}
\footnotetext{Le \Verb?\oplus? utilisé est celui de \Verb?stmaryrd?\\Pour récupérer celui de \LaTeX, il est possible d'utiliser la commande \Verb?\let\oldoplus\oplus? avant \Verb?\usepackage{al}? puis de faire \Verb?\let\oplus\oldoplus? après importation\\Comparaison \LaTeX~- \Verb?stmaryrd? avec le plus normal $\oldoplus\oplus+$}
\mcmd[\footnotemark]{\matgl{n}{\mathbb{K}}}
\footnotetext{Le \Verb?\matgl? de \Verb?AL.sty? correspond à la commande \Verb?\gl? de \Verb?Matrices.sty? qui a été renommé}
\mcmd{\gl{E}}
\mcmd[\footnotemark]{\olddim}
\footnotetext{Correspond à la commande usuelle \Verb?\dim?}
\mcmd{\dim{E}}
\mcmd{\oldrg}
\mcmd{\rg{u}}
\mcmd{\oldtr}
\mcmd{\tr{u}}
\mcmd{\oldmat}
\mcmd{\mat{\mathcal{B}}{}{u}}
\mcmd{\mat{\mathcal{B}}{\mathcal{C}}{u}}
\mmcmd{\lc}{$\lc\right.$}
\mmcmd{\rc}{$\left.\rc$}
\section{Analyse.sty}
Le package \hyperlink{section.6}{\Verb?BigOperators.sty?} sera importé automatiquement avec \Verb?Analyse.sty?.

\vspace{0.5cm}

\mhd
\mcmd[\footnotemark]{\oldd}
\footnotetext{$\oldd$ de dérivation}
\mcmd{\der{f(x)}}
\mcmd{\der[n]{f(x)}}
\mcmd{\der[][t]{f(t)}}
\mcmd[\footnotemark]{\oldint}
\footnotetext{Correspond à la commande usuelle \Verb?\int?}
\mcmd{\int{f}}
\mcmd{\int[t]{f(t)}}
\mcmd[\footnotemark]{\int[t][{[a,b]}]{f(t)}}
\footnotetext{L'argument \Verb?[a,b]? doit être mis entre accolades pour être traîté correctement par \LaTeX}
\mcmd{\int[t][a][b]{f(t)}}
\mcmd{\eval[{[a,b]}]{f(t)}}
\mcmd{\eval[a][b]{f(t)}}
\mcmd{\serie{a_n}}
\section{Arithmetique.sty}
\mhd
\mcmd[\footnotemark]{\olddiv}
\footnotetext{Correspond à la commande usuelle \Verb?\div?}
\mcmd[\footnotemark]{\div}
\footnotetext{Correspond à la commande usuelle \Verb?\mid?}
\mcmd{\cgr{a}{b}{n}}
\mcmd[\footnotemark]{\oldphi}
\footnotetext{Correspond à la commande usuelle \Verb?\phi?}
\mcmd[\footnotemark]{\phi}
\footnotetext{Correspond à la commande usuelle \Verb?\varphi?}
\section{BigOperators.sty}
\mhd
\mcmd[\footnotemark]{\oldsum}
\footnotetext{Correspond à la commande usuelle \Verb?\sum?}
\mcmd{\sum{n=0}{+\infty}{u_n}}
\mcmd[\footnotemark]{\oldprod}
\footnotetext{Correspond à la commande usuelle \Verb?\prod?}
\mcmd{\prod{n=0}{+\infty}{u_n}}
\mcmd[\footnotemark]{\oldcap}
\footnotetext{Correspond à la commande usuelle \Verb?\bigcap?}
\mcmd{\bigcap{n=0}{+\infty}{A_n}}
\mcmd[\footnotemark]{\oldcup}
\footnotetext{Correspond à la commande usuelle \Verb?\bigcup?}
\mcmd{\bigcup{n=0}{+\infty}{A_n}}
\mcmd[\footnotemark]{\olduplus}
\footnotetext{Correspond à la commande usuelle \Verb?\biguplus?}
\mcmd{\biguplus{n=0}{+\infty}{A_n}}
\mcmd{\bigop{n=0}{+\infty}{E_n}}
\section{Complexes.sty}
\mhd
\mcmd[\footnotemark]{\oldbar{z}}
\footnotetext{Correspond à la commande usuelle \Verb?\bar?}
\mcmd[\footnotemark]{\bar{z}}
\footnotetext{Se comporte comme \Verb?\overline?}
\mcmd[\footnotemark]{\e}
\footnotetext{$\e$ de la fonction exponentielle}
\mcmd[\footnotemark]{\i}
\footnotetext{$\i$ complexe\\L'ancienne commande \Verb?\i? s'obtient avec \Verb?\ii?}
\mcmd[\footnotemark]{\j}
\footnotetext{$\j=\e^{\oldfrac{2\i\pi}{3}}$\\L'ancienne commande \Verb?\j? s'obtient avec \Verb?\jj?}
\mcmd[\footnotemark]{\oldIm}
\footnotetext{Correspond à la commande usuelle \Verb?\Im?}
\mcmd{\Im}
\mcmd{\pIm{x}}
\mcmd[\footnotemark]{\oldRe}
\footnotetext{Correspond à la commande usuelle \Verb?\Re?}
\mcmd{\Re}
\mcmd{\pRe{x}}
\section{Dsft.sty}
Ce package remplace le $\mathds 1$ du package \Verb?Dsfonts.sty? disponible sur \href{https://www.ctan.org/pkg/doublestroke}{CTAN}.

Pour l'utiliser, il faut copier les fichiers \Verb?dsrom12.pfb? et \Verb?dsrom12.tfm? dans les dossiers où ils sont actuellement avec \Verb?dsfonts? (et éventuellement créer une copie des anciens fichiers).

\vspace{0.5cm}

\mhd
\mcmd{\mathds{1}}
\mcmd{\1{E}(x)}
\mcmd{\square}
\mcmd{\star}
\mcmd{\triangle}
\section{Equivalents.sty}
\mhd
\mcmd{\o{x}}
\mcmd{\o[x\to0]{x}}
\mcmd{\O{x}}
\mcmd{\O[x\to0]{x}}
\mcmd{\Th{x}}
\mcmd{\Th[x\to0]{x}}
\mcmd{\Om{x}}
\mcmd{\Om[x\to0]{x}}
\mcmd{\eq{u_n}{v_n}}
\mcmd{\eq[n\to+\infty]{u_n}{v_n}}
\mcmd{\eg{u_n}{v_n+\o{v_n}}}
\mcmd{\eg[n\to+\infty]{u_n}{v_n+\o{v_n}}}
\section{Matrices.sty}
\mhd
\mcmd{\mat{n}{p}{\mathbb{K}}}
\mcmd{\mat{n}{}{\mathbb{K}}}
\mcmd{\sym{n}{\mathbb{K}}}
\mcmd{\ant{n}{\mathbb{K}}}
\mcmd{\diag{n}{\mathbb{K}}}
\mcmd{\ts{n}{\mathbb{K}}}
\mcmd{\ti{n}{\mathbb{K}}}
\mcmd[\footnotemark]{\olddet}
\footnotetext{Correspond à la commande usuelle \Verb?\det?}
\mcmd{\det{M}}
\mmcmd[\footnotemark]{\gl{n}{\mathbb{K}}}{$\matgl{n}{\mathbb{K}}$}
\footnotetext{Si \Verb?AL.sty? est chargé, cette commande est remplacée et il faut utiliser \Verb?\matgl{n}{\mathbb{K}}? pour obtenir ce résultat}
\mmcmd{\mdots}{$\oldmdots$}
\mmcmd{\ddots}{$\oldddots$}
\mmcmd{\idots}{$\oldidots$}
\mmcmd{\vdots}{$\oldvdots$}
\mmcmd{\xdots}{$\oldxdots$}
\mcmd[\footnotemark]{\tmatrix({1\&0\\0\&1\\})}
\footnotetext{Les caractères \Verb?\&? sont utilisés au lieu du \Verb?&? utilisé habituellement avec tikz pour des raisons de compatibilité avec \texttt{Beamer}}

\vspace{0.5cm}

\textbf{\boldmath\Large Les commandes avec $\mcdot$}

Les commandes avec des points tel que $\xdots$ ont des définitions qui dépendent de la taille de la police (celle pour \LaTeX est adaptée pour 12pt, et celle de \texttt{Beamer} pour 17pt). Pour avoir des points alignés correctement, il est possible de modifier la valeur de \Verb?\dotsep? en utilisant \Verb?\setlength{\dotsep}{taille_en_pt}?.

Par exemple, avec 2pt, on obtient : \setlength{\dotsep}{2pt} $\xdots$.\setlength{\dotsep}{3.5pt}

\vspace{0.5cm}

\textbf{\Large La commande \Verb?\tmatrix?}

\Verb?\tmatrix? est composé de deux arguments optionnels (les éléments à ajouter à la matrice tikz et les éléments de mise en page de la matrice) ainsi que de trois arguments (le délimiteur d'ouverture, le contenu de la matrice et le délimiteur de fermeture).

Les commandes sont:

\vspace{0.5cm}

\mhd
\mmcmd{\mtxvline{params}{n}}{Crée une ligne verticale après la colonne \texttt{n} (ou left/right pour les extrémités) avec les paramètres tikz \texttt{params}}
\mmcmd{\mtxhline{params}{n}}{Crée une ligne horizontale après la ligne \texttt{n} (ou top/bottom pour les extrémités) avec les paramètres tikz \texttt{params}}
\pagebreak
\hrule
\mmcmd{\mtxvpartial{params}{n}{a}{b}}{Crée une ligne verticale après la colonne \texttt{n} (ou left/right pour les extrémités), la ligne ayant pour extrémités la fin de la ligne \texttt{a} et \texttt{b} (ou top/bottom) avec les paramètres tikz \texttt{params}}
\mmcmd{\mtxhpartial{params}{n}{a}{b}}{Crée une ligne horizontale après la ligne \texttt{n} (ou top/bottom pour les extrémités), la ligne ayant pour extrémités la fin de la ligne \texttt{a} et \texttt{b} (ou left/right) avec les paramètres tikz \texttt{params}}
\mmcmd{\mtxbox{params}{x}{y}}{Crée une boîte autour de la case de coordonnées \texttt{x} et \texttt{y} (l'indexation commence à 1) avec les paramètres tikz \texttt{params}}
\vspace{0.5cm}

\textbf{\Large Exemples avec \Verb?\tmatrix?}

$\det M=\tmatrix|{a\&b\\c\&d\\}|$ est produit par \Verb?$\det{M}=\tmatrix|{a\&b\\c\&d\\}|$?.

\vspace{0.25cm}

$I_{n,p,r}=\tmatrix[\mtxvline{line width = 0.05em}{1}\mtxhline{line width = 0.05em}{1}][minimum height = 5ex, row sep = 1ex,minimum width = 5ex, column sep = 1ex,]({I_r\&0_{r,p-r}\\0_{n-r,r}\&0_{n-r,p-r}\\})$ est produit par
\begin{Verbatim}
$
I_{n,p,r}=\tmatrix
    [\mtxvline{line width = 0.05em}{1}\mtxhline{line width = 0.05em}{1}]
    [minimum height = 5ex, row sep = 1ex, minimum width = 5ex,
        column sep = 1ex,]
    ({I_r\&0_{r,p-r}\\0_{n-r,r}\&0_{n-r,p-r}\\})
$
\end{Verbatim}

\vspace{0.25cm}

$\tmatrix[\mtxbox{red, dashed}{1}{1}\mtxbox{teal, dotted, ultra thick}{2}{2}\mtxbox{}{4}{4}][minimum height = 5ex, minimum width = 5ex, row sep = 10pt,inner sep = 5pt, column sep = 10pt,]{{[}}{A_1\&0\&0\&0\\0\&A_2\&\ddots\&0\\0\&\ddots\&\ddots\&0\\0\&0\&0\&A_n\\}{\}}$ est produit par
\begin{Verbatim}[breaklines,breakafter=\&,breakaftersymbolpre=,breaksymbol=~~~~,curlyquotes]
$
\tmatrix
    [\mtxbox{red, dashed}{1}{1}\mtxbox{teal, dotted,
        ultra thick}{2}{2}\mtxbox{}{4}{4}]
    [minimum height = 5ex, minimum width = 5ex, row sep = 10pt,
        inner sep = 5pt, column sep = 10pt,]
    {{[}} % Le crochet est entouré de deux paires d'accolades
    {A_1\&0\&0\&0\\0\&A_2\&\ddots\&0\\0\&\ddots\&\ddots\&0\\
        0\&0\&0\&A_n\\}
    {\}}
$
\end{Verbatim}
\section{Polynomes.sty}
\mhd
\mcmd{\pol{K}{X}}
\mcmd{\fr{K}{X}}
\mcmd[\footnotemark]{\olddeg}
\footnotetext{Correspond à la commande usuelle \Verb?\deg?}
\mcmd{\deg{P}}
\mcmd{\oldval}
\mcmd{\val{P}}
\mcmd{\oldcar}
\mcmd{\car{\mathbb{K}}}
\section{Probas.sty}
\mhd
\mcmd{\p{A}}
\mcmd{\p[B]{A}}
\mcmd[\footnotemark]{\oldOmega}
\footnotetext{Correspond à la commande usuelle \Verb?\Omega?}
\mcmd[\footnotemark]{\Omega}
\footnotetext{Correspond à la commande usuelle \Verb?\varOmega?}
\mmcmd[\footnotemark]{\sq}{$\left.\sq\right.$}
\footnotetext{Doit être utilisé entre \Verb?\left? et \Verb?\right?, ou dans la commande \Verb?\p?: $\p{A\sq\bigcap{k=1}{n}{B_i}}$}
\mcmd[\footnotemark]{\bor}
\footnotetext{Correspond à la commande usuelle \Verb?\mathcal{B}?}
\mcmd{\esp{X}}
\mcmd{\var{X}}
\mcmd{\ect{X}}
\mcmd{\oldcov}
\mcmd{\cov{X}{Y}}
\mcmd[\footnotemark]{\indep}
\footnotetext{Ce symbole est obtenu avec la commande \Verb?\perp\!\!\!\perp?}
\section{Structures.sty}
\mhd
\mcmd{\oldhom}
\mcmd{\hom{E}}
\mcmd{\oldaut}
\mcmd{\aut{E}}
\mcmd[\footnotemark]{\oldker}
\footnotetext{Correspond à la commande usuelle \Verb?\ker?}
\mcmd{\ker{f}}
\mcmd{\oldim}
\mcmd{\im{f}}
\mmcmd[\footnotemark]{\la}{$\la\right.$}
\footnotetext{Correspond à la commande usuelle \Verb?\left\langle?}
\mmcmd[\footnotemark]{\ra}{$\left.\ra$}
\footnotetext{Correspond à la commande usuelle \Verb?\right\rangle?}
\mcmd{\oldord}
\mcmd{\ord{x}}
\section{Tables.sty}
Ce package sert à mettre en forme des tables an latex grâce à tikz.

Pour insérer une table, il faut appeler \Verb?\setrowcol[width][height]{ncols}{nrows}? avec le nombre de colonnes et de lignes de la table, puis rentrer la table tikz, les arguments optionnels étant la largeur de la table et sa hauteur.

Une table a une largeur par défaut de 10cm et une hauteur de 6,5cm (est est réinitialisée à chaque appel de \Verb?\setrowcol?).

Il est possible d'utiliser \Verb?[ampersand replacement=\&]? puis \Verb?\&? pour la matrice lorsque \Verb?&? est déjà défini par \LaTeX.

Il est possible de récupérer la valeur de la largeur et de la hauteur avec \Verb?\tblw? et \Verb?\tblh?.

\vspace{0.5cm}

Par exemple, la table
\begin{center}
\setcolrow[\linewidth]{6}{5}\begin{tikzpicture}\matrix[table] { &$0$&$\oldfrac{\pi}{6}$&$\oldfrac{\pi}{4}$&$\oldfrac{\pi}{3}$&$\oldfrac{\pi}{2}$ \\ $\oldsin$&$0$&$\oldfrac{1}{2}$&$\oldfrac{\sqrt{2}}{2}$&$\oldfrac{\sqrt{3}}{2}$&$1$\\ $\oldcos$&$1$&$\oldfrac{\sqrt{3}}{2}$&$\oldfrac{\sqrt{2}}{2}$&$\oldfrac{1}{2}$&$0$\\ $\oldtan$&$0$&$\oldfrac{1}{\sqrt{3}}$&$1$&$\sqrt{3}$&--\\ $\oldcot$&--&$\sqrt{3}$&$1$&$\oldfrac{1}{\sqrt{3}}$&$0$\\ }; \draw [line width=0.5mm] (-\tblw/3,-\tblh/2) -- (-\tblw/3,\tblh/2); \draw [line width=0.5mm] (-\tblw/2,3*\tblh/10) -- (\tblw/2,3*\tblh/10); \draw [line width=0.5mm] (-\tblw/2,-\tblh/2) rectangle (\tblw/2,\tblh/2);\end{tikzpicture}
\end{center}
est produite avec le code suivant
\begin{Verbatim}[breaklines,breakafter=&,breakaftersymbolpre=,breaksymbol=~~~~]
\setcolrow{6}{5}
\begin{tikzpicture}
    \matrix[table] {
        &$0$&$\oldfrac{\pi}{6}$&$\oldfrac{\pi}{4}$&$\oldfrac{\pi}{3}$&$\oldfrac{\pi}{2}$\\
        $\oldsin$&$0$&$\oldfrac{1}{2}$&$\oldfrac{\sqrt{2}}{2}$&$\oldfrac{\sqrt{3}}{2}$&$1$\\
        $\oldcos$&$1$&$\oldfrac{\sqrt{3}}{2}$&$\oldfrac{\sqrt{2}}{2}$&$\oldfrac{1}{2}$&$0$\\
        $\oldtan$&$0$&$\oldfrac{1}{\sqrt{3}}$&$1$&$\sqrt{3}$&--\\
        $\oldcot$&--&$\sqrt{3}$&$1$&$\oldfrac{1}{\sqrt{3}}$&$0$\\
    };
    \draw [line width=0.5mm] (-\tblw/3,-\tblh/2) -- (-\tblw/3,\tblh/2);
    \draw [line width=0.5mm] (-\tblw/2,3*\tblh/10) -- (\tblw/2,3*\tblh/10);
    \draw [line width=0.5mm] (-\tblw/2,-\tblh/2) rectangle (\tblw/2,\tblh/2);
\end{tikzpicture}
\end{Verbatim}
\section{Trigo.sty}
\mhd
\mcmd[\footnotemark]{\oldcos}
\footnotetext{Correspond à la commande usuelle \Verb?\cos?}
\mcmd{\cos{x}}
\mcmd{\cos[n]{x}}
\mcmd[\footnotemark]{\oldsin}
\footnotetext{Correspond à la commande usuelle \Verb?\sin?}
\mcmd{\sin{x}}
\mcmd{\sin[n]{x}}
\mcmd[\footnotemark]{\oldtan}
\footnotetext{Correspond à la commande usuelle \Verb?\tan?}
\mcmd{\tan{x}}
\mcmd{\tan[n]{x}}
\mcmd[\footnotemark]{\oldcot}
\footnotetext{Correspond à la commande usuelle \Verb?\cot?}
\mcmd{\cot{x}}
\mcmd{\cot[n]{x}}
\mcmd{\acos{x}}
\mcmd{\acos[n]{x}}
\mcmd{\asin{x}}
\mcmd{\asin[n]{x}}
\mcmd{\atan{x}}
\mcmd{\atan[n]{x}}
\mcmd{\oldch}
\mcmd{\ch{x}}
\mcmd{\ch[n]{x}}
\mcmd{\oldsh}
\mcmd{\sh{x}}
\mcmd{\sh[n]{x}}
\mcmd{\oldth}
\mcmd{\th{x}}
\mcmd{\th[n]{x}}
\mcmd{\oldach}
\mcmd{\ach{x}}
\mcmd{\ach[n]{x}}
\mcmd{\oldash}
\mcmd{\ash{x}}
\mcmd{\ash[n]{x}}
\mcmd{\oldath}
\mcmd{\ath{x}}
\mcmd{\ath[n]{x}}
\end{document}

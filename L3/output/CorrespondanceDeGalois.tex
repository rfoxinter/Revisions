\documentclass[14pt,usepdftitle=false,aspectratio=169]{beamer}
\usepackage{preambule}
\setbeamercolor{structure}{fg=black}
\let\tilde\widetilde\DeclareMathOperator{\oldgal}{Gal}\newcommand{\gal}[1]{\oldgal\l#1\r}\DeclareMathOperator{\oldrev}{Rev}\newcommand{\rev}[2]{\oldrev\l#1,#2\r}\let\Pi\varPi\newcommand{\appl}[5]{\begin{array}[t]{@{}r@{}r@{}c@{}l@{}}#1\!:\!{}&#2&{}\longrightarrow{}&#3\\&#4&{}\longmapsto{}&#5\end{array}}\let\phi\varphi
\hypersetup{pdftitle=Groupe fondamental et revêtement -- Correspondance de Galois}
\title{Groupe fondamental et revêtement\\\emph{Correspondance de Galois}}
\author{}
\date{}
\begin{document}
\begin{frame}
    \titlepage
\end{frame}
\slideq{Lien entre $\Pi_1^\text{rev}\l B,b\r$, $\tilde B$ et $B$}{1/3}
\slider{L'action $\gal p=\Pi_1^\text{rev}\l B,b\r\curvearrowright\tilde B$ est libre et $\Pi_1^\text{rev}\l B,b\r\backslash\tilde B\cong B$}{1/3}
\slideq{Correspondance de Galois\linebreak Par les treillis}{2/3}
\slider{Les treillis de $\rev Bb$ et de $\Delta\l\Pi_1^\text{rev}\l B,b\r\r$ sont isomorphes\linebreak$\Delta\l G\r$ désigne les sous groupes de $G$}{2/3}
\slideq{Correspondance de Galois avec les morphismes}{3/3}
\slider{$\appl{\phi}{\Delta\l\Pi_1^\text{rev}\l B,b\r\r}{\rev{B}{b}}{\varGamma}{\varGamma\backslash B}$ et $\appl{\psi}{\rev{B}{b}}{\Delta\l\Pi_1^\text{rev}\l B,b\r\r}{\l Y,y\r}{\Pi_1^\text{rev}\l Y,y\r}$ sont deux bijections décroissantes réciproques\linebreak$\left[\Pi_1^\text{rev}\l B,b\r:\varGamma\right]=d$ ssi $\phi\l\varGamma\r$ est de degré $d$\linebreak$\varGamma\vartriangleleft\Pi_1^\text{rev}\l B,b\r$ ssi $\phi\l\varGamma\r$ est galoisien}{3/3}
\end{document}
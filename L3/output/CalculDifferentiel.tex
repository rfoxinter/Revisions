\documentclass[14pt,usepdftitle=false,aspectratio=169]{beamer}
\usepackage{preambule}
\setbeamercolor{structure}{fg=black}
\usepackage{al,analyse,topologie,equivalents}
\hypersetup{pdftitle=Topologie et calcul différentiel -- Calcul différentiel}
\title{Topologie et calcul différentiel\\\emph{Calcul différentiel}}
\author{}
\date{}
\begin{document}
\begin{frame}
    \titlepage
\end{frame}
\slideq{$\al EF$}{1/5}
\slider{Applications linéaires continues de $E$ dans $F$}{1/5}
\slideq{$R\l\left[a,b\right],F\r$}{2/5}
\slider{Adhérence des fonctions en escalier sur $\left[a,b\right]$\linebreak$\l R\l\left[a,b\right],F\r,\anrm[{\left[a,b\right]}]{\cdot}\r$ est un espace de Banach}{2/5}
\slideq{$f\in\al EF$\linebreak$\dd f_a$}{3/5}
\slider{$f$}{3/5}
\slideq{$f$ est différentiable en $a\in U$}{4/5}
\slider{$\exists g\in\al EF$\linebreak$f\l a+h\r=f\l a\r+g\l h\r+\o[\nrm h\to0]{\nrm{h}}$}{4/5}
\slideq{Gradient de $f$ en $a$}{5/5}
\slider{L'unique vecteur $\nabla f\l a\r$ tel que $\dd f_a\l h\r=\psc{\nabla f\l a\r}{h}$}{5/5}
\end{document}
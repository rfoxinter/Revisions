\documentclass[14pt,usepdftitle=false,aspectratio=169]{beamer}
\usepackage{preambule}
\setbeamercolor{structure}{fg=black}
\usepackage{bigoperators}
\hypersetup{pdftitle=Intégration et théorie de la mesure -- Théorie de la mesure}
\title{Intégration et théorie de la mesure\\\emph{Théorie de la mesure}}
\author{}
\date{}
\begin{document}
\begin{frame}
    \titlepage
\end{frame}
\slideq{Espace mesuré}{1/11}
\slider{$\l X,\mathcal A,\mu\r$}{1/11}
\slideq{Mesure (positive)}{2/11}
\slider{$\mu\!:\!\mathcal A\to\mathbb R_+$ est une mesure si $\mu\l\varnothing\r=0$ et pour tout $\l A_n\r\in\mathcal A^{\mathbb N}$ vérifiant $i\neq j\Rightarrow A_i\cap A_j=\varnothing$ alors $\mu\l\bigcup{n\in\mathbb N}{}{A_n}\r=\sum{n\in\mathbb N}{}{\mu\l A_n\r}$}{2/11}
\slideq{Fonction mesurable}{3/11}
\slider{$f\!:\!\l X,\mathcal A\r\to\l Y,\mathcal B\r$ est mesurable si $f^{-1}\l\mathcal B\r\subset\mathcal A$}{3/11}
\slideq{Tribu borélienne}{4/11}
\slider{Tribu engendrée par les ouverts}{4/11}
\slideq{Mesure $\sigma$-finie}{5/11}
\slider{$\exists\l X_n\r\in\mathcal A^{\mathbb N}$, $X=\bigcup{n\in\mathbb N}{}{A_n}$ et $\forall n\in\mathbb N$, $\mu\l A_n\r<+\infty$}{5/11}
\slideq{$\sigma$-algèbre (ou tribu)}{6/11}
\slider{$\mathcal A$ est une $\sigma$-algèbre si $\varnothing\in\mathcal A$, $\forall A\in\mathcal A$, $A^\complement\in\mathcal A$ et $\forall\l A_n\r\in\mathcal A^{\mathbb N}$, $\bigcup{n\in\mathbb N}{}{A_n}\in\mathcal A$}{6/11}
\slideq{Espace mesurable}{7/11}
\slider{$\l X,\mathcal A\r$}{7/11}
\slideq{Intersection de tribus}{8/11}
\slider{Toute intersection de tribus est une tribu}{8/11}
\slideq{$\sigma\l C\r$}{9/11}
\slider{$\bigcap{\substack{\mathcal A\text{ tribu}\\C\subset\mathcal A}}{}{\mathcal A}$}{9/11}
\slideq{Algèbre (de Boole)}{10/11}
\slider{$\mathcal A$ est une algèbre si $\varnothing\in\mathcal A$, $\forall A\in\mathcal A$, $A^\complement\in\mathcal A$ et $\forall\l A,B\r\in\mathcal A^2$, $A\cup B\in\mathcal A$}{10/11}
\slideq{Mesure finie}{11/11}
\slider{$\mu\l X\r<+\infty$}{11/11}
\end{document}
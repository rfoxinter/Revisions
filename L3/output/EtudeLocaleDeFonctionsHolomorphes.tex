\documentclass[14pt,usepdftitle=false,aspectratio=169]{beamer}
\usepackage{preambule}
\setbeamercolor{structure}{fg=black}
\let\phi\varphi
\hypersetup{pdftitle=Analyse complexe -- Étude locale de fonctions holomorphes}
\title{Analyse complexe\\\emph{Étude locale de fonctions holomorphes}}
\author{}
\date{}
\begin{document}
\begin{frame}
    \titlepage
\end{frame}
\slideq{<<~Expression locale~>> d'une fonction $f$ holomorphe sur $U$}{1/3}
\slider{Pour $z_0\in U$, et $m=v_{z_0}\l f-f\l z_0\r\r$ il existe un voisinnage ouvert $V$ de $z_0$ dans $U$ et $r>0$ et un biholomorphisme $\phi\!:\!V\to D\l f\l z_0\r,r\r$ tels que $f\l z\r=f\l z_0\r+\phi\l z\r^m$}{1/3}
\slideq{Théorème de l'image ouverte}{2/3}
\slider{Si $f\!:\!U\to\mathbb C$ est holomorphe non constante et $U$ est un ouvert connexe de $\mathbb C$ alors $f$ est ouverte (ie pour tout $V$ ouvert de $U$, $f\l V\r$ est un ouvert de $\mathbb C$)}{2/3}
\slideq{Théorème d'inversion globale}{3/3}
\slider{Si $f\!:\!U\to\mathbb C$ est holomorphe et injective alors $f^{|f\l U\r}\!:\!U\to f\l U\r$ est un biholomorphisme}{3/3}
\end{document}
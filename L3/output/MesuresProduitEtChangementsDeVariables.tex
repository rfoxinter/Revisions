\documentclass[14pt,usepdftitle=false,aspectratio=169]{beamer}
\usepackage{preambule}
\setbeamercolor{structure}{fg=black}
\usepackage{analyse}\togglebigoppar\toggleanalysepar\let\otimes\varotimes\newcommand{\appl}[5]{\begin{array}[t]{@{}r@{}r@{}c@{}l@{}}#1\!:\!{}&#2&{}\longrightarrow{}&#3\\&#4&{}\longmapsto{}&#5\end{array}}\makeatletter\def\intkern@{\mkern-9mu\mathchoice{\mkern-6mu}{}{}{}}\makeatother\newcommand{\lnrm}[2]{\anrm[L^#1]{#2}}\let\phi\varphi
\hypersetup{pdftitle=Intégration et théorie de la mesure -- Mesures produit et changements de variables}
\title{Intégration et théorie de la mesure\\\emph{Mesures produit et changements de variables}}
\author{}
\date{}
\begin{document}
\begin{frame}
    \titlepage
\end{frame}
\slideq{Changement de variable polaire en dimension $2$}{1/14}
\slider{$\altint\iint[x,y][\mathbb R^2]{f\l x,y\r}=\altint\iint[][{\mathbb R_+\times\left]-\pi,\pi\right[}]{\widetilde f\l r,\theta\r{{}\mathop{r}}\dd r\dd\theta}$}{1/14}
\slideq{Porpriétés de $\l f*g\r\l x\r$ pour $f\in L^p\l\mathbb R^d\r$, $g\in L^{p'}\l\mathbb R^d\r$, $p'=\frac{p}{p-1}$}{2/14}
\slider{$f*g\in L^\infty\l\mathbb R^d\r$ et $\lnrm\infty{f*g}\leqslant\lnrm1f\lnrm1g$\linebreak Si $p\in\left]1,+\infty\right[$ alors $f*g$ est uniformément continue}{2/14}
\slideq{Porpriétés de $\phi_n*f$ où $\l\phi_n\r$ est une approximation de l'unité}{3/14}
\slider{Si $f\in\mathcal C^0_b\l\mathbb R^d\r$ alors $\phi_n*f\xrightarrow[n\to+\infty]{\text{CVU}}f$\linebreak Si $f\in L^p\l\mathbb R^d\r$ alors $\phi_n*f\xrightarrow[n\to+\infty]{L^p}f$}{3/14}
\slideq{$\mathcal B\l X\times Y\r$}{4/14}
\slider{$\mathcal B\l X\r\otimes\mathcal{B}\l Y\r$ si $X$ et $Y$ sont à base dénombrable d'ouverts}{4/14}
\slideq{Théorème de Fubini-Tonelli}{5/14}
\slider{\toggleanalysedisplay Si $f\!:\!X\times Y\to\overline{\mathbb R_+}$ est mesurable pour $\mathcal A\otimes\mathcal B$ alors $\appl{f}{X}{\overline{\mathbb R_+}}{x}{\int[y][Y][][\nu]{f\l x,y\r}}$ et $\appl{f}{Y}{\overline{\mathbb R_+}}{y}{\int[x][X][][\mu]{f\l x,y\r}}$ sont mesurables et $\altint{\iint}[x,y][X\times Y][][\mu\otimes\nu]{f\l x,y\r}$\linebreak$=\int[x][X][][\mu]{\l\int[y][Y][][\nu]{f\l x,y\r}\r}$\linebreak$=\int[y][Y][][\nu]{\l\int[x][X][][\mu]{f\l x,y\r}\r}$\toggleanalysedisplay}{5/14}
\slideq{Porpriétés de $\l f*g\r\l x\r$ pour $f\in L^1\l\mathbb R^d\r$, $g\in\mathcal C^1_b\l\mathbb R^d\r:=\left\{g\in\mathcal C^1\l\mathbb R^d\r,g\in L^\infty,\nabla g\in L^\infty\right\}$}{6/14}
\slider{$f*g\in\mathcal C^1_b\l\mathbb R^d\r$ et $\nabla\l f*g\r=f*\l\nabla f\r$}{6/14}
\slideq{Changement de variable polaire en dimension $>2$}{7/14}
\slider{$\int[x][\mathbb R^d]{f\l x\r}=\altint\iint[][{\mathbb R_+\times\mathbb S^{d-1}}]{f\l ru\r{{}\mathop{r^{d-1}}}\dd r\sigma\l\dd u\r}$\linebreak$\sigma\l A\r=d\cdot\lambda\l\left\{ru,r\in\left[0,1\right],u\in A\right\}\r$, $A\subset\mathbb S^{d-1}$}{7/14}
\slideq{$\mu\otimes\nu$}{8/14}
\slider{L'unique mesure sur $\mathcal A\otimes\mathcal B$ vérifiant $\mu\otimes\nu\l A\times B\r=\mu\l A\r\times\nu\l B\r$\linebreak$\mu\otimes\nu\l C\r=\int[x][X][][\mu]{\nu\l C_x\r}=\int[y][Y][][\nu]{\mu\l C^y\r}$}{8/14}
\slideq{Formule de changements de variables}{9/14}
\slider{Si $U$ et $V$ sont deux ouverts de $\mathbb R^d$ et $\phi\!:\!U\to V$ est un $\mathcal C^1$-difféomorphisme et $f\!:\!V\to\mathbb R$ mesurable, alors $\int[x][V]{f\l x\r}=\int[y][V]{f\circ\phi\l y\r\times J_\phi\l y\r}$\linebreak où $J_\phi\l y\r=\left|\det\l\nabla\phi\l y\r\r\right|$}{9/14}
\slideq{$\l\phi_n\r$ est une approximation de l'unité}{10/14}
\slider{Pour tout $n\in\mathbb N$, $\int[][X]{\phi_n}=1$\linebreak Pour tout $R>0$, $\int[][B\l0,R\r^\complement]{\phi_n}\xrightarrow[n\to+\infty]{}=0$}{10/14}
\slideq{Théorème de Fubini-Lebesgue}{11/14}
\slider{\toggleanalysedisplay Si $f\in L^1\l X\times Y,\mu\otimes\nu\r$ alors $f_x\in L^1\l Y,\nu\r$, $f^y\in L^1\l X,\mu\r$, $\int[y][Y][][\nu]{f_x\l y\r}\in L^1\l X,\mu\r$ et $\int[x][X][][\mu]{f\l x\r}\in L^1\l Y,\nu\r$ et $\altint{\iint}[x,y][X\times Y][][\mu\otimes\nu]{f\l x,y\r}$\linebreak$=\int[x][X][][\mu]{\l\int[y][Y][][\nu]{f\l x,y\r}\r}$\linebreak$=\int[y][Y][][\nu]{\l\int[x][X][][\mu]{f\l x,y\r}\r}$\toggleanalysedisplay}{11/14}
\slideq{$\mathcal A\otimes\mathcal B$}{12/14}
\slider{$\sigma\l\left\{A\times B,\l A,B\r\in\mathcal A\times\mathcal B\right\}\r$\linebreak C'est la plus petite tribu qui rend les fonctions coordonnées mesurables }{12/14}
\slideq{Densité des fonctions $\mathcal C^\infty_c\l\mathbb R^d\r$ dans $L^p$}{13/14}
\slider{Si $p\in\left[1,+\infty\right[$ alors $\mathcal C^\infty_c\l\mathbb R^d\r$ est dense dans $L^p\l\mathbb R^d\r$}{13/14}
\slideq{$\l f*g\r\l x\r$\linebreak$\l f,g\r\in L^1\l\mathbb R^d\r$}{14/14}
\slider{$\int[y][\mathbb R^d]{f\l x-y\r g\l y\r}$\linebreak$f*g\in L^1\l\mathbb R^d\r$ et $\lnrm1{f*g}\leqslant\lnrm1f\lnrm1g$}{14/14}
\end{document}
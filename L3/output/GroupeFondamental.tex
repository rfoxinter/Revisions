\documentclass[14pt,usepdftitle=false,aspectratio=169]{beamer}
\usepackage{preambule}
\setbeamercolor{structure}{fg=black}
\let\Pi\varPi
\hypersetup{pdftitle=Groupe fondamental et revêtement -- Groupe fondamental}
\title{Groupe fondamental et revêtement\\\emph{Groupe fondamental}}
\author{}
\date{}
\begin{document}
\begin{frame}
    \titlepage
\end{frame}
\slideq{$c_1\sim c_2$\linebreak Homotopie de chemins}{1/3}
\slider{Il existe $H\!:\!\left[0,1\right]\times\left[0,1\right]\to B$ continue telle que $H\l0,\cdot\r=c_1$, $H\l1,\cdot\r=c_2$, pour tout $s\in\left[0,1\right]$, $H\l s,0\r=c_1\l0\r=c_2\l0\r$ et $H\l s,1\r=c_1\l1\r=c_2\l1\r$}{1/3}
\slideq{Espace pointé}{2/3}
\slider{$\l B,b\r$ où $B$ est un espace topologique et $b\in B$ est appelé point base}{2/3}
\slideq{$\Pi_1\l B,b\r$}{3/3}
\slider{$\l\left\{\text{classes d'homotopie de lacets basés en $b$}\right\},*\r$\linebreak$*$ désigne la loi de concaténation de chemins}{3/3}
\end{document}
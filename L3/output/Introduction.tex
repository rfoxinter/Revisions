\documentclass[14pt,usepdftitle=false,aspectratio=169]{beamer}
\usepackage{preambule}
\setbeamercolor{structure}{fg=black}
\usepackage{al}
\hypersetup{pdftitle=Groupe fondamental et revêtement -- Introduction}
\title{Groupe fondamental et revêtement\\\emph{Introduction}}
\author{}
\date{}
\begin{document}
\begin{frame}
    \titlepage
\end{frame}
\slideq{Foncteur covariant}{1/3}
\slider{Application $F$ d'un objet $X$ d'une catégorie dans un objet $Y$ d'une autre catégorie\linebreak Si $f\!:\!X\to Y$ est un morphisme alors $F\l f\r\!:\!F\l X\r\to F\l Y\r$ est un morphisme\linebreak$F\l\id_X\r=\id_{F\l X\r}$\linebreak$F\l g\circ f\r=F\l g\r\circ F\l f\r$\linebreak En particulier, si $f$ est un isomorphisme alors $F\l f\r$ aussi}{1/3}
\slideq{Théorème du point fixe de Brower}{2/3}
\slider{Toute application continue de $\mathcal B^n$ dans $\mathcal B^n$ a au moins un point fixe}{2/3}
\slideq{Théorème d'invariance du domaine de Brower}{3/3}
\slider{Si $m\neq n$ alors $\mathbb R^m$ et $\mathbb R^n$ ne sont pas homéomorphes}{3/3}
\end{document}
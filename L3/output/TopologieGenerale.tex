\documentclass[14pt,usepdftitle=false,aspectratio=169]{beamer}
\usepackage{preambule}
\setbeamercolor{structure}{fg=black}

\hypersetup{pdftitle=Topologie et calcul différentiel -- Topologie générale}
\title{Topologie et calcul différentiel\\\emph{Topologie générale}}
\author{}
\date{}
\begin{document}
\begin{frame}
    \titlepage
\end{frame}
\slideq{Distances équivalentes}{1/5}
\slider{Distances qui engendrent la même topologie}{1/5}
\slideq{Topologie sur $X$}{2/5}
\slider{Ensemble de parties de $X$ appelé ouvets qui vérifie\linebreak $\varnothing$ et $X$ sont ouverts\linebreak Une intersection finie d'ouverts est un ouvert\linebreak Une union quelconque d'ouverts est un ouvert}{2/5}
\slideq{Topologie engendrée par $C$ ensemble de parties de $X$}{3/5}
\slider{Topologie qui a pour base d'ouverts les intersections finies d'éléments de $C$}{3/5}
\slideq{Propriété $T_0$ d'un espace topologique}{4/5}
\slider{Tout espace topologique métrisable est séparé (il existe $U$ et $V$ des ouverts de $x$ et $y$ tels que $U\cap V=\varnothing$)}{4/5}
\slideq{Topologie produit $X\times Y$}{5/5}
\slider{Topologie qui a pour base d'ouverts les $U\times V$ avec $U$ un ouvert de $X$ et $V$ un ouvert de $Y$}{5/5}
\end{document}
\documentclass[14pt,usepdftitle=false,aspectratio=169]{beamer}
\usepackage{preambule}
\setbeamercolor{structure}{fg=black}
\let\Pi\varPi
\hypersetup{pdftitle=Groupe fondamental et revêtement -- Théorème de van Kampen}
\title{Groupe fondamental et revêtement\\\emph{Théorème de van Kampen}}
\author{}
\date{}
\begin{document}
\begin{frame}
    \titlepage
\end{frame}
\slideq{Lien entre $\Pi_1^\text{rev}\l B,b\r$ et $\Pi_1^\text{lacet}\l B,b\r$}{1/3}
\slider{Si $B$ est connexe, localement connexe par arcs, localement trivialisable et semi-localement simplement connexe alors $\Pi_1^\text{rev}\l B,b\r$ et $\Pi_1^\text{lacet}\l B,b\r$ sont isomorphes}{1/3}
\slideq{Espace semi-localement simplement connexe}{2/3}
\slider{Tout $x\in X$ admet un voisinage $V$ tel que tout lacet en $x$ dans $V$ est homotope au lacet constant}{2/3}
\slideq{Théorème de van Kampen}{3/3}
\slider{Si $X=U_1\cup U_2$ avec $U_1$ et $U_2$ deux ouverts non vides tels que $U_1\cap U_2$ est un ouvert non vide et connexe par arcs, soient $x\in U_1\cap U_2$ et $\psi_i\!:\!\Pi_1^\text{lacet}\l U_1\cup U_2,x\r\to\Pi_1^\text{lacet}\l U_i,x\r\to\Pi_1^\text{lacet}\l X,x\r$ alors\linebreak$\psi\!:\!\Pi_1^\text{lacet}\l U_1\r*\Pi_1^\text{lacet}\l U_2,x\r\to\!\!\!\!\!\to\Pi_1^\text{lacet}\l X,x\r$ est un morphisme de groupes dont le noyau vaut $\left\langle\!\!\left\langle\psi_1\l g\r*\psi_2\l g^{-1}\r,g\in\Pi_1^\text{lacet}\l U_1\cap U_2,x\r\right\rangle\!\!\right\rangle$}{3/3}
\end{document}
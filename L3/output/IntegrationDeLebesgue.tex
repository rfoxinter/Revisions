\documentclass[14pt,usepdftitle=false,aspectratio=169]{beamer}
\usepackage{preambule}
\setbeamercolor{structure}{fg=black}
\usepackage{bigoperators,analyse}\usepackage{dsft}\usepackage{structures}\usepackage{usuelles}\toggleanalysepar
\hypersetup{pdftitle=Intégration et théorie de la mesure -- Intégration de Lebesgue}
\title{Intégration et théorie de la mesure\\\emph{Intégration de Lebesgue}}
\author{}
\date{}
\begin{document}
\begin{frame}
    \titlepage
\end{frame}
\slideq{Lemme de Fatou}{1/8}
\slider{Si $\l f_n\r$ est une suite de fonctions mesurables alors $\liminf\l\int[\mu][X]{f_n}\r\leqslant\int[\mu][X]{\liminf\l f_n\r}$}{1/8}
\slideq{$f$ est étagée}{2/8}
\slider{$f=\sum{i=1}{n}{\lambda_i\1{\left\{f=\lambda_i\right\}}}$}{2/8}
\slideq{Inegailté de Tchebychev}{3/8}
\slider{Si $\alpha>0$, $\mu\l\left\{f\geqslant\alpha\right\}\r\leqslant\frac1\alpha\int[\mu][X]f$}{3/8}
\slideq{Intégrale d'une fonction étagée}{4/8}
\slider{$\int[\mu]{f}=\sum{\lambda\in\im f}{}{\lambda\mu\l\left\{f=\lambda_i\right\}\r}$}{4/8}
\slideq{Théorème de convergence dominée}{5/8}
\slider{Si $\l f_n\!:\!\l X,\mathcal A\r\to\overline{\mathbb R_+}\r$ est une suite de fonctions mesurables qui convergent simplement vers $f$ et telle qu'il existe $g\in L^1\l X\r$ pour laquelle $f_n\leqslant\left|g\right|$ $\mu$-pp pour tout $n\in\mathbb N$, alors $\lim{\int[\mu][X]{f_n}}=\int[\mu][X]f$}{5/8}
\slideq{Lien entre fonction mesurable et fonction étagée}{6/8}
\slider{Toute fonction mesurable est limite simple de fonctions mesurables croissantes}{6/8}
\slideq{Théorème de convergence monotone (ou Beppo-Levi)}{7/8}
\slider{Si $\l f_n\!:\!\l X,\mathcal A\r\to\overline{\mathbb R_+}\r$ est une suite croissantes de fonctions mesurables qui convergent simplement vers $f$ alors $\lim{\int[\mu][X]{f_n}}=\int[\mu][X]f$}{7/8}
\slideq{Intégrale d'une fonction positive $f$}{8/8}
\slider{$\int[\mu][X]f=\sup{\left\{\int[\mu][X]g,g\text{ étagée }, g\leqslant f\right\}}$}{8/8}
\end{document}
\documentclass[14pt,usepdftitle=false,aspectratio=169]{beamer}
\usepackage{preambule}
\setbeamercolor{structure}{fg=black}
\usepackage{al}\usepackage{footnotes}\let\phi\varphi\usepackage{structures}\usepackage{bigoperators}\togglebigoppar\usepackage{dsft}
\hypersetup{pdftitle=Algèbre 1 -- Représentations de groupes}
\title{Algèbre 1\\\emph{Représentations de groupes}}
\author{}
\date{}
\begin{document}
\begin{frame}
    \titlepage
\end{frame}
\slideq{$\rho\otimes\sigma$\linebreak$\l\rho,V\r$ et $\l\sigma,W\r$ deux représentations de $G$}{1/18}
\slider{$\appl{\rho\otimes\sigma}{G}{\gl{V\otimes W}}{g}{\l v\otimes w\mapsto g\cdot v\otimes g\cdot w\r}$}{1/18}
\slideq{$\l\rho,V\r$ est fidèle}{2/18}
\slider{$\rho$ est injetif}{2/18}
\slideq{Représentation régulière\linebreak$V_G$}{3/18}
\slider{$\appl\rho G{\Bbbk^G}f{\l h\mapsto f\l gh\r\r}$}{3/18}
\slideq{$\hom[G]{V,W}$}{4/18}
\slider{$\hom{V,W}^G=\left\{f\in\hom{V,W},g\circ f=f\circ g\right\}$}{4/18}
\slideq{Caractérisation des groupes abéliens par les représentations}{5/18}
\slider{Si $G$ est un groupe fini alors $G$ est abélien si et seulement si toutes ses représentations irréductibles sur $\mathbb C$ sont de degré $1$}{5/18}
\slideq{$\phi_{\hom{V,W}}$\linebreak$\l\rho,V\r$ et $\l\sigma,W\r$ deux représentations de $G$}{6/18}
\slider{$\appl{\phi_{\hom{V,W}}}{G}{\gl{\hom{V,W}}}{g}{\l f\mapsto\sigma\l g\r\circ f\circ\rho\l g^{-1}\r\r}$}{6/18}
\slideq{Représentation duale de $\l\rho,V\r$}{7/18}
\slider{$\appl{\rho^*}{G}{V^*}{g}{\l f\mapsto f\circ\rho\l g^{-1}\r\r}=\phi_{\hom{V,{\ifpdf\bshft=3.75pt\else\bshft=2.45pt\fi\fakebold{\mathds{1}}}}}$}{7/18}
\slideq{$\l\rho,V\r$ est irréductible}{8/18}
\slider{La représentation est de degré $\geqslant1$\linebreak Les seules sous-représentations sont $\left\{0\right\}$ et $V$}{8/18}
\slideq{Lemme de Schur}{9/18}
\slider{Si $V$ et $W$ sont deux représentations irréductibles de $G$ alors soit $\hom[G]{V,W}=\left\{0\right\}$ soit $\hom[G]{V,W}\cong\Bbbk$}{9/18}
\slideq{Supplémentaire stable}{10/18}
\slider{Si $\l\rho,V\r$ est une représentation de $G$ est $W$ une sous-représentation alors il existe un supplémentaire $W'$ de $W$ qui est une sous-représentation de $G$ et $V=W\oplus W'$}{10/18}
\slideq{Théorème de Maschke}{11/18}
\slider{Si $\car\Bbbk=0$ ou $\car\Bbbk\nmid\left|G\right|$ et $\left|G\right|<+\infty$ alors toute représentation de $G$ se décompose en somme directe de sous-représentation irréductibles}{11/18}
\slideq{Morphisme de représentations de $\l\rho,V\r$ vers $\l\sigma,W\r$}{12/18}
\slider{$f\!:\!V\to W$ linéaire tel que pour tout $g\in G$ et tout $v\in V$, $f\l g\cdot v\r=g\cdot f\l v\r$ ie, $f\l\phi\l g\r\l v\r\r=\sigma\l g\r\l f\l v\r\r$}{12/18}
\slideq{Torsion de la représentation $\l\rho,V\r$ par $\l\chi,\Bbbk\r$}{13/18}
\slider{$\l\rho,V\r\otimes\l\chi,\Bbbk\r=\l\nappl{G}{\gl{V}}{g}{\chi\l g\r\rho\l g\r},V\r$}{13/18}
\slideq{Degré de $\l\rho,V\r$}{14/18}
\slider{$\dim V$}{14/18}
\slideq{Sous-représentation de $\l\rho,V\r$}{15/18}
\slider{$W$ sev de $V$ tel que, pour tout $g\in G$, $g\cdot W\subset W$\footnote{On a en fait $g\cdot W=W$ car $g$ est injectif et $\dim W\leqslant\dim V<+\infty$}}{15/18}
\slideq{Représentation de groupe}{16/18}
\slider{$\l\rho,V\r$ avec $V$ un $\Bbbk$-ev de dimension finie et $\rho\!:\!G\to\gl V$ un morphisme de groupes}{16/18}
\slideq{$\rho\oplus\sigma$\linebreak$\l\rho,V\r$ et $\l\sigma,W\r$ deux représentations de $G$}{17/18}
\slider{$\appl{\rho\oplus\sigma}{G}{\gl{V\oplus W}}{g}{\l v\oplus w\mapsto g\cdot v\oplus g\cdot w\r}$}{17/18}
\slideq{Décomposition en comosantes irréductibles d'une représentation}{18/18}
\slider{Si $W$ est une représentation de $G$ est $\Bbbk$ est algébriquement clos tel que $\car\Bbbk=0$ ou $\car\Bbbk\nmid\left|G\right|$ et $\left|G\right|<+\infty$ alors $W=\bigop{V\in\mathcal I_G\l\Bbbk\r}{}{V^{\dim{\hom[G]{V,W}}}}$\linebreak Cette décomposition est unique à isomorphisme près}{18/18}
\end{document}
\documentclass[14pt,usepdftitle=false,aspectratio=169]{beamer}
\usepackage{preambule}
\setbeamercolor{structure}{fg=black}
\DeclareMathOperator{\oldstab}{stab}\newcommand{\stab}[1]{\oldstab\l#1\r}\DeclareMathOperator{\oldfix}{fix}\newcommand{\fix}[1]{\oldfix\l#1\r}\usepackage{bigoperators}\usepackage{matrices}\usepackage{arithmetique}
\hypersetup{pdftitle=Algèbre 1 -- Actions de groupes}
\title{Algèbre 1\\\emph{Actions de groupes}}
\author{}
\date{}
\begin{document}
\begin{frame}
    \titlepage
\end{frame}
\slideq{Formule des classes}{1/16}
\slider{Si $G$ agit sur $X$ et $\mathcal R$ est un ensemble de représentants des orbites de l'action alors $\left|X\right|=\sum{x\in\mathcal R}{}{\left|G/\stab x\right|}$}{1/16}
\slideq{$\left|G/\stab x\right|$ pour $G$ fini}{2/16}
\slider{$\left|G/\stab x\right|=\frac{\left|G\right|}{\left|\stab x\right|}=\left|G\cdot x\right|$}{2/16}
\slideq{Fixateurs de $g$}{3/16}
\slider{$\fix g=\left\{x\in X,g\cdot x=x\right\}$}{3/16}
\slideq{Non-trivialité de $X^G$ pour $G$ un $p$-groupe}{4/16}
\slider{$\cgr{\left|X^G\right|}{\left|X\right|}{p}$\linebreak En particulier, $Z\l G\r$ est non trivial}{4/16}
\slideq{Action transitive}{5/16}
\slider{$\forall\l x,y\r\in X^2$, $\exists g\in G$, $g\cdot x=y$}{5/16}
\slideq{Formule de Burnside}{6/16}
\slider{Le nombre d'orbites $N$ d'une action de $G$ sur $X$ vérifie $N=\frac1{\left|G\right|}\sum{g\in G}{}{\left|\fix g\right|}$}{6/16}
\slideq{Action libre}{7/16}
\slider{$\forall g\neq 1$, $\fix g\neq\varnothing$}{7/16}
\slideq{Action simplement transitive}{8/16}
\slider{Action libre et transitive\linebreak$\forall\l x,y\r\in X^2$, $\exists!g\in G$, $g\cdot x=y$}{8/16}
\slideq{Orbite de $x$}{9/16}
\slider{$G\cdot x=\left\{g\cdot x,g\in G\right\}$}{9/16}
\slideq{Lemme de Cauchy}{10/16}
\slider{Si $p\in\mathbb P$ divise l'ordre de $G$, alors $G$ possède un élément d'ordre $p$}{10/16}
\slideq{Action fidèle}{11/16}
\slider{$\alpha$ est injective}{11/16}
\slideq{Stabilisateur de $x$}{12/16}
\slider{$\stab x=\left\{g\in G,g\cdot x=x\right\}$}{12/16}
\slideq{$X^G$}{13/16}
\slider{$\bigcap{g\in G}{}{\fix g}$}{13/16}
\slideq{$p$-groupe}{14/16}
\slider{Pour $p\in\mathbb P$, un $p$-groupe est un groupe $G$ vérifiant $\left|G\right|=p^n$}{14/16}
\slideq{Lemme de Cayley et conséquence}{15/16}
\slider{Tout groupe $G$ fini se realise comme un sous-groupe de $\mathfrak S\l G\r\cong\mathfrak S_n$\linebreak Si $\Bbbk$ est un corps, $G$ est isomorphe à un sous-groupe de $\matgl n\Bbbk$}{15/16}
\slideq{Action d'un groupe $G$ sur un ensemble $X$}{16/16}
\slider{Morphisme $\alpha\!:\!G\to\l\mathfrak S\l X\r,\circ\r$}{16/16}
\end{document}
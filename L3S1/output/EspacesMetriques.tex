\documentclass[14pt,usepdftitle=false,aspectratio=169]{beamer}
\usepackage{preambule}
\setbeamercolor{structure}{fg=black}
\usepackage{topologie}
\hypersetup{pdftitle=Topologie et calcul différentiel -- Espaces métriques}
\title{Topologie et calcul différentiel\\\emph{Espaces métriques}}
\author{}
\date{}
\begin{document}
\begin{frame}
    \titlepage
\end{frame}
\slideq{Distance sur $E$}{1/7}
\slider{$d\!:\!E\times E\to\mathbb R$ vérifiant\linebreak$d\l x,y\r\geqslant0$ et $d\l x,y\r=0\Leftrightarrow x=y$\linebreak$d\l x,y\r=d\l y,x\r$\linebreak$d\l x,y\r\leqslant d\l x,z\r+d\l z,y\r$}{1/7}
\slideq{$f\!:\!\l X,d\r\to\l Y,d'\r$ est une isométrie}{2/7}
\slider{$d'\l f\l x\r,f\l x'\r\r=d\l x,x'\r$}{2/7}
\slideq{Lien entre espace métrique et les evn par des isomorphies}{3/7}
\slider{Si $\l X,d\r$ est un espace méteique, il existe un evn $\l E,\nrm\cdot\r$ et une isométrie $i\!:\!X\to E$}{3/7}
\slideq{$\l X,d\r$ est connexe\linebreak Caractérisation par les ouverts}{4/7}
\slider{Il n'existe pas d'ouverts disjoints $U$ et $V$ tels que $X=U\sqcup V$}{4/7}
\slideq{Lien entre connexe par arcs est connexe}{5/7}
\slider{Tout connexe par arcs est connexe\linebreak Si $X$ est connexe et localement connexe par arcs (connexe par arcs sur un voisinnage de chaque point) alors $X$ est connexe}{5/7}
\slideq{$f$ est un homéomorphisme}{6/7}
\slider{$f$ est continue, bijective et de réciproque continue}{6/7}
\slideq{$\l X,d\r$ est connexe\linebreak Caractérisation par les fonctions}{7/7}
\slider{$X$ est connexe si et seulement si toute fonction continue $f\!:\!X\to\left\{0,1\right\}$ est constante}{7/7}
\end{document}
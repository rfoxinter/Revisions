\documentclass[14pt,usepdftitle=false,aspectratio=169]{beamer}
\usepackage{preambule}
\setbeamercolor{structure}{fg=black}
\usepackage{matrices}\DeclareMathOperator{\ut}{UT}
\hypersetup{pdftitle=Algèbre 1 -- Groupes de Sylow}
\title{Algèbre 1\\\emph{Groupes de Sylow}}
\author{}
\date{}
\begin{document}
\begin{frame}
    \titlepage
\end{frame}
\slideq{$p$-groupe de Sylow}{1/3}
\slider{Si $\left|G\right|=p^lm$, $p\wedge m=1$, un $p$-groupe de Sylow est un sous-groupe $H$ de $G$ vérifiant $\left|H\right|=p^l$}{1/3}
\slideq{Théorèmes de Sylow de $G$ avec $p\mid\left|G\right|$}{2/3}
\slider{$G$ contient un $p$-Sylow\linebreak Les $p$-Sylow sont conjugués\linebreak Tout $p$-groupe est inclus dans un $p$-Sylow\linebreak Le nombre de $p$-Sylow est $\equiv1\ \left[p\right]$ et divise $\frac{\left|G\right|}{p^{v_p\l\left|G\right|\r}}$}{2/3}
\slideq{Groupe de Sylow de $\matgl n{\mathbb F_p}$}{3/3}
\slider{$\ut_n\l\mathbb F_p\r$}{3/3}
\end{document}
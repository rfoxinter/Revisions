\documentclass[14pt,usepdftitle=false,aspectratio=169]{beamer}
\usepackage{preambule}
\setbeamercolor{structure}{fg=black}

\hypersetup{pdftitle=Fractions continues -- Fractions continues simples infinies}
\title{Fractions continues\\\emph{Fractions continues simples infinies}}
\author{}
\date{}
\begin{document}
\begin{frame}
    \titlepage
\end{frame}
\slideq{Fraction continue périodique}{1/3}
\slider{Une fraction continue simple est périodique si et seulement si elle converge vers $\frac{\pm a\pm\sqrt b}c$ pour $\l a,b,c\r\in\mathbb N^3$}{1/3}
\slideq{Propriétés de $\l\frac{p_{2n}}{q_{2n}}\r$ et $\l\frac{p_{2n+1}}{q_{2n+1}}\r$ pour $x\notin\mathbb Q$}{2/3}
\slider{$\l\frac{p_{2n}}{q_{2n}}\r$ est strictement croissante\linebreak$\l\frac{p_{2n+1}}{q_{2n+1}}\r$ est strictement décroissante\linebreak Les suites $\l\frac{p_{2n}}{q_{2n}}\r$ et $\l\frac{p_{2n+1}}{q_{2n+1}}\r$ sont adjacentes}{2/3}
\slideq{Lien entre $x$, la suite des réduites et le reste de son développement}{3/3}
\slider{$x=\frac{p_nx_{n+1}+p_{n-1}}{q_nx_{n+1}+q_{n-1}}$}{3/3}
\end{document}
\section{Listes chaînées}
%%%%%%%%%%%%%%%%%%%%%%%%%%%%%%%%%%%%%%%%%%%%%%%%%%%%%%%%%%%%%%%%%%%%%%%%%%%%%%%%%%%%%%%%%%%%%%%%%%%
\c
\begin{tp*}{}
\begin{minted}{c}
struct cell {
    int value;
    struct cell* next;
};

typedef struct cell int_list;

/*
    Cette définition de listes chaînées ainsi que la majorité des
    fonctions à suivre s'adaptent également pour les autres types
*/
\end{minted}
\end{tp*}
\begin{fnc*}{premier élément d'une liste}
\begin{minted}{c}
int fst(int_list* lst) {
    return int_list->value;
}
\end{minted}
\info{}{Retourne le premier élément d'une liste pointée par \mintinline{c}|lst|}{\Th{1}}
\end{fnc*}
\begin{fnc*}{dernier élément d'une liste}
\begin{minted}{c}
int last(int_list* lst) {
    while (lst->nest != NULL) {lst = lst->next;}
    return int_list->value;
}
\end{minted}
\info{}{Retourne le dernier élément d'une liste pointée par \mintinline{c}|lst|}{\Th{n}}
\end{fnc*}
\begin{fnc*}{longueur d'une liste}
\begin{minted}{c}
int length(int_list* lst) {
    if (lst == NULL) {
        return 0;
    }
    return 1 + length(lst->next);
}
\end{minted}
\info{}{Renvoie la longueur de la liste}{\Th{n}}
\end{fnc*}
\begin{fnc*}{ajout d'un élément à gauche d'une liste}
\begin{minted}{c}
void add_l(int elem, int_list* lst) {
    int_list* new_p = (int_list*)malloc(sizeof(int_list));
    new_p->next = lst;
    new_p->value = elem;
}
\end{minted}
\info{}{Ajoute un élément \mintinline{c}|elem| à gauche de la liste pointée par \mintinline{c}|lst|}{\Th{1}}
\end{fnc*}
\begin{fnc*}{ajout d'un élément à droite d'une liste}
\begin{minted}{c}
void add_r(int elem, int_list* lst) {
    while (lst->next != NULL) {lst = lst->next;}
    int_list* new_p = (int_list*)malloc(sizeof(int_list));
    new_p->next = NULL;
    new_p->value = elem;
    lst->next = new_p;
}
\end{minted}
\info{}{Ajoute un élément \mintinline{c}|elem| à droite de la liste pointée par \mintinline{c}|lst|}{\Th{n}}
\end{fnc*}
\begin{fnc*}{appartenance à une liste}
\begin{minted}{c}
bool mem(int elem, int_list* lst) {
    if (lst->value == elem) {return true;}
    if (lst->next != NULL) {
        return mem(elem, lst->next);
    }
}
\end{minted}
\info{}{Vérifie si un élément \mintinline{c}|elem| est dans la liste pointée par \mintinline{c}|lst|}{\O{n}}
\end{fnc*}
\begin{fnc*}{retournement d'une liste}
\begin{minted}{c}
voir rev(int_list* lst) {
    int_list* rev_p = NULL;
    for (int_list* ptr = p_liste; ptr != NULL; ptr = ptr->next) {
        liste_int* tmp = (int_list*)malloc(sizeof(int_list));
        tmp->value = ptr->valeur;
        tmp->next = rev_p;
        rev_p = tmp;
    }
    return rev_p;
}
\end{minted}
\info{}{Retourne les éléments de la liste pointée par \mintinline{c}|lst|}{\Th{n}}
\end{fnc*}
\begin{fnc*}{suppression d'une liste}
\begin{minted}{c}
void del(int_list* lst){
    while (lst->next != NULL) {
        list_int* tmp = lst;
        lst = tmp->next;
        free(tmp);
    }
}
\end{minted}
\info{}{Supprime une liste pointée par \mintinline{c}|lst|}{\Th{n}}
\end{fnc*}
%%%%%%%%%%%%%%%%%%%%%%%%%%%%%%%%%%%%%%%%%%%%%%%%%%%%%%%%%%%%%%%%%%%%%%%%%%%%%%%%%%%%%%%%%%%%%%%%%%%
\ocaml[Le module \kywd{List}]
\fn{ocaml}{List.length}{'a list -> int}{Renvoie la longueur de la liste}{\Th{n}}
\begin{imp*}{}
\begin{minted}{ocaml}
let rec length = function
    | [] -> 0
    | h::t -> 1 + length t;;
\end{minted}
\info{'a list -> int}{Renvoie la longueur de la liste}{\Th{n}}
\end{imp*}
\fn{ocaml}{List.iter}{('a -> unit) -> 'a list -> unit}{Applique une fonction à tous les éléments de la liste}{\Th{n}}
\begin{imp*}{}
\begin{minted}{ocaml}
let rec iter f = function
    | [] -> ()
    | h::t -> f h; iter f t;;
\end{minted}
\info{('a -> unit) -> 'a list -> unit}{Applique une fonction à tous les éléments de la liste}{\Th{n}}
\end{imp*}
\fn{ocaml}{List.fold_left}{('a -> 'b -> 'a) -> 'a -> 'b list -> 'a}{Applique une fonction successivement à un élément de la liste et au résultat de l'itération précédente en partant de la fin de la liste}{\Th{n}}
\begin{imp*}{}
\begin{minted}{ocaml}
let rec fold_left f acc = function
    | [] -> acc
    | h::t -> fold_left f (f acc h) t;;
\end{minted}
\info{('a -> 'b -> 'a) -> 'a -> 'b list -> 'a}{Applique une fonction successivement à un élément de la liste et au résultat de l'itération précédente en partant de la fin de la liste}{\Th{n}}
\end{imp*}
\fn{ocaml}{List.map}{('a -> 'b) -> 'a list -> 'b list}{Applique une fonction à tous les éléments de la liste et renvoie une nouvelle liste avec les résultats}{\Th{n}}
\begin{imp*}{}
\begin{minted}{ocaml}
let rec map f = function
    | [] -> []
    | h::t -> (f h)::(map f t);;
\end{minted}
\info{('a -> 'b) -> 'a list -> 'b list}{Applique une fonction à tous les éléments de la liste et renvoie une nouvelle liste avec les résultats}{\Th{n}}
\end{imp*}
\fn{ocaml}{List.rev}{'a list -> 'a list}{Inverse l'ordre des éléments d'une liste}{\Th{n}}
\begin{imp*}{}
\begin{minted}{ocaml}
let rec rev l =
    let rec aux acc = function
        | [] -> acc
        | h::t -> aux (h::acc) t
    in aux [] l;;
\end{minted}
\info{'a list -> 'a list}{Inverse l'ordre des éléments d'une liste}{\Th{n}}
\end{imp*}
\fn{ocaml}{List.mem}{'a -> 'a list -> bool}{Vérifie si un élément appartient à une liste}{\O{n}}
\begin{imp*}{}
\begin{minted}{ocaml}
let rec mem elem = function
    | [] -> false
    | h::t -> elem = h || mem elem t;;
\end{minted}
\info{'a -> 'a list -> bool}{Vérifie si un élément appartient à une liste}{\O{n}}
\end{imp*}
\documentclass[14pt,usepdftitle=false,aspectratio=169]{beamer}
\usepackage{preambule}
\setbeamercolor{structure}{fg=black}
\let\phi\varphi\newcommand\eps[1]{\varepsilon\l#1\r}\usepackage{al}\usepackage{bigoperators}\usepackage{polynomes,structures}
\hypersetup{pdftitle=Algèbre 2 -- Déterminants}
\title{Algèbre 2\\\emph{Déterminants}}
\author{}
\date{}
\begin{document}
\begin{frame}
    \titlepage
\end{frame}
\slideq{Ensemble des formes $n$-linéaires alternées}{1/18}
\slider{$\vect{\olddet_{\mathcal B}}$}{1/18}
\slideq{Forme $n$-linéaire}{2/18}
\slider{Application linéaire à valeurs dans $\mathbb K$}{2/18}
\slideq{Caractérisation du déterminant par l'image d'une base}{3/18}
\slider{$\det u=\det[\mathcal B]{u\l\mathcal B\r}$}{3/18}
\slideq{$\sl E$}{4/18}
\slider{$\ker\olddet=\left\{u\in\al E{}\;\middle|\;\det u=1\right\}$}{4/18}
\slideq{$u\in\al{E}{}$\linebreak$\det{\lambda u}$}{5/18}
\slider{$\lambda^{\dim E}\det u$}{5/18}
\slideq{$\det A$}{6/18}
\slider{$\sum{\sigma\in\mathfrak S_n}{}{\eps\sigma a_{\sigma\l1\r,1}\cdots a_{\sigma\l n\r,n}}$\linebreak${}=\sum{\tau\in\mathfrak S_n}{}{\eps\tau a_{1,\tau\l1\r}\cdots a_{n,\tau\l n\r}}$}{6/18}
\slideq{$\phi$ est alternée}{7/18}
\slider{$\phi\l x_1,\cdots,x_n\r=0$ s'il existe $i\neq j$ tel que $x_i=x_j$}{7/18}
\slideq{Déterminant d'une famille de vecteurs $\l x_1,\cdots,x_n\r$ par rapport à $\mathcal B$}{8/18}
\slider{Si $\olddet_{\mathcal B}$ est l'unique forme $n$-linéaire alternée telle que $\det[\mathcal B]{\mathcal B}=1$\linebreak$\det[\mathcal B]{x_1,\cdots,x_n}$}{8/18}
\slideq{$\tmatrix[\mtxbox{}{1}{1}\mtxbox{}{2}{2}\mtxbox{}{4}{4}][inner sep = 0.5ex, minimum width = 30pt, minimum height = 30pt, row sep = 1ex, column sep = 1ex,]|{A_1\&0\&\mdots\&0\\0\&A_2\&\ddots\&\vdots\\\vdots\&\ddots\&\ddots\&0\\0\&\mdots\&0\&A_n\\}|$}{9/18}
\slider{$\prod{i=1}{n}{\det{A_i}}$}{9/18}
\slideq{$\phi$ est antisymétrique}{10/18}
\slider{$\phi\l x_1,\cdots,x_n\r=\eps\sigma\phi\l x_{\sigma\l1\r},\cdots,x_{\sigma\l n\r}\r$}{10/18}
\slideq{$\det{u\circ v}$}{11/18}
\slider{$\det u\det v$}{11/18}
\slideq{Application multilinéaire}{12/18}
\slider{$\phi\l x_1,\cdots,x_{i-1},\lambda x_i+x_i',x_{i+1},\cdots,x_n\r$\linebreak${}=\lambda\phi\l x_1,\cdots,x_{i-1},x_i,x_{i+1},\cdots,x_n\r$\linebreak${}+\phi\l x_1,\cdots,x_{i-1},x_i',x_{i+1},\cdots,x_n\r$}{12/18}
\slideq{Déterminant d'un endomorphisme}{13/18}
\slider{$\phi_u=\det u\phi$}{13/18}
\slideq{Lien forme antisymétrique -- forme alternée}{14/18}
\slider{Toute forme $n$-linéaire alternée est antisymétrique\linebreak Si $\car{\mathbb K}\neq2$, toute forme antisymétrique est alternée}{14/18}
\slideq{$\tmatrix|{\lambda_1\&\bullet\&\mdots\&\bullet\\0\&\ddots\&\ddots\&\vdots\\\vdots\&\ddots\&\ddots\&\bullet\\0\&\mdots\&0\&\lambda_n\\}|$}{15/18}
\slider{$\prod{i=1}{n}{\lambda_i}$}{15/18}
\slideq{$\olddet_{\mathcal B}$\linebreak Expression avec $\mathcal B'$}{16/18}
\slider{$\det[\mathcal B]{\mathcal B'}\olddet_{\mathcal B'}$}{16/18}
\slideq{Description du déterminant par les coordonnées\linebreak$\lc x_j\rc_{\mathcal B}=\tmatrix({a_{1,j}\\\vdots\\a_{n,j}\\})$}{17/18}
\slider{$\det[\mathcal B]{x_1,\cdots,x_n}=\sum{\sigma\in\mathfrak S_n}{}{\eps\sigma a_{\sigma\l1\r,1}\cdots a_{\sigma\l n\r,n}}$\linebreak${}=\sum{\tau\in\mathfrak S_n}{}{\eps\tau a_{1,\tau\l1\r}\cdots a_{n,\tau\l n\r}}$}{17/18}
\slideq{$\sl[n]{\mathbb K}$}{18/18}
\slider{$\ker\olddet=\left\{A\in\matgl[n]{\mathbb K}\;\middle|\;\det A=1\right\}$}{18/18}
\end{document}
\documentclass[14pt,usepdftitle=false,aspectratio=169]{beamer}
\usepackage{preambule}
\setbeamercolor{structure}{fg=black}
\usepackage{matrices}\usepackage{bigoperators}
\hypersetup{pdftitle=Algèbre 1 -- Matrices}
\title{Algèbre 1\\\emph{Matrices}}
\author{}
\date{}
\begin{document}
\begin{frame}
    \titlepage
\end{frame}
\slideq{$M=\tmatrix({a \& b \\ c \& d\\})\in\mat{2}{}{\mathbb{K}}$\linebreak$\det M=\tmatrix|{a \& b \\ c \& d \\}|$}{1/18}
\slider{$ad-bc$}{1/18}
\slideq{$\sym{n}{\mathbb K}$}{2/18}
\slider{$\left\{M\in\mat{n}{}{\mathbb{K}}\;|\;M=M^\top\right\}$}{2/18}
\slideq{Factorisation de $A^n-B^n$}{3/18}
\slider{$\l A,B\r\in \mat{n}{}{\mathbb{K}}^2$ tel que $AB=BA$\linebreak$\l A-B\r\sum{k=0}{n-1}{A^{n-k-1}B^k}$}{3/18}
\slideq{$D=\tmatrix({d_{1,1} \& 0 \& \mdots \& 0 \\ 0 \& \ddots \& \ddots \& \vdots \\ \vdots \& \ddots \& \ddots \& 0 \\ 0 \& \mdots \& 0 \& d_{n,n}\\})\in\diag{n}{\mathbb K}$\linebreak$D^m$}{4/18}
\slider{$D^m=\tmatrix({d_{1,1}^m \& 0 \& \mdots \& 0 \\ 0 \& \ddots \& \ddots \& \vdots \\ \vdots \& \ddots \& \ddots \& 0 \\ 0 \& \mdots \& 0 \& d_{n,n}^m\\})$}{4/18}
\slideq{$T=\tmatrix({t_{1,1} \& 0 \& \mdots \& 0 \\ \bullet \& \ddots \& \ddots \& \vdots \\ \vdots \& \ddots \& \ddots \& 0 \\ \bullet \& \mdots \& \bullet \& t_{n,n}\\})\in\ti{n}{\mathbb K}$\linebreak$T^m$}{5/18}
\slider{$T^m=\tmatrix({t_{1,1}^m \& 0 \& \mdots \& 0 \\ \bullet \& \ddots \& \ddots \& \vdots \\ \vdots \& \ddots \& \ddots \& 0 \\ \bullet \& \mdots \& \bullet \& t_{n,n}^m\\})$}{5/18}
\slideq{Structure de $\mat{n}{}{\mathbb{K}}$}{6/18}
\slider{$\mat{n}{}{\mathbb{K}}$ est un anneau non commutatif}{6/18}
\slideq{Factorisation de $\l A+B\r^n$}{7/18}
\slider{$\l A,B\r\in \mat{n}{}{\mathbb{K}}^2$ tel que $AB=BA$\linebreak$\sum{k=0}{n}{\binom{n}{k}A^kB^{n-k}}$}{7/18}
\slideq{Matrice identité}{8/18}
\slider{$I_n=\tmatrix({1 \& 0 \& \mdots \& 0 \\ 0 \& \ddots \& \ddots \& \vdots \\ \vdots \& \ddots \& \ddots \& 0 \\ 0 \& \mdots \& 0 \& 1\\})$}{8/18}
\slideq{Définition du produit matriciel}{9/18}
\slider{$A\in\mat{n}{p}{\mathbb{K}}$, $B\in\mat{p}{q}{\mathbb{K}}$, $C\in\mat{n}{q}{\mathbb{K}}$\linebreak$C=A\times B$\linebreak$\forall\l i,k\r\in\llb1,n\rrb\times\llb1,q\rrb,\;c_{i,k}=\sum{j=1}{p}{a_{i,j}b_{j,k}}$}{9/18}
\slideq{$\gl{n}{\mathbb K}$}{10/18}
\slider{$\left\{M\in\mat{n}{}{\mathbb{K}}\;|\;\exists N\in\mat{n}{}{\mathbb{K}}\;|\;MN=I_n\right\}$}{10/18}
\slideq{$T=\tmatrix({t_{1,1} \& \bullet \& \mdots \& \bullet \\ 0 \& \ddots \& \ddots \& \vdots \\ \vdots \& \ddots \& \ddots \& \bullet \\ 0 \& \mdots \& 0 \& t_{n,n}\\})\in\ts{n}{\mathbb K}$\linebreak$T^m$}{11/18}
\slider{$T^m=\tmatrix({t_{1,1}^m \& \bullet \& \mdots \& \bullet \\ 0 \& \ddots \& \ddots \& \vdots \\ \vdots \& \ddots \& \ddots \& \bullet \\ 0 \& \mdots \& 0 \& t_{n,n}^m\\})$}{11/18}
\slideq{$M=\tmatrix({a \& b \\ c \& d\\})\in\mat{2}{}{\mathbb{K}}$\linebreak$M^{-1}$}{12/18}
\slider{$M^{-1}=\frac{1}{ad-bc}\tmatrix({d \& -b \\ -c \& a\\})$\linebreak${}=\frac{1}{\det M}\tmatrix({d \& -b \\ -c \& a\\})$}{12/18}
\slideq{Description du produit matriciel par ligne\linebreak$\mathrm{L}_i\l M\r$ représente la $i$-ième ligne de $M$}{13/18}
\slider{$A\in\mat{n}{p}{\mathbb{K}}$, $B\in\mat{p}{q}{\mathbb{K}}$, $C\in\mat{n}{q}{\mathbb{K}}$\linebreak$C=A\times B$\linebreak$\forall i\in\llb1,n\rrb,\;\mathrm{L}_i\l{C}\r=\sum{j=1}{p}{a_{i,j}\mathrm{L}_j\l B\r}$}{13/18}
\slideq{Matrice élémentaite $E_{i,j}$ de $\mat{n}{p}{\mathbb{K}}$}{14/18}
\slider{$\forall\l k,l\r\in\llb1,n\rrb\times\llb1,p\rrb,\;e_{i,j}=\delta_{\l i,j\r,\l k,l\r}=\delta_{i,k}\delta_{j,l}$}{14/18}
\slideq{$\ant{n}{\mathbb K}$}{15/18}
\slider{$\left\{M\in\mat{n}{}{\mathbb{K}}\;|\;M=-M^\top\right\}$}{15/18}
\slideq{Propriété des matrices $E_{i,j}$ dans $\mat{n}{p}{\mathbb{K}}$}{16/18}
\slider{La famille $\l E_{i,j}\r_{\l i,j\r\in\llb1,n\rrb\times\llb1,p\rrb}$ est une base canonique de $\mat{n}{p}{\mathbb{K}}$\linebreak Toute matrice $A$ de $\mat{n}{p}{\mathbb{K}}$ peut s'exprimer de la forme $\sum{\l i,j\r\in\llb1,n\rrb\times\llb1,p\rrb}{}{\lambda_{i,j}E_{i,j}}$}{16/18}
\slideq{Transposée de la matrice $A=\tmatrix({a_{1,1} \& \cdots \& a_{1,p} \\ \vdots \& \xdots \& \vdots \\ a_{n,1} \& \mdots \& a_{n,p}\\})$}{17/18}
\slider{$A^\top=\tmatrix({a_{1,1} \& \mdots \& a_{n,1} \\ \vdots \& \xdots \& \vdots \\ a_{1,p} \& \mdots \& a_{n,p}\\})$}{17/18}
\slideq{Description du produit matriciel par colonne\linebreak$\mathrm{C}_i\l M\r$ représente la $i$-ième colonne de $M$}{18/18}
\slider{$A\in\mat{n}{p}{\mathbb{K}}$, $B\in\mat{p}{q}{\mathbb{K}}$, $C\in\mat{n}{q}{\mathbb{K}}$\linebreak$C=A\times B$\linebreak$\forall k\in\llb1,q\rrb,\;\mathrm{C}_k\l{C}\r=\sum{j=1}{p}{b_{j,k}\mathrm{C}_j\l A\r}$}{18/18}
\end{document}
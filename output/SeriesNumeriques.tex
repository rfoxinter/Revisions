\documentclass[14pt,usepdftitle=false,aspectratio=169]{beamer}
\usepackage{preambule}
\setbeamercolor{structure}{fg=black}
\usepackage{bigoperators}\renewcommand{\leq}{\leqslant}\renewcommand{\geq}{\geqslant}\usepackage{equivalents}\usepackage{analyse,complexes,trigo}\let\oldln\ln\renewcommand\ln[2][]{\oldln^{#1}\l#2\r}
\hypersetup{pdftitle=Analyse -- Séries Numériques}
\title{Analyse\\\emph{Séries Numériques}}
\author{}
\date{}
\begin{document}
\begin{frame}
    \titlepage
\end{frame}
\slideq{Théorème spécial de convergence des séries alternées}{1/13}
\slider{Une série alternée est convergente\linebreak Les sommes partielles sont du signe du premier terme\linebreak Les restes sont du signe de leur premier terme et de valeur absolue plus petite que celle de ce dernier}{1/13}
\slideq{$\serie{u_n}$ diverge grossièrement}{2/13}
\slider{$\l u_n\r$ ne tend pas vers $0$}{2/13}
\slideq{Encadrement des sommes par les intégrales\linebreak$f$ est continue et décroissante sur $\left[n_0,+\infty\right[$ avec $n_0\in\mathbb Z$}{3/13}
\slider{$\int[t][n_0+1][n+1]{f\l t\r}$\linebreak${}\leq\sum{k=n_0+1}{n}{f\l k\r}\leq{}$\linebreak$\int[t][n_0][n]{f\l t\r}$}{3/13}
\slideq{Convergence absolue}{4/13}
\slider{$\serie{u_n}$ converge absolument si $\serie{\left|u_n\right|}$ converge\linebreak Si $\serie{\left|u_n\right|}$ converge alors $\serie{u_n}$ converge}{4/13}
\slideq{Série de Riemann}{5/13}
\slider{$\sum{n=1}{+\infty}{\frac1{n^\alpha}}$\linebreak Une série de Riemann converge si et seulement si $\alpha>1$}{5/13}
\slideq{Série de Bertrand}{6/13}
\slider{$\sum{n=2}{+\infty}{\frac1{n^\alpha\ln[\beta]n}}$\linebreak Une série de Bertrand converge si et seulement si $\l\alpha,\beta\r>\l1,1\r$ pour l'ordre lexicographique}{6/13}
\slideq{Règle de Riemann}{7/13}
\slider{S'il existe $\alpha>1$ tel que $\l n^\alpha u_n\r$ est bornée, alors $\serie{u_n}$ converge\linebreak Si $\l nu_n\r$ est minorée par $m>0$ à partir de $n\in\mathbb N$, alors $\serie{u_n}$ diverge}{7/13}
\slideq{Règle de d'Alembert}{8/13}
\slider{Si $\left|\frac{u_{n+1}}{u_n}\right|\to l$ où $0\leq l<1$, alors $\serie{u_n}$ converge absolument\linebreak Si $\left|\frac{u_{n+1}}{u_n}\right|\to l$ où $l>1$, alors $\serie{u_n}$ diverge grossièrement}{8/13}
\slideq{Comparaison par dominance}{9/13}
\slider{$u_n=\O[]{v_n}$\linebreak Si $\serie{v_n}$ converge alors $\serie{u_n}$ converge\linebreak Si $\serie{u_n}$ ou $\serie{\left|u_n\right|}$ diverge alors $\serie{v_n}$ diverge}{9/13}
\slideq{Critère d'Abel}{10/13}
\slider{Si $\l a_n\r$ est une suite réelle positive décroissante de limite nulle, et la somme partielle de $\serie{b_n}$ est bornée, alors $\serie{a_nb_n}$ converge\linebreak Les suites $\e^{\i n\alpha}$, $\cos{n\alpha}$ et $\sin{n\alpha}$ vérifient les conditions pour $\l b_n\r$ lorsque $\alpha\not\equiv0\ \left[2\pi\right]$}{10/13}
\slideq{Semi-convergence}{11/13}
\slider{Convergence sans convergence absolue}{11/13}
\slideq{Théorème de comparaison des séries à termes positifs}{12/13}
\slider{$\exists N\in\mathbb N,\;\forall n\geq N,\; 0\leq u_n\leq v_n$\linebreak Si $\serie{v_n}$ converge alors $\serie{u_n}$ converge\linebreak Si $\serie{u_n}$ diverge alors $\serie{v_n}$ diverge}{12/13}
\slideq{Série alternée}{13/13}
\slider{$\serie{u_n}$ est alternée s'il existe une suite $\l a_n\r$ positive décroissante de limite nulle telle que $u_n=\l-1\r^na_n$}{13/13}
\end{document}
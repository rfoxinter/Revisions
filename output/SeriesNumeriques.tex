\documentclass[14pt,usepdftitle=false,aspectratio=169]{beamer}
\usepackage{preambule}
\setbeamercolor{structure}{fg=black}
\usepackage{bigoperators}\renewcommand{\leq}{\leqslant}\renewcommand{\geq}{\geqslant}\usepackage{equivalents}\usepackage{analyse}
\hypersetup{pdftitle=Analyse -- Séries Numériques}
\title{Analyse\\\emph{Séries Numériques}}
\author{}
\date{}
\begin{document}
\begin{frame}
    \titlepage
\end{frame}
\slideq{Comparaison par dominance}{1/6}
\slider{$u_n=\O[]{v_n}$\linebreak Si $\serie{v_n}$ converge alors $\serie{u_n}$ converge\linebreak Si $\serie{u_n}$ ou $\serie{\left|u_n\right|}$ diverge alors $\serie{v_n}$ diverge}{1/6}
\slideq{Théorème de comparaison des séries à termes positifs}{2/6}
\slider{$\exists N\in\mathbb N,\;\forall n\geq N,\; 0\leq u_n\leq v_n$\linebreak Si $\serie{v_n}$ converge alors $\serie{u_n}$ converge\linebreak Si $\serie{u_n}$ diverge alors $\serie{v_n}$ diverge}{2/6}
\slideq{Encadrement des sommes par les intégrales\linebreak$f$ est continue et décroissante sur $\left[n_0,+\infty\right[$ avec $n_0\in\mathbb Z$}{3/6}
\slider{$\int[t][n_0+1][n+1]{f\l t\r}$\linebreak${}\leq\sum{k=n_0+1}{n}{f\l k\r}\leq{}$\linebreak$\int[t][n_0][n]{f\l t\r}$}{3/6}
\slideq{Semi-convergence}{4/6}
\slider{Convergence sanc convergence absolue}{4/6}
\slideq{Convergence absolue}{5/6}
\slider{$\serie{u_n}$ converge absolument si $\serie{\left|u_n\right|}$ converge\linebreak Si $\serie{\left|u_n\right|}$ converge alors $\serie{u_n}$ converge}{5/6}
\slideq{$\serie{u_n}$ diverge grossièrement}{6/6}
\slider{$u_n$ ne tend pas vers $0$}{6/6}
\end{document}
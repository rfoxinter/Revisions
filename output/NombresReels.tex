\documentclass[14pt,usepdftitle=false,aspectratio=169]{beamer}
\usepackage{preambule}
\setbeamercolor{structure}{fg=black}
\usepackage{bigoperators}\newcommand\lva{\left|}\newcommand\rva{\right|}
\hypersetup{pdftitle=Fondements -- Nombres réels}
\title{Fondements\\\emph{Nombres réels}}
\author{}
\date{}
\begin{document}
\begin{frame}
    \titlepage
\end{frame}
\slideq{Inégalité de Cauchy-Schwarz numérique}{1/6}
\slider{$\l\sum{i=1}{n}{x_iy_i}\r^2\leqslant\l\sum{i=1}{n}{x_i^2}\r\l\sum{i=1}{n}{y_i^2}\r$}{1/6}
\slideq{Inégalité triangulaire 2}{2/6}
\slider{$\lva\lva a\rva-\lva b\rva\rva\leqslant\lva a+b\rva$}{2/6}
\slideq{Cas d'égalité de l'inégalité arithmético-géométrique}{3/6}
\slider{$\forall i\in\llb1,n\rrb,\;\forall j\in\llb1,n\rrb,\;x_i=x_j$}{3/6}
\slideq{Inégalité arithmético-géométrique}{4/6}
\slider{$\sqrt[n]{\prod{i=1}{n}{x_i}}\leqslant\frac{1}{n}\l\sum{i=1}{n}{x_i}\r$}{4/6}
\slideq{Cas d'égalité de l'inégalité de Cauchy-Schwarz}{5/6}
\slider{$k\in\mathbb{R}$ fixé\linebreak$\forall i\in\llb1,n\rrb,\;x_i=k\times y_i$}{5/6}
\slideq{Inégalité triangulaire 1}{6/6}
\slider{$\lva a+b\rva\leqslant\lva a\rva+\lva b\rva$}{6/6}
\end{document}
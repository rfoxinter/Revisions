\documentclass[14pt,usepdftitle=false,aspectratio=169]{beamer}
\usepackage{preambule}
\setbeamercolor{structure}{fg=black}
\usepackage{probas}\usepackage{bigoperators}
\hypersetup{pdftitle=Probabilités -- Variables aléatoires}
\title{Probabilités\\\emph{Variables aléatoires}}
\author{}
\date{}
\begin{document}
\begin{frame}
    \titlepage
\end{frame}
\slideq{Loi conjointe de $\l X_1,\cdots,X_n\r$}{1/13}
\slider{$\p{X_1,\cdots,X_n}$ définie sur $\l\mathbb R^n,\bor^n\r$}{1/13}
\slideq{Vecteur aléatoire réel}{2/13}
\slider{Variable aléatoire à valeurs dans $\l\mathbb R^n,\bor^n\r$}{2/13}
\slideq{Variable aléatoire réelle}{3/13}
\slider{Variable aléatoire à valeurs dans $\l\mathbb R,\bor^1\r$}{3/13}
\slideq{Application mesurable}{4/13}
\slider{Si $\l E,\mathcal S\r$ et $\l F,\mathcal T\r$ sont deux espaces mesurables et $f\!:\! E\to F$\linebreak$\forall B\in\mathcal T,\;f^{-1}\l B\r\in\mathcal S$}{4/13}
\slideq{$\p{f\l X\r=x}$}{5/13}
\slider{$\sum{y\in f^{-1}\l x\r\cap X\l\Omega\r}{}{\p{X=y}}$}{5/13}
\slideq{Si $X\!:\!\l\Omega,\mathcal T\r\to\l E;\mathcal T'\r$\linebreak$\mathcal T_X$}{6/13}
\slider{$\left\{X^{-1}\l A\r,\;A\in\mathcal T'\right\}$}{6/13}
\slideq{Variable aléatoire discrète}{7/13}
\slider{$X\l\Omega\r$ est fini}{7/13}
\slideq{Structure des variables aléatoires de $\mathbb R^\Omega$}{8/13}
\slider{Sous-algèbre de $\mathbb R^\Omega$}{8/13}
\slideq{$\p[f\l X\r]{}$}{9/13}
\slider{$\p[X]{}\circ\widehat{f^{-1}}$}{9/13}
\slideq{Variable aléatoire}{10/13}
\slider{Application mesurable $X\!:\!\l\Omega,\mathcal T\r\to\l E;\mathcal T'\r$\linebreak Si $\Omega'\in\mathcal T$ tel que $\p{\Omega'}=1$, on peut définir $X$ sur $\Omega'$}{10/13}
\slideq{Loi d'une variable aléatoire}{11/13}
\slider{$\p[X]A=\p{x^{-1}\l A\r}$}{11/13}
\slideq{$k$-ième loi marginale de $\l X_1,\cdots,X_n\r$}{12/13}
\slider{Loi de $X_k$}{12/13}
\slideq{Fonction de répartition de $X\!:\!\Omega\to\mathbb R$}{13/13}
\slider{$F_X\l x\r=\p{X\leqslant x}$}{13/13}
\end{document}
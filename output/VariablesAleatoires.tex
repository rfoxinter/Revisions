\documentclass[14pt,usepdftitle=false,aspectratio=169]{beamer}
\usepackage{preambule}
\setbeamercolor{structure}{fg=black}
\usepackage{probas}\usepackage{bigoperators}\usepackage{matrices}\usepackage{complexes}\let\oldlim\lim\renewcommand\lim[2]{\oldlim\limits_{#1}\l#2\r}
\hypersetup{pdftitle=Probabilités -- Variables aléatoires}
\title{Probabilités\\\emph{Variables aléatoires}}
\author{}
\date{}
\begin{document}
\begin{frame}
    \titlepage
\end{frame}
\slideq{$\esp X$}{1/53}
\slider{$\sum{x\in X\l\Omega\r}{}{x\p{X=x}}=\sum{\omega\in\Omega}{}{\p{\left\{\omega\right\}}X\l\omega\r}$}{1/53}
\slideq{Variable aléatoire réelle}{2/53}
\slider{Variable aléatoire à valeurs dans $\l\mathbb R,\bor^1\r$}{2/53}
\slideq{Loi hypergéométrique}{3/53}
\slider{$X\l\Omega\r\subset\llb1,n\rrb$\linebreak$X\sim\hypg{N}{n}{p}$\linebreak$\forall k\in X\l\Omega\r,\;\p{X=k}=\frac{\binom{Np}k\binom{Nq}{n-k}}{\binom Nn}$\linebreak$\esp X=np$\linebreak$\var X=np\l1-p\r\frac{N-n}{N-1}$}{3/53}
\slideq{$\var{X+Y}$}{4/53}
\slider{$\var X+\var Y+2\cov XY$}{4/53}
\slideq{Inégalité de Cauchy-Schwarz pour $\oldcov$}{5/53}
\slider{$\left|\cov XY\right|\leqslant\ect X\ect Y$}{5/53}
\slideq{Application mesurable}{6/53}
\slider{Si $\l E,\mathcal S\r$ et $\l F,\mathcal T\r$ sont deux espaces mesurables et $f\!:\! E\to F$\linebreak$\forall B\in\mathcal T,\;f^{-1}\l B\r\in\mathcal S$}{6/53}
\slideq{$\cov XY$}{7/53}
\slider{$\esp{\l X-\esp X\r\l Y-\esp Y\r}$\linebreak${}=\esp{XY}-\esp X\esp Y$}{7/53}
\slideq{Coefficient de corrélation}{8/53}
\slider{$\rho\l X,Y\r=\frac{\cov XY}{\ect X\ect Y}$}{8/53}
\slideq{Formule de Koenig-Huyghens}{9/53}
\slider{$\var X=\esp{X^2}-\esp X^2$}{9/53}
\slideq{Espérance conditionnelle}{10/53}
\slider{$\esp{X\sq A}=\sum{x\in X\l\Omega\r}{}{x\p{X=x\sq A}}$}{10/53}
\slideq{$\var{X_1+\cdots+X_n}$}{11/53}
\slider{$\sum{k=1}{n}{\var X_i}+2\sum{1\leqslant i<j\leqslant n}{}{\cov{X_i}{X_j}}$\linebreak${}=\tmatrix[][minimum width=0pt,minimum height=20pt,]({1\&\mdots\&1\\})\underline{\mathbb V}\l X_1,\cdots,X_n\r\tmatrix[][minimum width=0pt,minimum height=20pt,]({1\\\oldvdots\\1\\})$}{11/53}
\slideq{Loi d'une variable aléatoire}{12/53}
\slider{$\p[X]A=\p{X^{-1}\l A\r}$}{12/53}
\slideq{Variables indépendantes}{13/53}
\slider{$X\indep Y$\linebreak$\forall\l A_1,A_2\r\in\mathcal T_1\times\mathcal T_2$\linebreak$\p{X\in A_1,Y\in A_2}=\p{X\in A_1}\p{Y\in A_2}$}{13/53}
\slideq{Loi binomiale}{14/53}
\slider{$X\l\Omega\r=\llb0,n\rrb$\linebreak$X\sim\bin[n]p$\linebreak$\forall k\in\llb0,n\rrb,\;\p{X=k}=\binom{n}{k}p^k\l1-p\r^{n-k}$\linebreak$\esp X=np$\linebreak$\var X=np\l1-p\r$}{14/53}
\slideq{$L^1$}{15/53}
\slider{Variables aléatoires réelles discrètes admettant une espérance finie}{15/53}
\slideq{Loi de Bernoulli}{16/53}
\slider{$X\l\Omega\r=\left\{0,1\right\}$\linebreak$X\sim\bin{p}$\linebreak$\p{X=1}=p$, $\p{X=0}=1-p$\linebreak$\esp X=p$\linebreak$\var X=p\l1-p\r$}{16/53}
\slideq{Loi de Pascal}{17/53}
\slider{$X\l\Omega\r=\llb r,+\infty\right\llbracket$\linebreak$X\sim\pasc{r}{p}$\linebreak$\forall k\in\llb r,+\infty\right\llbracket,\;\p{X=k}=\binom{k-1}{r-1}p^r\l1-p\r^{k-r}$\linebreak$\esp X=\frac rp$\linebreak$\var X=\frac{rq}{p^2}$}{17/53}
\slideq{Si $X\!:\!\l\Omega,\mathcal T\r\to\l E;\mathcal T'\r$\linebreak$\mathcal T_X$}{18/53}
\slider{$\left\{X^{-1}\l A\r,\;A\in\mathcal T'\right\}$}{18/53}
\slideq{$\var X$}{19/53}
\slider{$\esp{\l X-\esp X\r^2}$}{19/53}
\slideq{Matrice des variances-covariances}{20/53}
\slider{$\underline{\mathbb V}\l X_1,\cdots,X_n\r=\l\cov{X_i}{X_j}\r_{\l i,j\r\in\llb1,n\rrb^2}$}{20/53}
\slideq{$\esp{XY}$}{21/53}
\slider{$\esp X\esp Y$ si $X\indep Y$}{21/53}
\slideq{Formule de polarisation}{22/53}
\slider{$\cov XY=\frac12\l\var{X+Y}-\var X-\var Y\r$}{22/53}
\slideq{$\esp{\lambda X+Y}$}{23/53}
\slider{$\lambda\esp X+\esp Y$}{23/53}
\slideq{Loi faible des grands nombres}{24/53}
\slider{$Z_n=\frac{X_1+\cdots+X_n}{n}$ avec les $X_i$ mutuellement indépendantes suivant une même loi et d'espérance $m$\linebreak$\forall\varepsilon>0,\;\forall n\in\mathbb N^*,\;\p{\left|Z_n-m\right|\geqslant\varepsilon}\leqslant\frac{\var X}{n\varepsilon^2}$}{24/53}
\slideq{Fonction de répartition de $X\!:\!\Omega\to\mathbb R$}{25/53}
\slider{$F_X\l x\r=\p{X\leqslant x}$}{25/53}
\slideq{Formule de l'espérance totale}{26/53}
\slider{Si $\l A_i\r$ est un système quasi-complet d'événements au plus dénombrale\linebreak$\esp{X}=\sum{i\in I}{}{\esp{X\sq A_i}\p{A_i}}$}{26/53}
\slideq{Loi quasi-certaine}{27/53}
\slider{$X=c$ presque sûrement\linebreak$\p{X=c}=1$, $\p{X\neq c}=0$\linebreak$\esp X=c$\linebreak$\var X=0$}{27/53}
\slideq{Inégalité de Cauchy-Schwarz pour $\mathbb E$}{28/53}
\slider{$\left|\esp {XY}\right|\leqslant\sqrt{\esp{X^2}\esp{Y^2}}$}{28/53}
\slideq{$\p[f\l X\r]{}$}{29/53}
\slider{$\p[X]{}\circ\widehat{f^{-1}}$}{29/53}
\slideq{Variable centrée}{30/53}
\slider{$\esp X=0$}{30/53}
\slideq{Variable réduite}{31/53}
\slider{$\var X=1$}{31/53}
\slideq{Loi binomiale négative}{32/53}
\slider{$X\l\Omega\r=\mathbb N$\linebreak$X\sim\nbin{r}{p}$\linebreak$\forall k\in\mathbb N,\;\p{X=k}=\binom{k+r-1}{k}p^r\l1-p\r^k$\linebreak$\esp X=\frac{rq}p$\linebreak$\var X=\frac{rq}{p^2}$}{32/53}
\slideq{Variable aléatoire}{33/53}
\slider{Application mesurable $X\!:\!\l\Omega,\mathcal T\r\to\l E,\mathcal T'\r$\linebreak Si $\Omega'\in\mathcal T$ tel que $\p{\Omega'}=1$, on peut définir $X$ sur $\Omega'$}{33/53}
\slideq{$\ect X$}{34/53}
\slider{$\sqrt{\var X}$}{34/53}
\slideq{Moment d'ordre $k$\linebreak Moment centré d'ordre $k$}{35/53}
\slider{$\esp{X^k}$\linebreak$\esp{\l X-\esp X\r^k}$}{35/53}
\slideq{Convergence en probabilités}{36/53}
\slider{$\forall\varepsilon>0,\;\lim{n\to+\infty}{\p{\left|X_n-X\right|\leqslant\varepsilon}}=0$}{36/53}
\slideq{$k$-ième loi marginale de $\l X_1,\cdots,X_n\r$}{37/53}
\slider{Loi de $X_k$}{37/53}
\slideq{Cas d'égalité de l'inégalité de Cauchy-Schwarz pour $\oldcov$}{38/53}
\slider{Il existe $\l a,b\r\neq\l0,0\r$ tel que $aX+bY=c$ presque sûrement}{38/53}
\slideq{$\var{\lambda X+\mu}$}{39/53}
\slider{$\lambda^2\var X$}{39/53}
\slideq{$\p{f\l X\r=x}$}{40/53}
\slider{$\sum{y\in f^{-1}\l x\r\cap X\l\Omega\r}{}{\p{X=y}}$}{40/53}
\slideq{Vecteur aléatoire réel}{41/53}
\slider{Vecteur aléatoire à valeurs dans $\l\mathbb R^n,\bor^n\r$}{41/53}
\slideq{Variables décorrélées}{42/53}
\slider{$\cov XY=0$}{42/53}
\slideq{Inégalités de Markov}{43/53}
\slider{$\p{X\geqslant\lambda\esp X}\leqslant\frac1\lambda$\linebreak$\p{X\geqslant\varepsilon}\leqslant\frac{\esp X}\varepsilon$\linebreak$\p{X>\varepsilon}\leqslant\frac{\esp X}\varepsilon$\linebreak$\p{\left|X\right|\geqslant\varepsilon}\leqslant\frac{\esp{X^2}}{\varepsilon^2}$}{43/53}
\slideq{Variable aléatoire discrète}{44/53}
\slider{$X\l\Omega\r$ est fini}{44/53}
\slideq{Théorème d'or de Bernoulli}{45/53}
\slider{$Z_n=\frac{X_1+\cdots+X_n}{n}$ avec les $X_i$ mutuellement indépendantes suivant une loi de Bernoulli de paramètre $p$\linebreak$\forall\varepsilon>0,\;\forall n\in\mathbb N^*,\;\p{\left|Z_n-p\right|\geqslant\varepsilon}\leqslant\frac1{4n\varepsilon^2}$}{45/53}
\slideq{Structure des variables aléatoires de $\mathbb R^\Omega$}{46/53}
\slider{Sous-algèbre de $\mathbb R^\Omega$}{46/53}
\slideq{Variable centrée réduite associée à $X$}{47/53}
\slider{$X^*=\frac{X-\esp X}{\ect X}$}{47/53}
\slideq{Loi conjointe de $\l X_1,\cdots,X_n\r$}{48/53}
\slider{$\p{X_1,\cdots,X_n}$ définie sur $\l\mathbb R^n,\bor^n\r$}{48/53}
\slideq{Loi uniforme}{49/53}
\slider{$X\l\Omega\r=\llb1,n\rrb$\linebreak$X\sim\unif{n}$\linebreak$\forall k\in\llb1,n\rrb,\;\p{X=k}=\frac1n$\linebreak$\esp X=\frac{n+1}{2}$\linebreak$\var X=\frac{n^2-1}{12}$}{49/53}
\slideq{Loi géométrique}{50/53}
\slider{$X\l\Omega\r=\mathbb N^*$\linebreak$X\sim\geom{n}$\linebreak$\forall k\in\mathbb N^*,\;\p{X=k}=p\l1-p\r^{k-1}$\linebreak$\esp X=\frac1p$\linebreak$\var X=\frac{q}{p^2}$}{50/53}
\slideq{Lemme des coalitions}{51/53}
\slider{Si $\l X_1,\cdots,X_n\r$ sont mutuellement indépendantes, alors $f\l X_1,\cdots,X_m\r$ et $g\l X_{m+1},\cdots,X_n\r$}{51/53}
\slideq{Inégalité de Bienaymé-Tchebychev}{52/53}
\slider{$\forall\varepsilon>0,\;\p{\left|X-\esp X\right|\geqslant\varepsilon}\leqslant\frac{\var X}{\varepsilon^2}$\linebreak$\forall\varepsilon>0,\;\p{\left|X-\esp X\right|\geqslant\varepsilon}\leqslant\frac{\ect X^2}{\varepsilon^2}$}{52/53}
\slideq{Loi de Poisson}{53/53}
\slider{$X\l\Omega\r=\mathbb N$\linebreak$X\sim\poiss{\lambda}$\linebreak$\forall k\in\mathbb N,\;\p{X=k}=\e^{-\lambda}\frac{\lambda^k}{k!}$\linebreak$\esp X=\lambda$\linebreak$\var X=\lambda$}{53/53}
\end{document}
\documentclass[14pt,usepdftitle=false,aspectratio=169]{beamer}
\usepackage{preambule}
\setbeamercolor{structure}{fg=black}
\usepackage{probas}\usepackage{bigoperators}\usepackage{matrices}
\hypersetup{pdftitle=Probabilités -- Variables aléatoires}
\title{Probabilités\\\emph{Variables aléatoires}}
\author{}
\date{}
\begin{document}
\begin{frame}
    \titlepage
\end{frame}
\slideq{Inégalité de Bienaymé-Tchebychev}{1/53}
\slider{À }{1/53}
\slideq{Loi quasi-certaine}{2/53}
\slider{$X=c$ presque sûrement\linebreak$\p{X=c}=1$, $\p{X\neq c}=0$\linebreak$\esp X=c$\linebreak$\var X=0$}{2/53}
\slideq{Formule de polarisation}{3/53}
\slider{$\cov XY=\frac12\l\var{X+Y}-\var X-\var Y\r$}{3/53}
\slideq{Loi d'une variable aléatoire}{4/53}
\slider{$\p[X]A=\p{X^{-1}\l A\r}$}{4/53}
\slideq{Loi faible des grands nombres}{5/53}
\slider{À faire}{5/53}
\slideq{$\esp{\lambda X+Y}$}{6/53}
\slider{$\lambda\esp X+\esp Y$}{6/53}
\slideq{Loi géométrique}{7/53}
\slider{$X\l\Omega\r=\mathbb N^*$\linebreak$X\sim\geom{n}$\linebreak$\forall k\in\mathbb N^*,\;\p{X=k}=p\l1-p\r^{k-1}$\linebreak$\esp X=\frac1p$\linebreak$\var X=\frac{q}{p^2}$}{7/53}
\slideq{Variable aléatoire réelle}{8/53}
\slider{Variable aléatoire à valeurs dans $\l\mathbb R,\bor^1\r$}{8/53}
\slideq{$\p{f\l X\r=x}$}{9/53}
\slider{$\sum{y\in f^{-1}\l x\r\cap X\l\Omega\r}{}{\p{X=y}}$}{9/53}
\slideq{Variable centrée réduite associée à $X$}{10/53}
\slider{$X^*=\frac{X-\esp X}{\ect X}$}{10/53}
\slideq{Loi uniforme}{11/53}
\slider{$X\l\Omega\r=\llb1,n\rrb$\linebreak$X\sim\unif{n}$\linebreak$\forall k\in\llb1,n\rrb,\;\p{X=k}=\frac1n$\linebreak$\esp X=\frac{n+1}{2}$\linebreak$\var X=\frac{n^2-1}{12}$}{11/53}
\slideq{Loi de Poisson}{12/53}
\slider{À faire}{12/53}
\slideq{Convergence en probabilités}{13/53}
\slider{À faire}{13/53}
\slideq{Formule de Koenig-Huyghens}{14/53}
\slider{$\var X=\esp{X^2}-\esp X^2$}{14/53}
\slideq{Fonction de répartition de $X\!:\!\Omega\to\mathbb R$}{15/53}
\slider{$F_X\l x\r=\p{X\leqslant x}$}{15/53}
\slideq{Loi hypergéométrique}{16/53}
\slider{À faire}{16/53}
\slideq{Inégalité de Cauchy-Schwarz pour $\oldcov$}{17/53}
\slider{$\left|\cov XY\right|\leqslant\ect X\ect Y$}{17/53}
\slideq{Structure des variables aléatoires de $\mathbb R^\Omega$}{18/53}
\slider{Sous-algèbre de $\mathbb R^\Omega$}{18/53}
\slideq{Variable aléatoire}{19/53}
\slider{Application mesurable $X\!:\!\l\Omega,\mathcal T\r\to\l E,\mathcal T'\r$\linebreak Si $\Omega'\in\mathcal T$ tel que $\p{\Omega'}=1$, on peut définir $X$ sur $\Omega'$}{19/53}
\slideq{Loi de Bernoulli}{20/53}
\slider{$X\l\Omega\r=\left\{0,1\right\}$\linebreak$X\sim\bin{p}$\linebreak$\p{X=1}=p$, $\p{X=0}=1-p$\linebreak$\esp X=p$\linebreak$\var X=p\l1-p\r$}{20/53}
\slideq{Si $X\!:\!\l\Omega,\mathcal T\r\to\l E;\mathcal T'\r$\linebreak$\mathcal T_X$}{21/53}
\slider{$\left\{X^{-1}\l A\r,\;A\in\mathcal T'\right\}$}{21/53}
\slideq{Loi binomiale}{22/53}
\slider{$X\l\Omega\r=\llb0,n\rrb$\linebreak$X\sim\bin[n]p$\linebreak$\forall k\in\llb0,n\rrb,\;\p{X=k}=\binom{n}{k}p^k\l1-p\r^{n-k}$\linebreak$\esp X=np$\linebreak$\var X=np\l1-p\r$}{22/53}
\slideq{Coefficient de corrélation}{23/53}
\slider{$\rho\l X,Y\r=\frac{\cov XY}{\ect X\ect Y}$}{23/53}
\slideq{Loi binomiale négative}{24/53}
\slider{À faire}{24/53}
\slideq{$\var{X+Y}$}{25/53}
\slider{$\var X+\var Y+2\cov XY$}{25/53}
\slideq{Variable réduite}{26/53}
\slider{$\var X=1$}{26/53}
\slideq{$L^1$}{27/53}
\slider{Variables aléatoires réelles discrètes admettant une espérance finie}{27/53}
\slideq{Inégalités de Markov}{28/53}
\slider{À faire}{28/53}
\slideq{Loi conjointe de $\l X_1,\cdots,X_n\r$}{29/53}
\slider{$\p{X_1,\cdots,X_n}$ définie sur $\l\mathbb R^n,\bor^n\r$}{29/53}
\slideq{$k$-ième loi marginale de $\l X_1,\cdots,X_n\r$}{30/53}
\slider{Loi de $X_k$}{30/53}
\slideq{$\esp{XY}$}{31/53}
\slider{$\esp X\esp Y$ si $X\indep Y$}{31/53}
\slideq{Cas d'égalité de l'inégalité de Cauchy-Schwarz pour $\oldcov$}{32/53}
\slider{Il existe $\l a,b\r\neq\l0,0\r$ tel que $aX+bY=c$ presque sûrement}{32/53}
\slideq{$\var{\lambda X+\mu}$}{33/53}
\slider{$\lambda^2\var X$}{33/53}
\slideq{$\ect X$}{34/53}
\slider{$\sqrt{\var X}$}{34/53}
\slideq{$\var{X_1+\cdots+X_n}$}{35/53}
\slider{$\sum{k=1}{n}{\var X_i}+2\sum{1\leqslant i<j\leqslant n}{}{\cov{X_i}{X_j}}$\linebreak${}=\tmatrix[][minimum width=0pt,minimum height=20pt,]({1\&\mdots\&1\\})\underline{\mathbb V}\l X_1,\cdots,X_n\r\tmatrix[][minimum width=0pt,minimum height=20pt,]({1\\\oldvdots\\1\\})$}{35/53}
\slideq{Matrice des variances-covariances}{36/53}
\slider{$\underline{\mathbb V}\l X_1,\cdots,X_n\r=\l\cov{X_i}{X_j}\r_{\l i,j\r\in\llb1,n\rrb^2}$}{36/53}
\slideq{Variables décorrélées}{37/53}
\slider{$\cov XY=0$}{37/53}
\slideq{Formule de l'espérance totale}{38/53}
\slider{Si $\l A_i\r$ est un système quasi-complet d'événements au plus dénombrale\linebreak$\esp{X}=\sum{i\in I}{}{\esp{X\sq A_i}\p{A_i}}$}{38/53}
\slideq{Vecteur aléatoire réel}{39/53}
\slider{Vecteur aléatoire à valeurs dans $\l\mathbb R^n,\bor^n\r$}{39/53}
\slideq{Théorème d'or de Bernoulli}{40/53}
\slider{À faire}{40/53}
\slideq{Lemme des coalitions}{41/53}
\slider{Si $\l X_1,\cdots,X_n\r$ sont mutuellement indépendantes, alors $f\l X_1,\cdots,X_m\r$ et $g\l X_{m+1},\cdots,X_n\r$}{41/53}
\slideq{$\esp X$}{42/53}
\slider{$\sum{x\in X\l\Omega\r}{}{x\p{X=x}}=\sum{\omega\in\Omega}{}{\p{\left\{\omega\right\}}X\l\omega\r}$}{42/53}
\slideq{Loi de Pascal}{43/53}
\slider{À faire}{43/53}
\slideq{Espérance conditionnelle}{44/53}
\slider{$\esp{X\sq A}=\sum{x\in X\l\Omega\r}{}{x\p{X=x\sq A}}$}{44/53}
\slideq{Variables indépendantes}{45/53}
\slider{$X\indep Y$\linebreak$\forall\l A_1,A_2\r\in\mathcal T_1\times\mathcal T_2$\linebreak$\p{X\in A_1,Y\in A_2}=\p{X\in A_1}\p{Y\in A_2}$}{45/53}
\slideq{Variable aléatoire discrète}{46/53}
\slider{$X\l\Omega\r$ est fini}{46/53}
\slideq{$\p[f\l X\r]{}$}{47/53}
\slider{$\p[X]{}\circ\widehat{f^{-1}}$}{47/53}
\slideq{$\cov XY$}{48/53}
\slider{$\esp{\l X-\esp X\r\l Y-\esp Y\r}$\linebreak${}=\esp{XY}-\esp X\esp Y$}{48/53}
\slideq{Variable centrée}{49/53}
\slider{$\esp X=0$}{49/53}
\slideq{Inégalité de Cauchy-Schwarz pour $\mathbb E$}{50/53}
\slider{$\left|\esp {XY}\right|\leqslant\sqrt{\esp{X^2}\esp{Y^2}}$}{50/53}
\slideq{Moment d'ordre $k$\linebreak Moment centré d'ordre $k$}{51/53}
\slider{$\esp{X^k}$\linebreak$\esp{\l X-\esp X\r^k}$}{51/53}
\slideq{Application mesurable}{52/53}
\slider{Si $\l E,\mathcal S\r$ et $\l F,\mathcal T\r$ sont deux espaces mesurables et $f\!:\! E\to F$\linebreak$\forall B\in\mathcal T,\;f^{-1}\l B\r\in\mathcal S$}{52/53}
\slideq{$\var X$}{53/53}
\slider{$\esp{\l X-\esp X\r^2}$}{53/53}
\end{document}
\documentclass[14pt,usepdftitle=false,aspectratio=169]{beamer}
\usepackage{preambule}
\setbeamercolor{structure}{fg=black}
\usepackage{probas}\usepackage{bigoperators}\usepackage{matrices}
\hypersetup{pdftitle=Probabilités -- Variables aléatoires}
\title{Probabilités\\\emph{Variables aléatoires}}
\author{}
\date{}
\begin{document}
\begin{frame}
    \titlepage
\end{frame}
\slideq{$\ect X$}{1/39}
\slider{$\sqrt{\var X}$}{1/39}
\slideq{Variable aléatoire réelle}{2/39}
\slider{Variable aléatoire à valeurs dans $\l\mathbb R,\bor^1\r$}{2/39}
\slideq{Moment d'ordre $k$\linebreak Moment centré d'ordre $k$}{3/39}
\slider{$\esp{X^k}$\linebreak$\esp{\l X-\esp X\r^k}$}{3/39}
\slideq{Loi d'une variable aléatoire}{4/39}
\slider{$\p[X]A=\p{X^{-1}\l A\r}$}{4/39}
\slideq{Loi conjointe de $\l X_1,\cdots,X_n\r$}{5/39}
\slider{$\p{X_1,\cdots,X_n}$ définie sur $\l\mathbb R^n,\bor^n\r$}{5/39}
\slideq{$\var{X_1+\cdots+X_n}$}{6/39}
\slider{$\sum{k=1}{n}{\var X_i}+2\sum{1\leqslant i<j\leqslant n}{}{\cov{X_i}{X_j}}$\linebreak${}=\tmatrix[][minimum width=0pt,minimum height=20pt,]({1\&\mdots\&1\\})\underline{\mathbb V}\l X_1,\cdots,X_n\r\tmatrix[][minimum width=0pt,minimum height=20pt,]({1\\\oldvdots\\1\\})$}{6/39}
\slideq{Structure des variables aléatoires de $\mathbb R^\Omega$}{7/39}
\slider{Sous-algèbre de $\mathbb R^\Omega$}{7/39}
\slideq{Variable centrée}{8/39}
\slider{$\esp X=0$}{8/39}
\slideq{Variables indépendantes}{9/39}
\slider{$X\indep Y$\linebreak$\forall\l A_1,A_2\r\in\mathcal T_1\times\mathcal T_2$\linebreak$\p{X\in A_1,Y\in A_2}=\p{X\in A_1}\p{Y\in A_2}$}{9/39}
\slideq{Cas d'égalité de l'inégalité de Cauchy-Schwarz pour $\oldcov$}{10/39}
\slider{Il existe $\l a,b\r\neq\l0,0\r$ tel que $aX+bY=c$ presque sûrement}{10/39}
\slideq{$L^1$}{11/39}
\slider{Variables aléatoires réelles discrètes admettant une espérance finie}{11/39}
\slideq{Si $X\!:\!\l\Omega,\mathcal T\r\to\l E;\mathcal T'\r$\linebreak$\mathcal T_X$}{12/39}
\slider{$\left\{X^{-1}\l A\r,\;A\in\mathcal T'\right\}$}{12/39}
\slideq{Lemme des coalitions}{13/39}
\slider{Si $\l X_1,\cdots,X_n\r$ sont mutuellement indépendantes, alors $f\l X_1,\cdots,X_m\r$ et $g\l X_{m+1},\cdots,X_n\r$}{13/39}
\slideq{Formule de polarisation}{14/39}
\slider{$\cov XY=\frac12\l\var{X+Y}-\var X-\var Y\r$}{14/39}
\slideq{$\var{\lambda X+\mu}$}{15/39}
\slider{$\lambda^2\var X$}{15/39}
\slideq{$\p[f\l X\r]{}$}{16/39}
\slider{$\p[X]{}\circ\widehat{f^{-1}}$}{16/39}
\slideq{Fonction de répartition de $X\!:\!\Omega\to\mathbb R$}{17/39}
\slider{$F_X\l x\r=\p{X\leqslant x}$}{17/39}
\slideq{$\esp X$}{18/39}
\slider{$\sum{x\in X\l\Omega\r}{}{x\p{X=x}}=\sum{\omega\in\Omega}{}{\p{\left\{\omega\right\}}X\l\omega\r}$}{18/39}
\slideq{$\p{f\l X\r=x}$}{19/39}
\slider{$\sum{y\in f^{-1}\l x\r\cap X\l\Omega\r}{}{\p{X=y}}$}{19/39}
\slideq{Espérance conditionnelle}{20/39}
\slider{$\esp{X\sq A}=\sum{x\in X\l\Omega\r}{}{x\p{X=x\sq A}}$}{20/39}
\slideq{Inégalité de Cauchy-Schwarz pour $\mathbb E$}{21/39}
\slider{$\left|\esp {XY}\right|\leqslant\sqrt{\esp{X^2}\esp{Y^2}}$}{21/39}
\slideq{Variable réduite}{22/39}
\slider{$\var X=1$}{22/39}
\slideq{$\var X$}{23/39}
\slider{$\esp{\l X-\esp X\r^2}$}{23/39}
\slideq{$\esp{XY}$}{24/39}
\slider{$\esp X\esp Y$ si $X\indep Y$}{24/39}
\slideq{Formule de l'espérance totale}{25/39}
\slider{Si $\l A_i\r$ est un système quasi-complet d'événements au plus dénombrale\linebreak$\esp{X}=\sum{i\in I}{}{\esp{X\sq A_i}\p{A_i}}$}{25/39}
\slideq{Variable aléatoire}{26/39}
\slider{Application mesurable $X\!:\!\l\Omega,\mathcal T\r\to\l E,\mathcal T'\r$\linebreak Si $\Omega'\in\mathcal T$ tel que $\p{\Omega'}=1$, on peut définir $X$ sur $\Omega'$}{26/39}
\slideq{Variables décorrélées}{27/39}
\slider{$\cov XY=0$}{27/39}
\slideq{Coefficient de corrélation}{28/39}
\slider{$\rho\l X,Y\r=\frac{\cov XY}{\ect X\ect Y}$}{28/39}
\slideq{Inégalité de Cauchy-Schwarz pour $\oldcov$}{29/39}
\slider{$\left|\cov XY\right|\leqslant\ect X\ect Y$}{29/39}
\slideq{$\cov XY$}{30/39}
\slider{$\esp{\l X-\esp X\r\l Y-\esp Y\r}$\linebreak${}=\esp{XY}-\esp X\esp Y$}{30/39}
\slideq{Matrice des variances-covariances}{31/39}
\slider{$\underline{\mathbb V}\l X_1,\cdots,X_n\r=\l\cov{X_i}{X_j}\r_{\l i,j\r\in\llb1,n\rrb^2}$}{31/39}
\slideq{$\var{X+Y}$}{32/39}
\slider{$\var X+\var Y+2\cov XY$}{32/39}
\slideq{Variable centrée réduite associée à $X$}{33/39}
\slider{$X^*=\frac{X-\esp X}{\ect X}$}{33/39}
\slideq{Formule de Koenig-Huyghens}{34/39}
\slider{$\var X=\esp{X^2}-\esp X^2$}{34/39}
\slideq{$\esp{\lambda X+Y}$}{35/39}
\slider{$\lambda\esp X+\esp Y$}{35/39}
\slideq{Application mesurable}{36/39}
\slider{Si $\l E,\mathcal S\r$ et $\l F,\mathcal T\r$ sont deux espaces mesurables et $f\!:\! E\to F$\linebreak$\forall B\in\mathcal T,\;f^{-1}\l B\r\in\mathcal S$}{36/39}
\slideq{$k$-ième loi marginale de $\l X_1,\cdots,X_n\r$}{37/39}
\slider{Loi de $X_k$}{37/39}
\slideq{Vecteur aléatoire réel}{38/39}
\slider{Vecteur aléatoire à valeurs dans $\l\mathbb R^n,\bor^n\r$}{38/39}
\slideq{Variable aléatoire discrète}{39/39}
\slider{$X\l\Omega\r$ est fini}{39/39}
\end{document}
\documentclass[14pt,usepdftitle=false,aspectratio=169]{beamer}
\usepackage{preambule}
\setbeamercolor{structure}{fg=black}
\newcommand\p{\mathbb P}\let\oldOmega\Omega\let\Omega\varOmega\usepackage{bigoperators}\let\oldlim\lim\renewcommand\lim[2]{\oldlim\limits_{#1}\l#2\r}\let\oldlims\limsup\newcommand\lims[1]{\oldlims\l#1\r}\let\oldlimi\liminf\newcommand\limi[1]{\oldlimi\l#1\r}
\hypersetup{pdftitle=Probabilités -- Espaces probabilisés}
\title{Probabilités\\\emph{Espaces probabilisés}}
\author{}
\date{}
\begin{document}
\begin{frame}
    \titlepage
\end{frame}
\slideq{$\sigma$-algèbre\linebreak Tribu}{1/13}
\slider{Une $\sigma$-algèbre $\mathcal T$ est un sous-ensemble de $\mathcal P\l \Omega\r$ vérifiant\linebreak $\Omega\in\mathcal{T}$\linebreak$A\in\mathcal T\Rightarrow\overline A\in\mathcal T$\linebreak Si $I$ est dénombrable et $\l A_i\r_{i\in I}$ une famille d'éléments de $\mathcal T$, $\bigcup{i\in I}{}{A_i}\in\mathcal T$}{1/13}
\slideq{$\limi{A_n}$}{2/13}
\slider{$\bigcup{n\in\mathbb N}{}{\bigcap{k=n}{+\infty}{A_k}}$}{2/13}
\slideq{$\p\l\bigcup{n\in\mathbb N}{}{A_n}\r$}{3/13}
\slider{$\lim{N\to+\infty}{\p\l\bigcup{n=0}{N}{A_n}\r}$}{3/13}
\slideq{Tribu engendrée par une famille}{4/13}
\slider{$\sigma\l\l A_i\r_{i\in I}\r$ avec $A_i$ des éléments de $\mathcal{P}\l\Omega\r$\linebreak Plus petite $\sigma$-algèbre de $\Omega$ contenant $\l A_i\r_{i\in I}$}{4/13}
\slideq{Intersection de tribus}{5/13}
\slider{Si $\l\mathcal T_i\r_{i\in I}$ est une famille de $\sigma$-algèbres sur $\Omega$, alors $\bigcap{i\in I}{}{\mathcal T_i}$ est une $\sigma$-algèbre sur $\Omega$}{5/13}
\slideq{$\p\l\bigcap{n\in\mathbb N}{}{A_n}\r$}{6/13}
\slider{$\lim{N\to+\infty}{\p\l\bigcap{n=0}{N}{A_n}\r}$}{6/13}
\slideq{Espace probabilisable}{7/13}
\slider{$\l \mathcal T,\Omega\r$\linebreak$\mathcal T$ est une $\sigma$-algèbre sur $\Omega$}{7/13}
\slideq{Mesure de probabilités}{8/13}
\slider{Application $\p\!:\:\mathcal T\to\mathbb R$ vérifiant\linebreak$0\leqslant\p\l A\r\leqslant 1$\linebreak$\p\l\varOmega\r=1$\linebreak$\p\l\biguplus{n\in\mathbb N}{}{A_n}\r=\sum{n\in\mathbb N}{}{\p\l A_n\r}$}{8/13}
\slideq{Système complet d'événements}{9/13}
\slider{Famille $\left\{A_i,\;i\in I\right\}$ formant une partition de $\Omega$}{9/13}
\slideq{$\p\l\bigcup{i=1}{n}{A_i}\r$}{10/13}
\slider{$\sum{\substack{I\subset\llb1,n\rrb\\I\neq\varnothing}}{}{\l-1\r^{\left|I\right|-1}\p\l\bigcap{i\in I}{}{A_i}\r}$}{10/13}
\slideq{Tribu des boréliens}{11/13}
\slider{$\mathcal B^1$ ou $\mathcal B$\linebreak$\sigma\l\l\left]-\infty,a\right[\r_{a\in\mathbb R}\r$\linebreak\linebreak$\mathcal B^1$ est aussi engendrée par n'importe quel type d'intervalle de $\mathbb R$}{11/13}
\slideq{Tribu des boréliens sur $\mathbb R^n$}{12/13}
\slider{$\mathcal B^n$\linebreak Tribu engendrée par les $I_1\times\cdots\times I_n$ où les $I_k$ sont des intervalles}{12/13}
\slideq{$\lims{A_n}$}{13/13}
\slider{$\bigcap{n\in\mathbb N}{}{\bigcup{k=n}{+\infty}{A_k}}$}{13/13}
\end{document}
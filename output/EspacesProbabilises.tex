\documentclass[14pt,usepdftitle=false,aspectratio=169]{beamer}
\usepackage{preambule}
\setbeamercolor{structure}{fg=black}
\newcommand\p{\mathbb P}\let\oldOmega\Omega\let\Omega\varOmega\usepackage{bigoperators}\let\oldlim\lim\renewcommand\lim[2]{\oldlim\limits_{#1}\l#2\r}\let\oldlims\limsup\newcommand\lims[1]{\oldlims\l#1\r}\let\oldlimi\liminf\newcommand\limi[1]{\oldlimi\l#1\r}\newcommand{\sq}{\;\middle|\;}
\hypersetup{pdftitle=Probabilités -- Espaces probabilisés}
\title{Probabilités\\\emph{Espaces probabilisés}}
\author{}
\date{}
\begin{document}
\begin{frame}
    \titlepage
\end{frame}
\slideq{$\lims{A_n}$}{1/26}
\slider{$\bigcap{n\in\mathbb N}{}{\bigcup{k=n}{+\infty}{A_k}}$}{1/26}
\slideq{$\p\l\bigcap{n\in\mathbb N}{}{A_n}\r$}{2/26}
\slider{$\lim{N\to+\infty}{\p\l\bigcap{n=0}{N}{A_n}\r}$}{2/26}
\slideq{Tribu des boréliens sur $\mathbb R^n$}{3/26}
\slider{$\mathcal B^n$\linebreak Tribu engendrée par les $I_1\times\cdots\times I_n$ où les $I_k$ sont des intervalles}{3/26}
\slideq{Espace probabilisé\linebreak Modèle probabiliste de Kolmogorov}{4/26}
\slider{$\l\Omega,\mathcal T,\p\r$ où $\l\Omega,\mathcal T\r$ est un espace probabilisable et $\p$ une mesure de probabilités}{4/26}
\slideq{Formule des probabilités totales pour le système complet $\l A,\overline A\r$}{5/26}
\slider{$\p\l B\r=\p\l A\r\p\l B\sq A\r+\p\l\overline A\r\p\l B\sq\overline A\r$}{5/26}
\slideq{Formule des probabilités composées}{6/26}
\slider{Si $\p\l \bigcap{i=1}{n-1}{A_i}\r\neq0$\linebreak$\p\l \bigcap{i=1}{n}{A_i}\r=\p\l A_1\r\prod{i=2}{n}{\p\l A_i\sq\bigcap{j=1}{i-1}{A_i}\r}$}{6/26}
\slideq{Système complet d'événements}{7/26}
\slider{Famille $\left\{A_i,\;i\in I\right\}$ formant une partition de $\Omega$}{7/26}
\slideq{Espace probabilisable}{8/26}
\slider{$\l \mathcal T,\Omega\r$\linebreak$\mathcal T$ est une $\sigma$-algèbre sur $\Omega$}{8/26}
\slideq{$\sigma$-algèbre\linebreak Tribu}{9/26}
\slider{Une $\sigma$-algèbre $\mathcal T$ est un sous-ensemble de $\mathcal P\l \Omega\r$ vérifiant\linebreak $\Omega\in\mathcal{T}$\linebreak$A\in\mathcal T\Rightarrow\overline A\in\mathcal T$\linebreak Si $I$ est dénombrable et $\l A_i\r_{i\in I}$ une famille d'éléments de $\mathcal T$, $\bigcup{i\in I}{}{A_i}\in\mathcal T$}{9/26}
\slideq{Formule de Bayes sur un système complet}{10/26}
\slider{Si $\l A_i\r_{i\in I}$ est un système quasi-complet au plus dénombrable tel que pour tout $i\in I$, $\p\l A_i\r\neq0$ et $\p\l B\r\neq0$\linebreak$\p\l A_j\sq B\r=\frac{\p\l B\sq A_j\r\p\l A_j\r}{\oldsum\limits_{i\in I}\l\p\l B\sq A_i\r\p\l A_i\r\r}$}{10/26}
\slideq{Intersection de tribus}{11/26}
\slider{Si $\l\mathcal T_i\r_{i\in I}$ est une famille de $\sigma$-algèbres sur $\Omega$, alors $\bigcap{i\in I}{}{\mathcal T_i}$ est une $\sigma$-algèbre sur $\Omega$}{11/26}
\slideq{Distribution de probabilités}{12/26}
\slider{Famille $\l p_i\r_{i\in I}$ tel que $\sum{i\in I}{}{p_i}=1$}{12/26}
\slideq{$A$ et $B$ sont indépendants}{13/26}
\slider{$\p\l A\cap B\r=\p\l A\r\p\l B\r$}{13/26}
\slideq{$\p_b\l A\r=\p\l A\sq B\r$}{14/26}
\slider{$\frac{\p\l A\cap B\r}{\p\l B\r}$}{14/26}
\slideq{Formule des probabilités totales associée à une variable aléatoire réelle discrète}{15/26}
\slider{$\p\l B\r=\sum{x\in X\l\Omega\r}{}{\p\l X=x\r\p\l B\sq X=x\r}$}{15/26}
\slideq{$\p\l\bigcup{i=1}{n}{A_i}\r$}{16/26}
\slider{$\sum{\substack{I\subset\llb1,n\rrb\\I\neq\varnothing}}{}{\l-1\r^{\left|I\right|-1}\p\l\bigcap{i\in I}{}{A_i}\r}$}{16/26}
\slideq{Formule de Bayes simple}{17/26}
\slider{Si $\p\l A\r\neq0$ et $\p\l B\r\neq0$\linebreak$\p\l A\sq B\r=\frac{\p\l B\sq A\r\p\l A\r}{\p\l B\r}$}{17/26}
\slideq{Les $A_i$, $i\in I$ sont mutuellement indépendants}{18/26}
\slider{$\forall J\in\mathcal P_f\l I\r,\;\p\l\bigcap{j\in J}{}{A_j}\r=\prod{j\in J}{}{\p\l A_j\r}$}{18/26}
\slideq{$\p\l\bigcup{n\in\mathbb N}{}{A_n}\r$}{19/26}
\slider{$\lim{N\to+\infty}{\p\l\bigcup{n=0}{N}{A_n}\r}$}{19/26}
\slideq{Tribu des boréliens}{20/26}
\slider{$\mathcal B^1$ ou $\mathcal B$\linebreak$\sigma\l\l\left]-\infty,a\right[\r_{a\in\mathbb R}\r$\linebreak\linebreak$\mathcal B^1$ est aussi engendrée par n'importe quel type d'intervalle de $\mathbb R$}{20/26}
\slideq{$\limi{A_n}$}{21/26}
\slider{$\bigcup{n\in\mathbb N}{}{\bigcap{k=n}{+\infty}{A_k}}$}{21/26}
\slideq{Formule des probabilités totales}{22/26}
\slider{Si $\l A_i\r_{i\in I}$ est un système quasi-complet au plus dénombrable\linebreak$\p\l B\r=\sum{i\in I}{}{\p\l A_i\r\p\l B\sq A_i\r}$}{22/26}
\slideq{Mesure de probabilités}{23/26}
\slider{Application $\p\!:\:\mathcal T\to\mathbb R$ vérifiant\linebreak$0\leqslant\p\l A\r\leqslant 1$\linebreak$\p\l\varOmega\r=1$\linebreak$\p\l\biguplus{n\in\mathbb N}{}{A_n}\r=\sum{n\in\mathbb N}{}{\p\l A_n\r}$}{23/26}
\slideq{$\p\l\left\{\omega\right\}\r$}{24/26}
\slider{$\frac1{\left|\Omega\right|}$}{24/26}
\slideq{Tribu engendrée par une famille}{25/26}
\slider{$\sigma\l\l A_i\r_{i\in I}\r$ avec $A_i$ des éléments de $\mathcal{P}\l\Omega\r$\linebreak Plus petite $\sigma$-algèbre de $\Omega$ contenant $\l A_i\r_{i\in I}$}{25/26}
\slideq{Système quasi-complet d'événements}{26/26}
\slider{$\mathcal{C}$ est quasi-complet si\linebreak Les événements de $\mathcal C$ ne sont pas impossibles\linebreak Les événements de $\mathcal C$ sont deux à deux disjoints\linebreak$\sum{A\in\mathcal C}{}{\p\l A\r}=1$}{26/26}
\end{document}
\documentclass[14pt,usepdftitle=false,aspectratio=169]{beamer}
\usepackage{preambule}
\setbeamercolor{structure}{fg=black}
\usepackage{dsfont}\usepackage{bigoperators}\usepackage{structures,arithmetique}
\hypersetup{pdftitle=Algèbre 1 -- Groupes}
\title{Algèbre 1\\\emph{Groupes}}
\author{}
\date{}
\begin{document}
\begin{frame}
    \titlepage
\end{frame}
\slideq{Si $\l G,\star\r$ est un groupe et $H\subset G$\linebreak Caractérisation(s) des sous-groupes}{1/43}
\slider{$H\neq\varnothing$\quad$\forall\l x,y\r\in H,\;x\star y\in H$\quad$\forall x\in H,\;x^s\in H$\linebreak$H\neq\varnothing$\quad$\forall\l x,y\r\in H^2,\;x\star y^s\in H$\linebreak$e_G\in H$\quad$\forall\l x,y\r\in H^2,\;x\star y^s\in H$}{1/43}
\slideq{Endomorphisme de X}{2/43}
\slider{Homomorphisme de X de $E$ dans lui-même (muni des mêmes lois)}{2/43}
\slideq{Si $\l G,\star\r$ est un groupe\linebreak Un sous-ensemble $H$ de $G$ est appelé sous-groupe de $G$}{3/43}
\slider{$H$ est stable pour la loi de $G$ et la loi induite définit sur $H$ une structure de groupe}{3/43}
\slideq{Ordre d'un élément d'un groupe}{4/43}
\slider{$\ord{x}=\min\left(\left\{n\in\mathbb{N}^*\;|\;x^n=e\right\}\right)$}{4/43}
\slideq{Ordre d'un groupe\linebreak Si $G$ est un groupe}{5/43}
\slider{$\ord{G}=\left|G\right|$}{5/43}
\slideq{$x$ et $y$ sont dans la même classe à gauche modulo $H$}{6/43}
\slider{$x\equiv_gy\,\left[H\right]\Leftrightarrow x^{-1}y\in H$}{6/43}
\slideq{Théorème de Lagrange pour l'ordre des groupes}{7/43}
\slider{Si $G$ est un groupe fini et $H$ un sous-groupe de $G$\linebreak$\left|H\right|\div\left|G\right|$}{7/43}
\slideq{Isomorphisme de X}{8/43}
\slider{Homomorphisme de X bijectif}{8/43}
\slideq{Premier théorème d'isomorphisme}{9/43}
\slider{Si $f\in\hom{G,H}$\linebreak$\ker{f}$ est un sous-groupe distingué de $G$, et $f$ passe au quotient, définissant un morphisme de groupes $\tilde{f}\!:\!G/\ker{f}\to H$\linebreak $\tilde{f}$ est injectif et sa corestriction à son image est un isomorphisme}{9/43}
\slideq{Les classes à gauche modulo $H$}{10/43}
\slider{$\left\{aH,\;a\in G\right\}$}{10/43}
\slideq{Si $\ker{f}=\left\{e_G\right\}$}{11/43}
\slider{$f$ est injectif (la réciproque est vraie)}{11/43}
\slideq{Si $H$ est un sous-groupe distingué de $G$}{12/43}
\slider{$\forall a\in G,\;aH=Ha$\linebreak${}\Leftrightarrow\forall a\in G,\;\forall h\in H,\;aha^{-1}\in H$}{12/43}
\slideq{Si $G$ et $H$ sont deux groupes et $f\in\hom{G,H}$\linebreak$f\l x^{-1}\r$}{13/43}
\slider{$f\l x\r^{-1}$}{13/43}
\slideq{Image directe et réciproque de sous-groupes par un homomorphisme}{14/43}
\slider{Si $G$ et $H$ sont deux groupes, et $f\in\hom{G,H}$ un morphisme de groupes, $G'$ et $H'$ deux sous-groupes respectivement de $G$ et $H$\linebreak $f\l G'\r$ est un sous-groupe de $H$\linebreak $f^{-1}\l H'\r$ est un sous-groupe de $G$}{14/43}
\slideq{Si $G$ est un gruope\linebreak Structure de $\l\aut G,\circ\r$}{15/43}
\slider{$\l\aut G,\circ\r$ est un groupe}{15/43}
\slideq{Sous-groupe monogène}{16/43}
\slider{$\la x\ra=\left\{x^n,\;n\in\mathbb{N}\right\}$}{16/43}
\slideq{Description des groupes monogènes\linebreak Si $G=\la x\ra$}{17/43}
\slider{Si $\ord{x}=+\infty$, $G$ est isomorphe à $\mathbb{Z}$\linebreak Si $\ord{x}=n\in\mathbb{N}^*$, $G$ est isomorphe à $\mathbb{Z}/n\mathbb{Z}$}{17/43}
\slideq{Si $f\in\hom{G,K}$ et $H$ est un sous-groupe distingué et $H\subset\ker{f}$}{18/43}
\slider{$f=\tilde{f}\circ\pi$\linebreak La réciproque est vraie}{18/43}
\slideq{Groupe abélien}{19/43}
\slider{La loi $\star$ de $G$ est commutative}{19/43}
\slideq{Réciproque d'isomorphisme}{20/43}
\slider{Si $f\!:\!F\to F$ est un isomorphisme, alors $f^{-1}$ est un isomorphisme}{20/43}
\slideq{Propriété des groupes monogènes}{21/43}
\slider{Un groupe monogène est abélien}{21/43}
\slideq{Sous-groupe propre de $G$}{22/43}
\slider{Sous-groupe de $G$ distinct de $G$ et $\left\{e_G\right\}$}{22/43}
\slideq{Résolution de $x^n=e$}{23/43}
\slider{$\left\{n\in\mathbb{N}^*\;|\;x^n=e\right\}$ est de la forme $a\mathbb{Z}$\linebreak$x$ est d'ordre fini si et seulement si $a\neq0$ (on a donc $\ord{x}=a$)}{23/43}
\slideq{Description par le bas du sous-groupe engendré par une partie}{24/43}
\slider{$\la X\ra=\left\{x_1\cdots x_n,\;\l x_1,\cdots,x_n\r\in X^n\right\}$\linebreak${}\cup\left\{x^{-1},\;x\in X\right\}$\linebreak $e$ correspond au produit vide}{24/43}
\slideq{Propriétés d'un groupe $\l G,\star\r$}{25/43}
\slider{$G$ admet un uique élément neutre pour $\star$\linebreak$\forall x\in G,\;\exists!x^s\in G$}{25/43}
\slideq{Passage au quotient de la loi dans le cas d'un sous-groupe distingué\linebreak Si $G$ est un groupe et $H$ un sous-groupe distingué de $G$}{26/43}
\slider{${\equiv_g}={\equiv_d}$ et on note la relation $\equiv$\linebreak La loi induite corrrespond au produit des classes élément par élément\linebreak$\l ab\r H=\l aH\r\cdot\l bH\r$\linebreak${}=\left\{x\cdot y,\;x\in aH,\;y\in bH\right\}$\linebreak La loi induite sur l'ensemble quotient munit celui-ci d'une structure de groupe}{26/43}
\slideq{Les classes à droite modulo $H$}{27/43}
\slider{$\left\{Ha,\;a\in G\right\}$}{27/43}
\slideq{Soient $\l G,\star\r$ et $\l H,\diamond\r$ deux groupes\linebreak$f\!:\!G\to H$ est un homomorphisme de groupe}{28/43}
\slider{$\forall\l x,y\r\in G^2,\;f\l x\star y\r=f\l x\r\diamond f\l y\r$\linebreak L'ensemble des homomorphisme de $G$ dans $H$ est noté $\hom{G,H}$\linebreak Si $\l G,\star\r=\l H,\diamond\r$, f est un endomorphisme\linebreak L'ensemble des automorphismes de $G$ est noté $\aut{G}$}{28/43}
\slideq{Groupe cyclique}{29/43}
\slider{Groupe monogène fini}{29/43}
\slideq{Fibres de $f$\linebreak Soit $x\in f^{-1}\l\left\{y\right\}\r$}{30/43}
\slider{$f^{-1}\l\left\{y\right\}\r=x\times\ker{f}$\linebreak${}=\left\{x\times z,\;z\in\ker{f}\right\}=\ker{f}\times x$}{30/43}
\slideq{Groupe}{31/43}
\slider{Muni d'une loi d'une composition interne, de l'associativité, d'un élément neutre et de symétriques}{31/43}
\slideq{Ensemble formé par les classes à gauche et à droite}{32/43}
\slider{$\left\{Ha,\;a\in G\right\}$ est une partition de $G$\linebreak$\left\{aH,\;a\in G\right\}$ est une partition de $G$}{32/43}
\slideq{Si $G$ et $H$ sont deux groupes et $f\in\hom{G,H}$\linebreak$f\l e_G\r$}{33/43}
\slider{$f\l e_H\r$}{33/43}
\slideq{Si $f\in\hom{G,K}$ et $H$ est un sous-groupe distingué}{34/43}
\slider{$f$ passe au quotient avec $\tilde{f}\!:\!G/H\to K$}{34/43}
\slideq{Sous-groupe engendrée par une partie $X$}{35/43}
\slider{$\la X\ra$\linebreak C'est le plus petit sous-groupe contenant $X$}{35/43}
\slideq{Théorème de Lagrange pour l'ordre des éléments d'un groupe}{36/43}
\slider{Si $G$ est un groupe fini et $x\in G$\linebreak$\ord{x}\div\left|G\right|$}{36/43}
\slideq{Cardinal des classes de congruence}{37/43}
\slider{$\left|Ha,\;a\in G\right|=\left|Ha,\;a\in G\right|=\left|H\right|$}{37/43}
\slideq{Description par le haut du sous-groupe engendré par une partie}{38/43}
\slider{Soient $\mathcal{G}$ l'ensemble des sous-groupes de $G$ et $\mathcal{H}=\left\{H\in\mathcal{G}\;|\;X\subset H\right\}$\linebreak$\la X\ra=\bigcap{H\in\mathcal{H}}{}{H}$}{38/43}
\slideq{Automorphisme de X}{39/43}
\slider{Endomorphisme et isomorphisme de X}{39/43}
\slideq{Si $G$ et $H$ sont deux groupes et $f\in\hom{g,h}$ un morphisme de groupes\linebreak$\ker{f}$}{40/43}
\slider{$f^{-1}\l e_H\r=\left\{y\in G\;|\;f\l y\r=e_H\right\}$}{40/43}
\slideq{Passage au quotient de la loi dans le cas abélien\linebreak Si $G$ est un groupe abélien et $H$ un sous-groupe de $G$}{41/43}
\slider{${\equiv_g}={\equiv_d}$ et on note la relation $\equiv$\linebreak La loi induite corrrespond au produit des classes élément par élément\linebreak$\l ab\r H=\l aH\r\cdot\l bH\r$\linebreak${}=\left\{x\cdot y,\;x\in aH,\;y\in bH\right\}$\linebreak La loi induite sur l'ensemble quotient munit celui-ci d'une structure de groupe abélien}{41/43}
\slideq{$x$ et $y$ sont dans la même classe à droite modulo $H$}{42/43}
\slider{$x\equiv_dy\,\left[H\right]\Leftrightarrow xy^{-1}\in H$}{42/43}
\slideq{Intersection de sous-groupes\linebreak Si $G$ est un groupe, et $\l H_i\r_{i\in I}$ une famille de sous-groupes de $G$}{43/43}
\slider{$\bigcap{i\in I}{}{H_i}$ est un sous-groupe de $G$}{43/43}
\end{document}
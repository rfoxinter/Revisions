\documentclass[14pt,usepdftitle=false,aspectratio=169]{beamer}
\usepackage{preambule}
\setbeamercolor{structure}{fg=black}
\DeclareSymbolFont{cmsy}{OMS}{cmsy}{m}{n}\DeclareMathSymbol{\R}{\mathbin}{cmsy}{"52}
\hypersetup{pdftitle=Fondements -- Relations}
\title{Fondements\\\emph{Relations}}
\author{}
\date{}
\begin{document}
\begin{frame}
    \titlepage
\end{frame}
\slideq{Antisymértie}{1/11}
\slider{$\l x\R y\r\wedge\l y\R x\r\Rightarrow\l x=y\r$}{1/11}
\slideq{Symétrie}{2/11}
\slider{$x\R y\Rightarrow y\R x$}{2/11}
\slideq{Théorème de la factorisation d'une application constante sur les classes d'équivalences}{3/11}
\slider{$\l\forall\l x,y\r\in E^2,\;x\R y\Rightarrow f\l x\r=f\l y\r\r\linebreak\Leftrightarrow\l\exists g\!:\!E/\R\to F\;\middle|\;f=g\circ\pi_\R\r$}{3/11}
\slideq{Relation d'équivalence}{4/11}
\slider{ Réflexive, symétrique et transitive\linebreak Notée $\equiv$ ou $\sim$}{4/11}
\slideq{Asymétrie}{5/11}
\slider{$\l x\R y\r\Rightarrow\neg\l y\R x\r$}{5/11}
\slideq{Relation d'ordre large}{6/11}
\slider{ Réflexive, antisymétrique et transitive\linebreak Notée $\leqslant$ ou $\geqslant$}{6/11}
\slideq{Irréflixivité ou antiréfléxivité}{7/11}
\slider{$\neg\l x\R x\r$}{7/11}
\slideq{Transitivité}{8/11}
\slider{$\l x\R y\r\wedge\l y\R z\r\Rightarrow\l x\R z\r$}{8/11}
\slideq{Réflexivité}{9/11}
\slider{$x\R x$}{9/11}
\slideq{Relation d'ordre strict}{10/11}
\slider{ Irréflexive et transitive\linebreak Notée $<$ ou $>$}{10/11}
\slideq{La relation d'équivalence $\R$ est une congruence sur $\l E,\times_1,\cdots,\times_n\r$}{11/11}
\slider{$\forall\l x,x^\prime,y,y^\prime\r\in E^4,\;\forall i\in\llb1,n\rrb\linebreak\l x\R x^\prime\r\wedge\l y\R y^\prime\r\Rightarrow\l x\times_iy\r\R\l x^\prime\times_iy^\prime\r$}{11/11}
\end{document}
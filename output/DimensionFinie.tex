\documentclass[14pt,usepdftitle=false,aspectratio=169]{beamer}
\usepackage{preambule}
\setbeamercolor{structure}{fg=black}
\usepackage{al}\usepackage{bigoperators}\DeclareMathOperator{\oldim}{im}\newcommand{\im}[1]{\oldim\l#1\r}\let\oldmin\min\renewcommand{\min}[1]{\oldmin\l#1\r}\let\oldker\ker\renewcommand{\ker}[1]{\oldker\l#1\r}\renewcommand\b{\mathcal{B}}\renewcommand\c{\mathcal{C}}\renewcommand\d{\mathcal{D}}\DeclareMathOperator{\id}{id}
\hypersetup{pdftitle=Algèbre 2 -- Dimension finie}
\title{Algèbre 2\\\emph{Dimension finie}}
\author{}
\date{}
\begin{document}
\begin{frame}
    \titlepage
\end{frame}
\slideq{Matrices équivalentes}{1/30}
\slider{$N\in\mat n{}K$, $M\in\mat n{}K$\linebreak$\exists\l P,Q\r\in\matgl nK^2,\;N=PMQ$}{1/30}
\slideq{Dual}{2/30}
\slider{$E^*$\linebreak$\al E{\mathbb K}$ constitué des formes linéaires}{2/30}
\slideq{Matrice associée à une composition}{3/30}
\slider{$\almat{g\circ f}\b\d=\almat g\c\d\times\almat f\b\c$}{3/30}
\slideq{Image d'une matrice}{4/30}
\slider{$\im M=\vect{\mathrm C_1\l M\r,\cdots,\mathrm C_n\l M\r}$}{4/30}
\slideq{Conservation du rang pour les matrices}{5/30}
\slider{$M\in\mat npK$, $P\in\matgl nK$, $Q\in\matgl nK$\linebreak$\rg{PMQ}=\rg M$}{5/30}
\slideq{Trace d'un projecteur et d'une symétrie}{6/30}
\slider{$\tr p=\rg p$\linebreak$\tr s=n-2\rg{s-\id}$}{6/30}
\slideq{Dimension d'une somme directe}{7/30}
\slider{$\dim{\bigop{i=1}n{E_i}}=\sum{i=1}n{\dim{E_i}}$}{7/30}
\slideq{Dimension d'un supplémentaire $S$ de $F$ dans $E$}{8/30}
\slider{$\dim S=\dim E-\dim F$}{8/30}
\slideq{Matrice diagonalisable}{9/30}
\slider{Matrice semblable à une matrice diagonale}{9/30}
\slideq{Formule de Grassmann}{10/30}
\slider{$\dim{E+F}=\dim E+\dim F-\dim{E\cap F}$}{10/30}
\slideq{$I_{n,p,r}$}{11/30}
\slider{\tmatrix[\mtxvline{line width = 0.05em}{1}\mtxhline{line width = 0.05em}{1}][minimum height = 5ex, row sep = 1ex,minimum width = 5ex, column sep = 1ex,]({I_r\&0_{r,p-r}\\0_{n-r,r}\&0_{n-r,p-r}\\})}{11/30}
\slideq{Endomorphisme diagonalisable}{12/30}
\slider{Il existe une base $\b$ dans laquelle $\almat u\b{}$ est diagonale}{12/30}
\slideq{Matrices semblables}{13/30}
\slider{$A\in\mat n{}K$, $B\in\mat n{}K$\linebreak$\exists\l P\r\in\matgl nK,\;B=P^{-1}AP$}{13/30}
\slideq{Rang d'une famille}{14/30}
\slider{$\rg{x_1,\cdots,x_n}=\dim{\vect{x_1,\cdots,x_n}}$}{14/30}
\slideq{Trace d'une matrice}{15/30}
\slider{$A\in\mat n{}K$\linebreak$\tr A=\sum{i=1}n{\lc A\rc_{i,i}}$}{15/30}
\slideq{Hyperplan}{16/30}
\slider{$\exists\in E^*\setminus\left\{0\right\},\;H=\ker\varphi$\linebreak$\varphi$ est l'équation caractéristique de $H$}{16/30}
\slideq{Majoration du rang d'une application linéaire $u\in\al EF$}{17/30}
\slider{$\rg u\leqslant\min{\dim E,\dim F}$}{17/30}
\slideq{$\almat u\b{}$ est inversible}{18/30}
\slider{$u\in\gl E$\linebreak La réciproque est vraie}{18/30}
\slideq{Matrice de passage}{19/30}
\slider{$P_{\b_1}^{\b_2}=\almat \id{\b_2}{\b_1}=\lc\b_2\rc_{\b_1}$}{19/30}
\slideq{Rang d'une application linéaire}{20/30}
\slider{$\rg u=\dim{\im u}$}{20/30}
\slideq{Classification des matrices équivalentes par le rang}{21/30}
\slider{$N$ est équivalent à $M$ si et seulement si $\rg M=\rg N$}{21/30}
\slideq{Produit matriciel avec l'évaluation}{22/30}
\slider{$\lc f\l X\r\rc_\c=\almat f\b\c\lc X\rc_\b$}{22/30}
\slideq{Effet d'une composition sur le rang}{23/30}
\slider{$\rg{v\circ u}\leqslant\min{\rg u,\rg v}$\linebreak Si $v$ est injective, $\rg{v\circ u}=\rg u$\linebreak Si $u$ est surjective, $\rg{v\circ u}=\rg v$}{23/30}
\slideq{Théorème du rang}{24/30}
\slider{$\dim{\ker f}+\rg f=\dim E$}{24/30}
\slideq{Propriétés de la trace}{25/30}
\slider{C'est une forme linéaire\linebreak$\tr A=\tr{A^\top}$\linebreak$\tr{AB}=\tr{BA}$\linebreak Si $M$ et $N$ sont semblables, $\tr N=\tr M$}{25/30}
\slideq{Conservation de l'image et du noyau pour les matrices}{26/30}
\slider{$M\in\mat npK$, $P\in\matgl nK$, $Q\in\matgl nK$\linebreak$\ker{PM}=\ker M$\linebreak$\im{MQ}=\im M$}{26/30}
\slideq{Forme linéaire}{27/30}
\slider{Application linéaire (sur un $\mathbb K$-espace vectoriel $E$) de $E$ vers $\mathbb K$\linebreak Un élément de $\al E{\mathbb K}$}{27/30}
\slideq{Dimension de $\al EF$}{28/30}
\slider{$\dim{\al EF}=\dim E\times\dim F$}{28/30}
\slideq{Dimension d'un produit cartésien}{29/30}
\slider{$\dim{E\times F}=\dim E+\dim F$}{29/30}
\slideq{Formule de changement de base}{30/30}
\slider{$\b_1$, $\b_2$ des bases de $E$, $\c_1$, $\c_2$ des bases de $F$\linebreak$\almat f{\b_2}{\c_2}=P_{\c_2}^{\c_1}\almat f{\b_1}{\c_1}P_{\b_1}^{\b_2}$\linebreak${}=\l P_{\c_1}^{\c_2}\r^{-1}\almat f{\b_1}{\c_1}P_{\b_1}^{\b_2}$}{30/30}
\end{document}
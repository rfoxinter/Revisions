\documentclass[14pt,usepdftitle=false,aspectratio=169]{beamer}
\usepackage{preambule}
\setbeamercolor{structure}{fg=black}
\usepackage{analyse}\usepackage{complexes}\usepackage{trigo}
\hypersetup{pdftitle=Analyse -- Équations différentielles}
\title{Analyse\\\emph{Équations différentielles}}
\author{}
\date{}
\begin{document}
\begin{frame}
    \titlepage
\end{frame}
\slideq{Solutions de l'équation $y^\prime=a\l x\r y$}{1/12}
\slider{$x\mapsto C\e^{\l\oldint{\l a\r}\r\l x\r}$}{1/12}
\slideq{Solutions de l'équation différentielle $y^\prime{}^\prime+ay^\prime+by=0$ si $\varDelta=0$ où $\varDelta$ est le discriminant dy polynôme caractéristique et $r$ sa racine double}{2/12}
\slider{$\mathcal{S}=\left\{y\!:\!x\mapsto\l c+dx\r\e^{rx}\right\},\;\l c,d\r\in\mathbb{C}^2$}{2/12}
\slideq{Solutions réelles de l'équation différentielle $y^\prime{}^\prime+ay^\prime+by=0$ si $\varDelta=0$ où $\varDelta$ est le discriminant dy polynôme caractéristique et $r$ sa racine double}{3/12}
\slider{$\mathcal{S}=\left\{y\!:\!x\mapsto\l c+dx\r\e^{rx}\right\},\;\l c,d\r\in\mathbb{R}^2$}{3/12}
\slideq{Solutions de l'équation $y^\prime=ay+b$}{4/12}
\slider{$\mathcal{S}=\left\{y\!:\!x\mapsto C\e^{ax}-\frac{b}{a}\right\}$}{4/12}
\slideq{Théorème de Cauchy-Lipschitz pour les équations différentielles linéaires d'ordre 1}{5/12}
\slider{Il existe une unique solution $y$ de l'équation différentielle $y^\prime=a\l x\r y+b\l x\r$ telle que $y\l x_0\r=y_0$}{5/12}
\slideq{Théorème de Cauchy-Lipschitz pour les équations différentielles linéaires d'ordre 2\linebreak Soit $f\l x\r$ continue}{6/12}
\slider{Il existe une unique solution $y$ de l'équation différentielle $y^\prime{}^\prime+ay+b=f\l x\r\l x\r$ telle que $y\l x_0\r=y_0$ et $y^\prime\l x_0\r=y_1$}{6/12}
\slideq{Solutions réelles de l'équation différentielle $y^\prime{}^\prime+ay^\prime+by=0$ si $\varDelta>0$ où $\varDelta$ est le discriminant dy polynôme caractéristique et $r$ et $s$ sont ses racines}{7/12}
\slider{$\mathcal{S}=\left\{y\!:\!x\mapsto c\e^{rx}+d\e^{sx}\right\},\;\l c,d\r\in\mathbb{R}^2$}{7/12}
\slideq{Solutions réelles de l'équation différentielle $y^\prime{}^\prime+ay^\prime+by=0$ si $\varDelta<0$ où $\varDelta$ est le discriminant dy polynôme caractéristique, $\omega=\oldfrac{\sqrt{-\varDelta}}{2}$ et $\alpha=\oldfrac{-a}{2}$}{8/12}
\slider{$\mathcal{S}=\left\{y\!:\!x\mapsto\e^{\alpha x}\l c\cos{\omega x}+d\sin{\omega x}\r\right\}$\linebreak$\l c,d\r\in\mathbb{R}^2$}{8/12}
\slideq{Solution de l'équation $y^\prime=ay+b$ telle que $y\l x_0\r=y_0$}{9/12}
\slider{$y\!:\!x\mapsto\l\frac{b}{a}+y_0\r\e^{a\l x-x_0\r}-\frac{b}{a}$}{9/12}
\slideq{Solution particulière $y_P$ pour un second membre $Q\l x\r\e^{\lambda x}$ avec $m$ la muntiplicité de $\lambda$ comme racine du polynôme caractéristique de l'équation différentielle}{10/12}
\slider{$x^mR\l x\r\e^{\lambda x}$}{10/12}
\slideq{$\mathcal{S}_\mathbb{R}$}{11/12}
\slider{$\left\{\Re\l y\r,\;y\in\mathcal{S}_\mathbb{C}\right\}$}{11/12}
\slideq{Solutions de l'équation différentielle $y^\prime{}^\prime+ay^\prime+by=0$ si $\varDelta\neq0$ où $\varDelta$ est le discriminant dy polynôme caractéristique et $r$ et $s$ sont ses racines}{12/12}
\slider{$\mathcal{S}=\left\{y\!:\!x\mapsto c\e^{rx}+d\e^{sx}\right\},\;\l c,d\r\in\mathbb{C}^2$}{12/12}
\end{document}
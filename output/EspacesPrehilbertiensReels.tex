\documentclass[14pt,usepdftitle=false,aspectratio=169]{beamer}
\usepackage{preambule}
\setbeamercolor{structure}{fg=black}
\let\phi\varphi\usepackage{al}\usepackage{structures}\usepackage{bigoperators}
\hypersetup{pdftitle=Algèbre 2 -- Espaces préhilbertiens réels}
\title{Algèbre 2\\\emph{Espaces préhilbertiens réels}}
\author{}
\date{}
\begin{document}
\begin{frame}
    \titlepage
\end{frame}
\slideq{Groupe orthogonal}{1/40}
\slider{$\orth[n]{\mathbb R}=\left\{P\in\mat{n}{}{\mathbb{R}}\;\middle|\;P^\top P=I_n\right\}$}{1/40}
\slideq{Projeté orthogonal sur un sous-espace vectoriel}{2/40}
\slider{$z$ est le projeté orthogonal de $y$ sur $F$ si et seulement si $z\in F$ et $\l y-z\r\perp F$}{2/40}
\slideq{Espace euclidien}{3/40}
\slider{Espace préhilbertien réel de dimension finie}{3/40}
\slideq{Expression de $\la x,y\ra$ dans la base orthonormée $\mathcal{B}=\l b_1,\cdots,b_n\r$}{4/40}
\slider{$\la x,y\ra=\sum{i=1}{n}{\la x,b_i\ra\la y,b_i\ra}$}{4/40}
\slideq{Expression matricielle de $\phi\l x,y\r$}{5/40}
\slider{$\phi\l x,y\r=\lc x\rc_{\mathcal B}^\top\almat\phi{\mathcal B}{}\lc y\rc_{\mathcal B}$}{5/40}
\slideq{$d\l x,F\r$}{6/40}
\slider{$\min\limits_{y\in F}\l\left\|x-y\right\|\r=\left\|x-p_F\l x\r\right\|$}{6/40}
\slideq{Supplémentaire orthogonal}{7/40}
\slider{En dimension finie, tout sev $F$ de $E$ admet un unique supplémentaire $F^\perp$ tel que $F\perp F^\perp$ et $F\oplus F^\perp=E$}{7/40}
\slideq{Matrice orthogonale}{8/40}
\slider{$P\in\mat{n}{}{\mathbb R}$ est orthogonale si et seulement si $P^\top P=I_n$}{8/40}
\slideq{Norme}{9/40}
\slider{$\forall x\in E,\;N\l x\r=0\Leftrightarrow x=0$ (séparation)\linebreak$\forall\l\lambda,x\r\in\mathbb R\times E,\;N\l\lambda x\r=\left|\lambda\right|N\l x\r$ (absolue homogénéité)\linebreak$\forall\l x,y\r\in E^2,\;N\l x+y\r\leqslant N\l x\r+N\l y\r$ (inégalité triangulaire)}{9/40}
\slideq{Inégalité de Cauchy-Schwarz pour un produit scalaire\linebreak Cas d'égalité}{10/40}
\slider{$\left|\la x,y\ra\right|\leqslant\left\|x\right\|\times\left\|y\right\|$\linebreak Égalité si et seulement si $x$ et $y$ sont colinéaires}{10/40}
\slideq{Projeté orthogonal sur un sous-espace vectoriel en dimension finie}{11/40}
\slider{$z=\sum{i=1}{m}{\la y,b_i\ra b_i}$}{11/40}
\slideq{$\phi$ est négative}{12/40}
\slider{$\im{q_\phi}\subset\mathbb R_-$}{12/40}
\slideq{Produit scalaire}{13/40}
\slider{Forme bilinéaire symétrique, définie et positive\linebreak Noté $\la x,y\ra$ ou $\l x\middle|y\r$}{13/40}
\slideq{Double orthogonal}{14/40}
\slider{$X\subset\l X^\perp\r^\perp$\linebreak En dimension finie, $X=\l X^\perp\r^\perp$}{14/40}
\slideq{Expression de $\left\|x\right\|^2$ dans la base orthonormée $\mathcal{B}=\l b_1,\cdots,b_n\r$}{15/40}
\slider{$\left\|x\right\|=\sum{i=1}{n}{\la x,b_i\ra^2}$}{15/40}
\slideq{$\left\|x\right\|$}{16/40}
\slider{$\sqrt{\la x,x\ra}$}{16/40}
\slideq{Forme quadratique}{17/40}
\slider{$q\!:\!E\to\mathbb R$ tel qu'il existe $\phi\in\mathcal B\l E\r$ tel que $q\l x\r=\phi\l x,x\r$\linebreak$q_\phi$ est la forme quadratique associée à $\phi$}{17/40}
\slideq{Formule de changement de base pour les formes bilinéaires}{18/40}
\slider{$\almat\phi{\mathcal D}{}=\l P_{\mathcal C}^{\mathcal D}\r^\top\almat\phi{\mathcal C}{}P_{\mathcal C}^{\mathcal D}$}{18/40}
\slideq{Groupe spécial orthogonal}{19/40}
\slider{$\so[n]{\mathbb R}=\left\{P\in\orth[n]{\mathbb R}\;\middle|\;\det P=1\right\}$}{19/40}
\slideq{Vecteurs orthogonaux}{20/40}
\slider{$x\perp y\Leftrightarrow\la x,y\ra=0$}{20/40}
\slideq{Structure de l'ensemble des formes bilinéaires}{21/40}
\slider{$\mathcal B\l E\r$ est un $\mathbb R$-espace vectoriel}{21/40}
\slideq{Norme euclidienne}{22/40}
\slider{Une norme $N$ est euclidienne si et seulement s'il existe un produit scalaire dont $N$ est la norme associée}{22/40}
\slideq{Théorème de Pythagore}{23/40}
\slider{$x\perp y\Leftrightarrow\left\|x+y\right\|^2=\left\|x\right\|^2+\left\|y\right\|^2$}{23/40}
\slideq{Procédé d'orthonormalisation de Gram-Schmidt}{24/40}
\slider{$f_1=\frac{e_1}{\left\|e_1\right\|}$\linebreak$\forall k\in\llb2,n\rrb,\;f_k=\frac{u_k}{\left\|u_k\right\|}$\linebreak$u_k=e_k-\sum{i=1}{k-1}{\la e_k,f_i\ra f_i}$}{24/40}
\slideq{$\phi$ est symétrique}{25/40}
\slider{$\forall\l x,y\r\in E^2,\;\phi\l x,y\r=\phi\l y,x\r$}{25/40}
\slideq{$\almat\phi{\mathcal B}{}$}{26/40}
\slider{$\l\phi\l e_i,e_j\r\r_{\l i,j\r\in\llb1,n\rrb^2}$}{26/40}
\slideq{Si $\mathcal{B}=\l b_1,\cdots,b_n\r$ est une base orthonormée et $u\in\al E{}$\linebreak$\almat u{\mathcal B}{}$}{27/40}
\slider{$\l\la b_i,u\l b_j\r\ra\r_{\left\{i,j\right\}\in\llb1,n\rrb^2}$\linebreak${}=\tmatrix({\la b_1,u\l b_1\r\ra\&\mdots\&\la b_1,u\l b_n\r\ra\\\vdots\&\plusdots\&\vdots\\\la b_n,u\l b_1\r\ra\&\mdots\&\la b_n,u\l b_n\r\ra\\})$}{27/40}
\slideq{Famille orthogonale\linebreak Famille orthonormale}{28/40}
\slider{$\l x_i\r_{i\in I}$ est orthogonale : $\forall\l i,j\r\in I^2,\;i\neq j\Rightarrow x_i\perp x_j$\linebreak$\l x_i\r_{i\in I}$ est orthonormée :\linebreak$\l x_i\r_{i\in I}$ est orthogonale et $\forall i\in I,\;\left\|x_i\right\|$}{28/40}
\slideq{Sous-espaces orthogonaux}{29/40}
\slider{$F\perp G\Leftrightarrow\forall\l x,y\r\in F\times G,\;x\perp y$}{29/40}
\slideq{Structure de l'orthogonal}{30/40}
\slider{$X^\perp$ est un sous-espace vectoriel de $E$}{30/40}
\slideq{Formule de polarisation}{31/40}
\slider{$\phi\l x,y\r=\frac12\l q\l x+y\r-q\l x\r-q\l y\r\r$}{31/40}
\slideq{$\dim{\mathcal B\l E\r}$}{32/40}
\slider{$n^2$}{32/40}
\slideq{Coordonnées de $x$ dans la base orthonormée $\mathcal{B}=\l b_1,\cdots,b_n\r$}{33/40}
\slider{$\lc x\rc_{\mathcal B}\tmatrix({\la x,b_1\ra\\\vdots\\\la x,b_n\ra\\})$}{33/40}
\slideq{$\phi$ est positive}{34/40}
\slider{$\im{q_\phi}\subset\mathbb R_+$}{34/40}
\slideq{Espace préhilbertien réel}{35/40}
\slider{$\l E,\la\cdot,\cdot\ra\r$ où $E$ est un $\mathbb R$-espace vectoriel muni d'un produit scalaire $\la\cdot,\cdot\ra$}{35/40}
\slideq{Théorème de Pythagore généralisé}{36/40}
\slider{$\left\|\sum{i=1}{n}{x_i}\right\|^2=\sum{i=1}{n}{\left\|x_i\right\|^2}$}{36/40}
\slideq{Projeté orthogonal sur $\vect x$}{37/40}
\slider{$z=\la y,x\ra\frac x{\left\|x\right\|^2}$}{37/40}
\slideq{Orhtogonal d'une union\linebreak Orthogonal d'une somme\linebreak Orthogonal d'une intersection}{38/40}
\slider{$\l F\cup G\r^\perp=F^\perp\cap G^\perp$\linebreak$\l F+G\r^\perp=F^\perp\cap G^\perp$\linebreak$\l F\cap G\r^\perp=F^\perp+G^\perp$ (en dimension finie)\linebreak$\l F\cap G\r^\perp\supset F^\perp+G^\perp$ (sinon)}{38/40}
\slideq{$X^\perp$}{39/40}
\slider{$\left\{x\in E\;\middle|\;x\perp X\right\}$}{39/40}
\slideq{$\phi$ est définie}{40/40}
\slider{$\forall x\in E,\;\phi\l x,x\r=0\Leftrightarrow x=0$}{40/40}
\end{document}
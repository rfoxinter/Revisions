\documentclass[14pt,usepdftitle=false,aspectratio=169]{beamer}
\usepackage{preambule}
\setbeamercolor{structure}{fg=black}
%packages
\hypersetup{pdftitle=Algèbre 2 -- Algèbre linéaire}
\title{Algèbre 2\\\emph{Algèbre linéaire}}
\author{}
\date{}
\begin{document}
\begin{frame}
    \titlepage
\end{frame}
\slideq{Un ensemble $E$ est un espace vectoriel sur $\mathbb K$\linebreak$E$ est un $\mathbb K$-ev}{1/1}
\slider{$\l E,+\r$ est un groupe abélien\linebreak$E$ est muni d'une loi de composition externe $\cdot$ avec $\forall\l\lambda,\mu,x,y\r\in\mathbb K^2\times E^2$\linebreak$\l \lambda\mu\r x=\lambda\l\mu x\r$ (associativité externe ou pseudo-associativité)\linebreak$1_{\mathbb K} x=x$ (compatibilité du neutre de $\l \mathbb K,\times\r$)\linebreak$\lambda\l x+y\r=\lambda x+\lambda y$ (distributivité de $\cdot$ sur $+_{\scriptscriptstyle E}$)\linebreak$\l \lambda+\mu\r x=\lambda x+\mu x$ (distributivité de $\cdot$ sur $+_{\scriptscriptstyle\mathbb K}$)}{1/1}
\end{document}
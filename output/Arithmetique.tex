\documentclass[14pt,usepdftitle=false,aspectratio=169]{beamer}
\usepackage{preambule}
\setbeamercolor{structure}{fg=black}
\let\oldmax\max\renewcommand\max[2][]{\oldmax_{#1}\l#2\r}\let\oldmin\min\renewcommand\min[2][]{\oldmin_{#1}\l#2\r}\let\oldsup\sup\renewcommand\sup[2][]{\oldsup_{#1}\l#2\r}\let\oldinf\inf\renewcommand\inf[2][]{\oldinf_{#1}\l#2\r}\newcommand{\cgr}[3]{#1\equiv#2\ \left[#3\right]}\usepackage{bigoperators}\usepackage{matrices}\renewcommand\phi\varphi
\hypersetup{pdftitle=Algèbre 1 -- Arithmétique}
\title{Algèbre 1\\\emph{Arithmétique}}
\author{}
\date{}
\begin{document}
\begin{frame}
    \titlepage
\end{frame}
\slideq{Théorème de Fermat}{1/12}
\slider{$p\in\mathbb P,\;a\in\mathbb N,\;\cgr{a^p}{a}{p}$\linebreak Si $p$ ne divise pas $a$, $\cgr{a^{p-1}}{1}{p}$}{1/12}
\slideq{Lemme d'Euclide}{2/12}
\slider{Si $a\mid bc$ et $a\in\mathbb P$, alors $a\mid b\vee a\mid c$\linebreak Si $a\wedge b=1$ et $a\wedge c=1$, alors $a\wedge bc=1$}{2/12}
\slideq{Formule de Legendre}{3/12}
\slider{$v_p\l n!\r=\sum{k=1}{+\infty}{\left\lfloor\frac n{p^k}\right\rfloor}$}{3/12}
\slideq{$a\wedge b$}{4/12}
\slider{$\max{\left\{n\in\mathbb N\;\vert\;n\mid a\wedge n\mid b\right\}}$\linebreak$\max[\l\mathbb N^*,\vert\r]{\left\{n\in\mathbb N\;\vert\;n\mid a\wedge n\mid b\right\}}$\linebreak$\inf[\l\mathbb N^*,\vert\r]{a,b}$\linebreak$a\mathbb Z+b\mathbb Z$\linebreak$\l a\r+\l b\r$ pour un anneau principal}{4/12}
\slideq{Théorème des restes chinois\linebreak\linebreak\tmatrix\{{\cgr{x}{b_1}{a_1} \\ \vdots \\ \cgr{x}{b_n}{a_n} \\}.\linebreak\line$\forall i\in\llb1,n\rrb,\;\forall j\in\llb1,n\rrb\setminus\left\{i\right\},\;a_i\wedge a_j=1$}{5/12}
\slider{$\widehat{a_i}=\prod{j\in\llb1,n\rrb\setminus\left\{i\right\}}{}{a_j}$\linebreak\linebreak$a_iu_i+\widehat{a_i}v_i=1$\linebreak\linebreak$\cgr{x}{\sum{i=1}{n}{b_iv_i\widehat{a_i}}}{\prod{i=1}{n}{a_i}}$}{5/12}
\slideq{Lemme de Gauss}{6/12}
\slider{Si $a\mid bc$ et $a\wedge b=1$, alors $a\mid c$}{6/12}
\slideq{Théorème d'Euler}{7/12}
\slider{$n\in\mathbb N,\;x\in\mathbb N^*,\;\cgr{x^{\phi\l n\r}}{1}{n}$}{7/12}
\slideq{$a\vee b$}{8/12}
\slider{$\min{\left\{n\in\mathbb N\;\vert\;a\mid n\wedge b\mid n\right\}}$\linebreak$\max[\l\mathbb N^*,\vert\r]{\left\{n\in\mathbb N\;\vert\;a\mid n\wedge b\mid n\right\}}$\linebreak$\sup[\l\mathbb N^*,\vert\r]{a,b}$\linebreak$a\mathbb Z\cap b\mathbb Z$\linebreak$\l a\r\cap\l b\r$ pour un anneau principal}{8/12}
\slideq{Relation entre $\wedge$ et $\vee$}{9/12}
\slider{$\l a\wedge b\r\l a\vee b\r=ab$}{9/12}
\slideq{$\phi(n)$}{10/12}
\slider{$\left|\l\mathbb Z/n\mathbb Z\r^\times\right|$}{10/12}
\slideq{Divisibilité avec le produit}{11/12}
\slider{Si $a\wedge b=1$, $a\mid c\wedge b\mid c$, alors $ab\mid c$}{11/12}
\slideq{Anneau euclidien}{12/12}
\slider{Si $\mathbb A$ est un anneau intègre, avec un stathme ($v\!:\!\mathbb A\setminus\left\{0\right\}\to\mathbb N$)\linebreak$A$ est euclidien si $\forall a\in\mathbb A,\;\forall b\in\mathbb A\setminus\left\{0\right\}\;\exists\l q,r\r\in\mathbb A^2,\;a=bq+r$\linebreak$r=0\vee v\l r\r<v\l b\r$}{12/12}
\end{document}
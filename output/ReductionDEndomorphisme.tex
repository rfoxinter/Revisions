\documentclass[14pt,usepdftitle=false,aspectratio=169]{beamer}
\usepackage{preambule}
\setbeamercolor{structure}{fg=black}
\usepackage{structures}\usepackage{bigoperators}\usepackage{polynomes}\usepackage{al}
\hypersetup{pdftitle=Algèbre -- Réduction d’endomorphisme}
\title{Algèbre\\\emph{Réduction d’endomorphisme}}
\author{}
\date{}
\begin{document}
\begin{frame}
    \titlepage
\end{frame}
\slideq{Théorème de Cayley-Hamilton}{1/13}
\slider{$\pi_A\mid\chi_A$}{1/13}
\slideq{Lien entre spectre et racines}{2/13}
\slider{$P\l u\r=0\Rightarrow\sp{u}\subset\rac{P\l u\r}$}{2/13}
\slideq{Expression de $\tr A$ avec $\sp A$}{3/13}
\slider{$\tr A=\sum{\lambda\in\sp A}{}{\mu_\lambda\l A\r\lambda}$}{3/13}
\slideq{$u$ est diagonalisable\linebreak Critère avec les espaces propres}{4/13}
\slider{$E=\bigop{\lambda\in\sp u}{}{E_\lambda\l u\r}$\linebreak$\dim E=\sum{\lambda\in\sp u}{}{\dim{E_\lambda\l u\r}}$}{4/13}
\slideq{Sous-espace caractéristique}{5/13}
\slider{$F_\lambda\l u\r=\ker{\l u-\lambda\id\r^{\mu_\lambda\l u\r}}$}{5/13}
\slideq{Lemme de décomposision des noyaux}{6/13}
\slider{Si $\l P_1,\cdots,P_n\r\in\pol KX^n$ et $\forall\l i,j\r\in\llb1,n\rrb^2$, $i\neq j\Rightarrow P_i\wedge P_j=1$\linebreak$\ker{\l\prod{k=1}{n}{P_k}\r\l u\r}=\bigop{k=1}{n}{\ker{P_k\l u\r}}$}{6/13}
\slideq{$u$ est trigonalisable}{7/13}
\slider{$\chi_u$ est scindé\linebreak$P\l u\r=0$ avec $P$ scindé\linebreak$\pi_u$ est scindé}{7/13}
\slideq{$\chi_A\l X\r$}{8/13}
\slider{$\det{XI_n-A}$}{8/13}
\slideq{Expression de $\det A$ avec $\sp A$}{9/13}
\slider{$\det A=\prod{\lambda\in\sp A}{}{\lambda^{\mu_\lambda\l A\r}}$}{9/13}
\slideq{Coefficients de $\chi_A$ avec $\tr{A}$ et $\det{A}$}{10/13}
\slider{$\chi_A\l X\r=X^n-\tr AX^{n-1}+\cdots+\l-1\r^n\det A$}{10/13}
\slideq{$u$ est diagonalisable\linebreak Critère avec $\chi_u$ et $\sp u$}{11/13}
\slider{$\chi_u$ est sindé et $\forall\lambda\in\sp u$, $\dim{E_\lambda\l u\r}=\mu_\lambda\l u\r$}{11/13}
\slideq{$\almat{u}{\mathcal B}{}$ dans $\mathcal B$ adaptée à $\bigop{\lambda\in\sp u}{}{F_\lambda\l u\r}$}{12/13}
\slider{\setlength{\matmin}{4.75ex}$\tmatrix({\lambda_1I_n+T_1\&0\&\mdots\&0\\0\&\ddots\&\ddots\&\vdots\\\vdots\&\ddots\&\ddots\&0\\0\&\mdots\&0\&\lambda_pI_n+T_p\\})$\linebreak$\l T_1,\cdots,T_p\r\in\mathcal T^{++}_n\l\mathbb K\r$}{12/13}
\slideq{$u$ est diagonalisable\linebreak Critère avec un annulateur}{13/13}
\slider{$P\l u\r=0$ avec $P$ scindé à racines simples\linebreak$\pi_u$ est scindé à racines simples}{13/13}
\end{document}
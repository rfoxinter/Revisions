\documentclass[14pt,usepdftitle=false,aspectratio=169]{beamer}
\usepackage{preambule}
\setbeamercolor{structure}{fg=black}
\usepackage{dsft}\usepackage{bigoperators}\usepackage{structures,arithmetique}
\hypersetup{pdftitle=Algèbre 1 -- Structures algébriques}
\title{Algèbre 1\\\emph{Structures algébriques}}
\author{}
\date{}
\begin{document}
\begin{frame}
    \titlepage
\end{frame}
\slideq{Si $G$ est un gruope\linebreak Structure de $\l\aut G,\circ\r$}{1/90}
\slider{$\l\aut G,\circ\r$ est un groupe}{1/90}
\slideq{Si $K$ est un corps de caractéristique finie $p$\linebreak Propriété pour les éléments de $K$}{2/90}
\slider{$\forall x\in K,\;px=0_K$}{2/90}
\slideq{Si $G$ et $H$ sont deux groupes et $f\in\hom{G,H}$\linebreak$f\l e_G\r$}{3/90}
\slider{$f\l e_H\r$}{3/90}
\slideq{Magma}{4/90}
\slider{Muni d'une loi de composition interne}{4/90}
\slideq{Passage au quotient de la loi dans le cas d'un sous-groupe distingué\linebreak Si $G$ est un groupe et $H$ un sous-groupe distingué de $G$}{5/90}
\slider{${\equiv_g}={\equiv_d}$ et on note la relation $\equiv$\linebreak La loi induite corrrespond au produit des classes élément par élément\linebreak$\l ab\r H=\l aH\r\cdot\l bH\r$\linebreak${}=\left\{x\cdot y,\;x\in aH,\;y\in bH\right\}$\linebreak La loi induite sur l'ensemble quotient munit celui-ci d'une structure de groupe}{5/90}
\slideq{Groupe cyclique}{6/90}
\slider{Groupe monogène fini}{6/90}
\slideq{Image directe et réciproque de sous-corps par un homomorphisme}{7/90}
\slider{Si $K$ et $L$ sont deux corps, et $f\!:\!K\to L$ un morphisme de corps, $K'$ et $L'$ deux sous-corps respectivement de $K$ et $L$\linebreak $f\l K'\r$ est un sous-corps de $L$\linebreak $f^{-1}\l L'\r$ est un sous-corps de $K$}{7/90}
\slideq{Si $K$ est un corps de caractéristique nulle\linebreak Propriété pour les éléments de $K$}{8/90}
\slider{$\forall \l n,x\r\in\mathbb{Z}\times K$\linebreak$n\times x=0_K\Leftrightarrow\l x=0_K\vee n=0\r$}{8/90}
\slideq{Soit $x\in E$\linebreak$x$ est un élement absorbant pour $\star$}{9/90}
\slider{$\forall y\in E,\;x\star y=x=y\star x$}{9/90}
\slideq{Groupe des inversibles d'un anneau}{10/90}
\slider{$A^\times$\linebreak$A^\times$ est un groupe multiplicatif}{10/90}
\slideq{Élément régulier ou simplifiable}{11/90}
\slider{$x$ est régulier à gauche si et seulement si$\forall\l y,z\r\in E^2,\;x\star y=x\star z\Rightarrow y=z$\linebreak$x$ est régulier à droite si et seulement si$\forall\l y,z\r\in E^2,\;y\star x=z\star x\Rightarrow y=z$\linebreak$x$ est régulier si et seulement s'il est régulier à gauche et à droite\linebreak Si $x$ admet un symétrique, alors il est régulier}{11/90}
\slideq{Si $\l G,\star\r$ est un groupe et $H\subset G$\linebreak Caractérisation(s) des sous-groupes}{12/90}
\slider{$H\neq\varnothing$\quad$\forall\l x,y\r\in H,\;x\star y\in H$\quad$\forall x\in H,\;x^s\in H$\linebreak$H\neq\varnothing$\quad$\forall\l x,y\r\in H^2,\;x\star y^s\in H$\linebreak$e_G\in H$\quad$\forall\l x,y\r\in H^2,\;x\star y^s\in H$}{12/90}
\slideq{Si $\l G,\star\r$ est un groupe\linebreak Un sous-ensemble $H$ de $G$ est un sous-groupe de $G$}{13/90}
\slider{$H$ est stable pour la loi de $G$ et la loi induite définit sur $H$ une structure de groupe}{13/90}
\slideq{Si $A$ est un anneau commutatif et $I$ un idéal de $A$\linebreak Anneau quotient}{14/90}
\slider{$A/I$ peut être muni d'une multiplication avec pour tout $\l a,b\r\in A$, $\overline{ab}=\overline{a}\overline{b}$\linebreak$A/I$ est muni d'une structure d'anneau}{14/90}
\slideq{Élément absorbant dans un anneau $\l A,+,\times\r$}{15/90}
\slider{$0$}{15/90}
\slideq{Factorisation de $a^n-b^n$ dans un anneau $A$}{16/90}
\slider{$\l a,b\r\in A^2$ tel que $ab=ba$\linebreak$\l a-b\r\sum{k=0}{n-1}{a^{n-k-1}b^k}$}{16/90}
\slideq{Groupe}{17/90}
\slider{Muni d'une loi de composition interne, de l'associativité, d'un élément neutre et de symétriques\linebreak Un groupe est un monoïde}{17/90}
\slideq{Si $\l K,+,\times\r$ est un groupe et $L\subset K$\linebreak Caractérisation des sous-corps}{18/90}
\slider{$1_K\in L$\quad$\forall\l x,y\r\in L,\;x-y\in L$\quad$\forall\l x,y\r\in L,\;y\neq0\Rightarrow xy^{-1}\in L$}{18/90}
\slideq{Monoïde}{19/90}
\slider{Muni d'une loi de composition interne, de l'associativité et d'un élément neutre\linebreak Un monoïde est un magma}{19/90}
\slideq{Propriété sur $1$ et $0$ si l'anneau $A$ a plus d'un élément}{20/90}
\slider{$1\neq0$}{20/90}
\slideq{Propriété des homomorphismes de corps}{21/90}
\slider{Un homomorphisme de corps est injectif}{21/90}
\slideq{Cardinal des classes de congruence}{22/90}
\slider{$\left|Ha,\;a\in G\right|=\left|Ha,\;a\in G\right|=\left|H\right|$}{22/90}
\slideq{Propriété de $\mathbb{F}_p=\mathbb{Z}/p\mathbb{Z}$}{23/90}
\slider{$\mathbb{F}_p$ est un corps si et seulement si $p$ est premier}{23/90}
\slideq{Si $\star$ est une loi associative sur $E$ et $\l x_1,\cdots,x_n\r\in E^n$}{24/90}
\slider{$x_1\star\cdots\star x_n$ ne dépend pas du parenthésage admissible}{24/90}
\slideq{Idéal principal}{25/90}
\slider{Idéal engendré par un unique élément $a$ de la forme $I=aA=\left\{ay,\;y\in A\right\}$\linebreak$I$ est souvent noté $\l a\r$}{25/90}
\slideq{Ensemble formé par les classes à gauche et à droite}{26/90}
\slider{$\left\{Ha,\;a\in G\right\}$ est une partition de $G$\linebreak$\left\{aH,\;a\in G\right\}$ est une partition de $G$}{26/90}
\slideq{Commutativité}{27/90}
\slider{$\star$ est commutative si et seulement si$\forall\l x,y\r\in E^2,\;x\star y=y\star x$}{27/90}
\slideq{Si $\l A,+,\times\r$ est un anneau\linebreak Un sous-ensemble $B$ de $A$ est un sous-anneau de $A$}{28/90}
\slider{$B$ est stable pour les lois $+$ et $\times$\linebreak$1_A\in B$\linebreak Les lois induites sur $B$ définissent sur $B$ une structure d'anneau}{28/90}
\slideq{Distributivité généralisée\linebreak$\prod{i=1}{n}{\sum{j\in J_i}{}{x_{i,j}}}$}{29/90}
\slider{$\sum{\l j_1,\cdots,j_n\r\in J_1\times\cdots\times J_n}{}{\prod{i=1}{n}{x_{i,j_i}}}$}{29/90}
\slideq{Ordre d'un groupe\linebreak Si $G$ est un groupe}{30/90}
\slider{$\ord{G}=\left|G\right|$}{30/90}
\slideq{Résolution de $x^n=e$}{31/90}
\slider{$\left\{n\in\mathbb{N}^*\;\middle|\;x^n=e\right\}$ est de la forme $a\mathbb{Z}$\linebreak$x$ est d'ordre fini si et seulement si $a\neq0$ (on a donc $\ord{x}=a$)}{31/90}
\slideq{Soit $E$ et $F$ deux ensembles munis d'une structure de X, munis respectivement des lois de composition internes $\l\underset{\scriptscriptstyle1}{\star},\cdots,\underset{\scriptscriptstyle n}{\star}\r$ et $\l\underset{\scriptscriptstyle1}{\diamond},\cdots,\underset{\scriptscriptstyle n}{\diamond}\r$, et externes $\l\underset{\scriptscriptstyle1}{\square},\cdots,\underset{\scriptscriptstyle m}{\square}\r$ et $\l\underset{\scriptscriptstyle1}{\triangle},\cdots,\underset{\scriptscriptstyle m}{\triangle}\r$ sur $K_1,\cdots,K_m$\linebreak$f\!:\!E\to F$ est un homomorphisme}{32/90}
\slider{$f$ respecte les lois interne : soit $k\in\llb1,n\rrb$\linebreak$\forall \l x,y\r\in E^2,\;f\l x\underset{\scriptscriptstyle k}{\star}y\r=f\l x\r\underset{\scriptscriptstyle k}{\diamond}f\l y\r$\linebreak$f$ respecte les lois externes : soit $k\in\llb1,m\rrb$\linebreak$\forall \l\lambda,x\r\in K_k\times E,\;f\l \lambda\underset{\scriptscriptstyle k}{\square}y\r=\lambda\underset{\scriptscriptstyle k}{\triangle}f\l x\r$\linebreak$f$ est compatible avec le neutre (si le neutre $e_i$ pour la loi $\underset{\scriptscriptstyle i}{\star}$ est imposé dans les axiomes, donc le neutre $e_i'$ existe pour la loi $\underset{\scriptscriptstyle i}{\diamond}$) : $f\l e_i\r=e_i'$}{32/90}
\slideq{Si $K$ est un corps, d'élément neutre $1_K\neq0_K$, $H=\left\{n\times1_K,\;n\in\mathbb{Z}\right\}$ le sous-groupe monogène de $\l K,+\r$ engendré par $1_K$\linebreak Caractéristique d'un corps}{33/90}
\slider{Si $H$ est infini, $K$ est de caractéristique nulle\linebreak Si $H$ est fini de cardinal $p$, $K$ est de caractéristique $p$}{33/90}
\slideq{Si $\ker{f}=\left\{e_G\right\}$}{34/90}
\slider{$f$ est injectif (la réciproque est vraie)}{34/90}
\slideq{$x$ et $y$ sont dans la même classe à gauche modulo $H$}{35/90}
\slider{$x\equiv_gy\,\left[H\right]\Leftrightarrow x^{-1}y\in H$}{35/90}
\slideq{Soient $E$ muni d'une structure de X et $F\subset E$\linebreak$F$ est un sous-X de $E$}{36/90}
\slider{$F$ est stable par les lois de $E$\linebreak$F$ contient les neutres imposés par $E$\linebreak Les lois induites sur $F$ par les lois de $E$ vérifient les axiomes de la structure de X}{36/90}
\slideq{Si $f\in\hom{G,K}$ et $H$ est un sous-groupe distingué et $H\subset\ker{f}$}{37/90}
\slider{$f=\tilde{f}\circ\pi$\linebreak La réciproque est vraie}{37/90}
\slideq{Commutativité généralisée}{38/90}
\slider{Si $\star$ est une loi commutative et associative sur $E$, $\l x_1,\cdots,x_n\r\in E^n$ et $\sigma\in\mathfrak{S}_n$\linebreak$x_1\star\cdots\star x_n=x_{\sigma\l1\r}\star\cdots\star x_{\sigma\l n\r}$}{38/90}
\slideq{Soient $\l K,\underset{\scriptscriptstyle K}{+},\underset{\scriptscriptstyle K}{\times}\r$ et $\l L,\underset{\scriptscriptstyle L}{+},\underset{\scriptscriptstyle L}{\times}\r$ deux corps\linebreak$f\!:\!K\to L$ est un homomorphisme de corps}{39/90}
\slider{$f$ est un homomorphisme des anneaux de $K$ et $L$}{39/90}
\slideq{Les classes à droite modulo $H$}{40/90}
\slider{$\left\{Ha,\;a\in G\right\}$}{40/90}
\slideq{Intersection de sous-groupes\linebreak Si $G$ est un groupe, et $\l H_i\r_{i\in I}$ une famille de sous-groupes de $G$}{41/90}
\slider{$\bigcap{i\in I}{}{H_i}$ est un sous-groupe de $G$}{41/90}
\slideq{Soient $\l A,\underset{\scriptscriptstyle A}{+},\underset{\scriptscriptstyle A}{\times}\r$ et $\l B,\underset{\scriptscriptstyle B}{+},\underset{\scriptscriptstyle B}{\times}\r$ deux anneaux\linebreak$f\!:\!A\to B$ est un homomorphisme d'anneaux}{42/90}
\slider{$\forall\l x,y\r\in A^2,\;f\l x\underset{\scriptscriptstyle A}{+}y\r=f\l x\r\underset{\scriptscriptstyle B}{+}f\l y\r$\linebreak$\forall\l x,y\r\in A^2,\;f\l x\underset{\scriptscriptstyle A}{\times}y\r=f\l x\r\underset{\scriptscriptstyle B}{\times}f\l y\r$\linebreak$f\l 1_A\r=1_B$}{42/90}
\slideq{Propriété de la caractéristique d'un corps}{43/90}
\slider{Si $K$ est un corps de caractéristique $p$ non nulle, $p$ est premier}{43/90}
\slideq{Anneau commutatif}{44/90}
\slider{Anneau dont la loi $\times$ est commutative}{44/90}
\slideq{Automorphisme de X}{45/90}
\slider{Endomorphisme et isomorphisme de X}{45/90}
\slideq{Si $G$ et $H$ sont deux groupes et $f\in\hom{G,H}$\linebreak$f\l x^{-1}\r$}{46/90}
\slider{$f\l x\r^{-1}$}{46/90}
\slideq{Anneau intègre}{47/90}
\slider{Anneau commutatif non réduit à $\left\{0\right\}$ et sans diviseurs de $0$}{47/90}
\slideq{Anneau}{48/90}
\slider{Muni de deux lois de composition internes (généralement notées $+$ et $\times$)\linebreak$\l A,+\r$ est un groupe abélien\linebreak$\l A,\times\r$ est un monoïde\linebreak$\times$ est distributive sur $+$}{48/90}
\slideq{Image directe et réciproque de sous-groupes par un homomorphisme}{49/90}
\slider{Si $G$ et $H$ sont deux groupes, et $f\in\hom{G,H}$ un morphisme de groupes, $G'$ et $H'$ deux sous-groupes respectivement de $G$ et $H$\linebreak $f\l G'\r$ est un sous-groupe de $H$\linebreak $f^{-1}\l H'\r$ est un sous-groupe de $G$}{49/90}
\slideq{Théorème de Lagrange pour l'ordre des groupes}{50/90}
\slider{Si $G$ est un groupe fini et $H$ un sous-groupe de $G$\linebreak$\left|H\right|\div\left|G\right|$}{50/90}
\slideq{Anneau principal}{51/90}
\slider{Un anneau intègre dont tous les idéaux sont principaux}{51/90}
\slideq{Associativité}{52/90}
\slider{$\star$ est associative si et seulement si$\forall\l x,y,z\r\in E^3,\;\l x\star y\r\star z=x\star\l y\star z\r$}{52/90}
\slideq{$x$ et $y$ sont dans la même classe à droite modulo $H$}{53/90}
\slider{$x\equiv_dy\,\left[H\right]\Leftrightarrow xy^{-1}\in H$}{53/90}
\slideq{Sous-groupe propre de $G$}{54/90}
\slider{Sous-groupe de $G$ distinct de $G$ et $\left\{e_G\right\}$}{54/90}
\slideq{Les classes à gauche modulo $H$}{55/90}
\slider{$\left\{aH,\;a\in G\right\}$}{55/90}
\slideq{Intersection de sous-anneaux\linebreak Si $A$ est un groupe, et $\l B_i\r_{i\in I}$ une famille de sous-anneaux de $A$}{56/90}
\slider{$\bigcap{i\in I}{}{B_i}$ est un sous-anneau de $A$}{56/90}
\slideq{Symétrique de $x\star y$}{57/90}
\slider{$y^s\star x^s$}{57/90}
\slideq{Corps}{58/90}
\slider{Muni de deux lois de composition internes (généralement notées $+$ et $\times$)\linebreak$\l K,+,\times\r$ est un anneau commutatif\linebreak$\l K^*,\times\r$ est un groupe}{58/90}
\slideq{Description par le bas du sous-groupe engendré par une partie}{59/90}
\slider{$\la X\ra=\left\{x_1\cdots x_n,\;\l x_1,\cdots,x_n\r\in X^n\right\}$\linebreak${}\cup\left\{x^{-1},\;x\in X\right\}$\linebreak $e$ correspond au produit vide}{59/90}
\slideq{Ordre d'un élément d'un groupe}{60/90}
\slider{$\ord{x}=\min\left(\left\{n\in\mathbb{N}^*\;\middle|\;x^n=e\right\}\right)$}{60/90}
\slideq{Passage au quotient de la loi dans le cas abélien\linebreak Si $G$ est un groupe abélien et $H$ un sous-groupe de $G$}{61/90}
\slider{${\equiv_g}={\equiv_d}$ et on note la relation $\equiv$\linebreak La loi induite corrrespond au produit des classes élément par élément\linebreak$\l ab\r H=\l aH\r\cdot\l bH\r$\linebreak${}=\left\{x\cdot y,\;x\in aH,\;y\in bH\right\}$\linebreak La loi induite sur l'ensemble quotient munit celui-ci d'une structure de groupe abélien}{61/90}
\slideq{Fibres de $f$\linebreak Soit $x\in f^{-1}\l\left\{y\right\}\r$}{62/90}
\slider{$f^{-1}\l\left\{y\right\}\r=x\times\ker{f}$\linebreak${}=\left\{x\times z,\;z\in\ker{f}\right\}=\ker{f}\times x$}{62/90}
\slideq{Si $\l A,+,\times\r$ est un groupe et $B\subset A$\linebreak Caractérisation des sous-anneaux}{63/90}
\slider{$1_A\in B$\quad$\forall\l x,y\r\in B,\;x-y\in B$\quad$\forall\l x,y\r\in B,\;xy\in B$}{63/90}
\slideq{Si $H$ est un sous-groupe distingué de $G$}{64/90}
\slider{$\forall a\in G,\;aH=Ha$\linebreak${}\Leftrightarrow\forall a\in G,\;\forall h\in H,\;aha^{-1}\in H$}{64/90}
\slideq{Si $f\in\hom{G,K}$ et $H$ est un sous-groupe distingué}{65/90}
\slider{$f$ passe au quotient avec $\tilde{f}\!:\!G/H\to K$}{65/90}
\slideq{Endomorphisme de X}{66/90}
\slider{Homomorphisme de X de $E$ dans lui-même (muni des mêmes lois)}{66/90}
\slideq{Soit $e\in E$\linebreak$e$ est un élément neutre pour la loi $\star$}{67/90}
\slider{$\forall x\in E,\;e\star x=x=x\star e$}{67/90}
\slideq{Propriété des groupes monogènes}{68/90}
\slider{Un groupe monogène est abélien}{68/90}
\slideq{Description par le haut du sous-groupe engendré par une partie}{69/90}
\slider{Soient $\mathcal{G}$ l'ensemble des sous-groupes de $G$ et $\mathcal{H}=\left\{H\in\mathcal{G}\;\middle|\;X\subset H\right\}$\linebreak$\la X\ra=\bigcap{H\in\mathcal{H}}{}{H}$}{69/90}
\slideq{Sous-groupe engendrée par une partie $X$}{70/90}
\slider{$\la X\ra$\linebreak C'est le plus petit sous-groupe contenant $X$}{70/90}
\slideq{Diviseurs de zéro dans un anneau $A$}{71/90}
\slider{$a\in A$ est un diviseur de $0$ à gauche si et seulement s'il existe $b\in A$ tel que $ab=0$\linebreak$a\in A$ est un diviseur de $0$ à droite si et seulement s'il existe $b\in A$ tel que $ba=0$\linebreak$a\in A$ est un diciseur de$0$ si et seulement si $a$ est diviseur de $0$ à gauche et à droite}{71/90}
\slideq{Si $\l K,+,\times\r$ est un corps\linebreak Un sous-ensemble $L$ de $K$ est un sous-corps de $K$}{72/90}
\slider{$L$ est stable pour les lois $+$ et $\times$\linebreak$1_K\in L$\linebreak Les lois induites sur $L$ définissent sur $L$ une structure de corps}{72/90}
\slideq{Description des groupes monogènes\linebreak Si $G=\la x\ra$}{73/90}
\slider{Si $\ord{x}=+\infty$, $G$ est isomorphe à $\mathbb{Z}$\linebreak Si $\ord{x}=n\in\mathbb{N}^*$, $G$ est isomorphe à $\mathbb{Z}/n\mathbb{Z}$}{73/90}
\slideq{Distributivité}{74/90}
\slider{La loi $\star$ est distributive à gauche sur $\diamond$ si et seulement si$\forall\l x,y,z\r\in E^3,\;x\star\l y\diamond z\r=\l x\star y\r\diamond\l x\star z\r$\linebreak La loi $\star$ est distributive à droite sur $\diamond$ si et seulement si$\forall\l x,y,z\r\in E^3,\;\l y\diamond z\r\star x=\l y\star x\r\diamond\l z\star x\r$\linebreak La loi $\star$ est distributive sur $\diamond$ si et seulement si elle est distributive à gauche et à droite}{74/90}
\slideq{Propriétés d'un groupe $\l G,\star\r$}{75/90}
\slider{$G$ admet un uique élément neutre pour $\star$\linebreak$\forall x\in G,\;\exists!x^s\in G$}{75/90}
\slideq{Soient $e\in E$ un élément neutre pour la loi $\star$ et $x\in E$\linebreak$y$ est un symétrique de $x$ pour la loi $\star$}{76/90}
\slider{$x\star y=e=y\star x$}{76/90}
\slideq{Soient $\l G,\star\r$ et $\l H,\diamond\r$ deux groupes\linebreak$f\!:\!G\to H$ est un homomorphisme de groupes}{77/90}
\slider{$\forall\l x,y\r\in G^2,\;f\l x\star y\r=f\l x\r\diamond f\l y\r$\linebreak L'ensemble des homomorphisme de $G$ dans $H$ est noté $\hom{G,H}$\linebreak Si $\l G,\star\r=\l H,\diamond\r$, $f$ est un endomorphisme\linebreak L'ensemble des automorphismes de $G$ est noté $\aut{G}$}{77/90}
\slideq{Sous-groupe monogène}{78/90}
\slider{$\la x\ra=\left\{x^n,\;n\in\mathbb{N}\right\}$}{78/90}
\slideq{Si $G$ et $H$ sont deux groupes et $f\in\hom{g,h}$ un morphisme de groupes\linebreak$\ker{f}$}{79/90}
\slider{$f^{-1}\l e_H\r=\left\{y\in G\;\middle|\;f\l y\r=e_H\right\}$}{79/90}
\slideq{Élément réguulier d'un anneau}{80/90}
\slider{L'élément n'est pas diviseur de $0$\linebreak La réciproque est vraie\linebreak S'adapte à gauche et à droite}{80/90}
\slideq{Réciproque d'isomorphisme}{81/90}
\slider{Si $f\!:\!F\to F$ est un isomorphisme, alors $f^{-1}$ est un isomorphisme}{81/90}
\slideq{Premier théorème d'isomorphisme}{82/90}
\slider{Si $f\in\hom{G,H}$\linebreak$\ker{f}$ est un sous-groupe distingué de $G$, et $f$ passe au quotient, définissant un morphisme de groupes $\tilde{f}\!:\!G/\ker{f}\to H$\linebreak $\tilde{f}$ est injectif et sa corestriction à son image est un isomorphisme}{82/90}
\slideq{Image directe et réciproque de sous-anneaux par un homomorphisme}{83/90}
\slider{Si $A$ et $B$ sont deux anneaux, et $f\!:\!A\to B$ un morphisme d'anneaux, $A'$ et $B'$ deux sous-anneaux respectivement de $A$ et $B$\linebreak $f\l A'\r$ est un sous-anneau de $B$\linebreak $f^{-1}\l B'\r$ est un sous-anneau de $A$}{83/90}
\slideq{Associativité externe\linebreak$E$ est muni d'une loi decomposition externe $\diamond$ sur $\mathbb{K}$, muni d'une loi de composition interne $\star$}{84/90}
\slider{$\forall\l\lambda,\mu,x\r\in\mathbb{K}^2\times E,\;\l\lambda\star\mu\r\diamond x=\lambda\diamond\l\mu\diamond x\r$}{84/90}
\slideq{Soient $E$ muni d'une loi $\star$, $F\subset E$\linebreak$F$ est stable par $\star$}{85/90}
\slider{$\forall\l x,y\r\in F^2,\;x\star y\in F$\linebreak La loi de $E$ se restreint en une loi $\star_F$ appelée loi induite sur $F$ par $\star$}{85/90}
\slideq{Théorème de Lagrange pour l'ordre des éléments d'un groupe}{86/90}
\slider{Si $G$ est un groupe fini et $x\in G$\linebreak$\ord{x}\div\left|G\right|$}{86/90}
\slideq{Isomorphisme de X}{87/90}
\slider{Homomorphisme de X bijectif}{87/90}
\slideq{Groupe abélien}{88/90}
\slider{La loi $\star$ de $G$ est commutative}{88/90}
\slideq{Factorisation de $\l a+b\r^n$ dans un anneau $A$}{89/90}
\slider{$\l a,b\r\in A^2$ tel que $ab=ba$\linebreak$\sum{k=0}{n}{\binom{n}{k}a^kb^{n-k}}$}{89/90}
\slideq{Si $\l A,+,\times\r$ est un anneau commutatif\linebreak Un sous-ensemble $I$ de $A$ est un sous-anneau idéal de $A$}{90/90}
\slider{$I$ est un sous-groupe de $\l A,+\r$\linebreak$\forall i\in I,\;\forall a\in A,\;ia\in I$}{90/90}
\end{document}
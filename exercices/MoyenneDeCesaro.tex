\documentclass[a4paper,12pt]{article}
\usepackage[utf8]{inputenc}
\usepackage[french]{babel}
\usepackage[T1]{fontenc}
\usepackage[margin=2.5cm]{geometry}
\usepackage{amsmath, amsfonts, amssymb}
\usepackage{xcolor}
\usepackage[most]{tcolorbox}

\setlength{\parindent}{0cm}
\everymath{\displaystyle}
\mathcode`l="8000
\begingroup
\makeatletter
\lccode`\~=`\l
\DeclareMathSymbol{\lsb@l}{\mathalpha}{letters}{`l}
\lowercase{\gdef~{\ifnum\the\mathgroup=\m@ne \ell \else \lsb@l \fi}}%
\endgroup

\renewcommand\l{\left(}
\renewcommand\r{\right)}
\let\oldleft\left
\renewcommand{\left}{\mathopen{}\mathclose\bgroup\oldleft}
\let\oldright\right
\renewcommand{\right}{\aftergroup\egroup\oldright}
\let\oldsum\sum\renewcommand\sum[3]{\oldsum_{#1}^{#2}\l#3\r}
\let\le\leqslant
\let\ge\geqslant
\let\epsilon\varepsilon
\newcommand\va[1]{\left|#1\right|}

\pagenumbering{gobble}
\title{\vspace{-1.5cm}Moyenne de Cesàro}
\author{}
\date{}

\newtcbtheorem{thm}{Théorème}{enhanced,breakable,sharp corners,titlerule=0.5pt,titlerule style={black,double},colback=white,coltitle=black,colframe=white,fonttitle=\boldmath\bfseries,borderline={0.5pt}{0.5pt}{black,double},separator sign={~:}}{theorem}
\newtcbtheorem{prv}{Preuve}{enhanced,breakable,sharp corners,titlerule=0.5pt,titlerule style=black,colback=white,coltitle=black,colframe=white,fonttitle=\boldmath\bfseries,borderline={0.5pt}{0.5pt}{black},separator sign={~:}}{proof}

\begin{document}
\maketitle
\vspace{-2cm}
\begin{thm*}{Moyenne de Cesàro}
Soit $\l u_n\r_{n\in\mathbb N}$ une suite de nombres réels ou complexes.

Si $\l u_n\r$ converge vers $l$, alors ${\frac1{n+1}}\sum{k=0}{n}{u_k}$ converge également, et sa limite est $l$. 
\end{thm*}

\begin{prv*}{Moyenne de Cesàro}
Soit $\l u_n\r$ une suite tel que $u_n\to l$.

Montrons que $\frac1n\sum{k=0}{n}{u_k}\to l$.

Soit $\epsilon\in\mathbb{R}_+^*$.

Soit $n_0\in\mathbb N$ tel que pour tour $n\ge n_0$, $\va{u_n-l}\le\frac\epsilon2$.
\begin{flalign*}
    \va{\frac1{n+1}\sum{k=0}{n}{u_k}-l}&=\va{\frac1{n+1}\sum{k=0}{n}{u_k}-\frac1{n+1}\sum{k=0}{n}{l}}&\\
    &=\va{\frac1{n+1}\sum{k=0}{n}{u_k-l}}&\\
    &\le\frac{1}{n+1}\sum{k=0}{n}{\va{u_k-l}}&\\
    &=\frac{1}{n+1}\sum{k=0}{n_0-1}{\va{u_k-l}}+\frac{1}{n+1}\sum{k=n_0}{n}{\va{u_k-l}}&\\
    &\le\frac{1}{n+1}\sum{k=0}{n_0-1}{\va{u_k-l}}+\frac{1}{n+1}\sum{k=n_0}{n}{\frac\epsilon2}&\\
    &=\frac{1}{n+1}\sum{k=0}{n_0-1}{\va{u_k-l}}+\frac{n-n_0+1}{n+1}\times\frac\epsilon2&\\
    &\le\frac{1}{n+1}\sum{k=0}{n_0-1}{\va{u_k-l}}+\frac\epsilon2&
\end{flalign*}
De plus, il existe $n_1\ge n_0$ tel que $\frac{1}{n_1+1}\sum{k=0}{n_0-1}{u_k}\le\frac\epsilon2$.

Donc, pour tout $n\ge n_1$, $\va{\frac1{n+1}\sum{k=0}{n}{u_k}-l}\le\epsilon$.

Donc, $\va{\frac1{n+1}\sum{k=0}{n}{u_k}-l}\to0$.

Donc, $\frac1{n+1}\sum{k=0}{n}{u_k}\to l$.
\end{prv*}
\end{document}

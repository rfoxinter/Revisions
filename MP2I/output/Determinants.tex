\documentclass[14pt,usepdftitle=false,aspectratio=169]{beamer}
\usepackage{preambule}
\setbeamercolor{structure}{fg=black}
\let\phi\varphi\newcommand\eps[1]{\varepsilon\l#1\r}\usepackage{al}\usepackage{bigoperators}\usepackage{polynomes,structures}
\hypersetup{pdftitle=Algèbre 2 -- Déterminants}
\title{Algèbre 2\\\emph{Déterminants}}
\author{}
\date{}
\begin{document}
\begin{frame}
    \titlepage
\end{frame}
\slideq{$\olddet_{\mathcal B}$\linebreak Expression avec $\mathcal B'$}{1/25}
\slider{$\det[\mathcal B]{\mathcal B'}\olddet_{\mathcal B'}$}{1/25}
\slideq{Application multilinéaire}{2/25}
\slider{$\phi\l x_1,\cdots,x_{i-1},\lambda x_i+x_i',x_{i+1},\cdots,x_n\r$\linebreak${}=\lambda\phi\l x_1,\cdots,x_{i-1},x_i,x_{i+1},\cdots,x_n\r$\linebreak${}+\phi\l x_1,\cdots,x_{i-1},x_i',x_{i+1},\cdots,x_n\r$}{2/25}
\slideq{Forme $n$-linéaire}{3/25}
\slider{Application linéaire à valeurs dans $\mathbb K$}{3/25}
\slideq{Déterminant d'une famille de vecteurs $\l x_1,\cdots,x_n\r$ par rapport à $\mathcal B$}{4/25}
\slider{Si $\olddet_{\mathcal B}$ est l'unique forme $n$-linéaire alternée telle que $\det[\mathcal B]{\mathcal B}=1$\linebreak$\det[\mathcal B]{x_1,\cdots,x_n}$}{4/25}
\slideq{$\tmatrix[\mtxbox{}{1}{1}\mtxbox{}{2}{2}\mtxbox{}{4}{4}][inner sep = 0.5ex, minimum width = 30pt, minimum height = 30pt, row sep = 1ex, column sep = 1ex,]|{A_1\&0\&\mdots\&0\\0\&A_2\&\ddots\&\vdots\\\vdots\&\ddots\&\ddots\&0\\0\&\mdots\&0\&A_n\\}|$}{5/25}
\slider{$\prod{i=1}{n}{\det{A_i}}$}{5/25}
\slideq{Caractérisation du déterminant par l'image d'une base}{6/25}
\slider{$\det u=\det[\mathcal B]{u\l\mathcal B\r}$}{6/25}
\slideq{$\det{u\circ v}$}{7/25}
\slider{$\det u\det v$}{7/25}
\slideq{Déterminant de Vandermonde\linebreak$V\l x_1,\cdots,x_n\r$}{8/25}
\slider{$\tmatrix|{1\&1\&\mdots\&1\\x_1\&x_2\&\mdots\&x_n\\x_1^2\&x_2^2\&\mdots\&x_n^2\\\vdots\&\vdots\&\plusdots\&\vdots\\x_1^{n-1}\&x_2^{n-1}\&\mdots\&x_n^{n-1}\\}|=\prod{\l i,j\r\in\llb1,n\rrb^2}{}{x_j-x_i}$}{8/25}
\slideq{Lien forme antisymétrique -- forme alternée}{9/25}
\slider{Toute forme $n$-linéaire alternée est antisymétrique\linebreak Si $\car{\mathbb K}\neq2$, toute forme antisymétrique est alternée}{9/25}
\slideq{Déterminant d'un endomorphisme}{10/25}
\slider{$\phi_u=\det u\phi$}{10/25}
\slideq{$\sl E$}{11/25}
\slider{$\ker\olddet=\left\{u\in\al E{}\;\middle|\;\det u=1\right\}$}{11/25}
\slideq{$u\in\al{E}{}$\linebreak$\det{\lambda u}$}{12/25}
\slider{$\lambda^{\dim E}\det u$}{12/25}
\slideq{Développement suivant une colonne\linebreak Développement suivant une ligne}{13/25}
\slider{$\det M=\sum{i=1}{n}{\l-1\r^{i+j}m_{i,j}\Delta_{i,j}\l M\r}$\linebreak$\det M=\sum{j=1}{n}{\l-1\r^{i+j}m_{i,j}\Delta_{i,j}\l M\r}$}{13/25}
\slideq{Mineur de position $\l i,j\r$ de $M$}{14/25}
\slider{$\Delta_{i,j}\l M\r=\det{(m_{k,l})_{\l k,l\r\in\llb1,n\rrb\setminus\left\{i\right\}\times\llb1,n\rrb\setminus\left\{j\right\}}}$}{14/25}
\slideq{Comatrice de $M$\linebreak$\com M$}{15/25}
\slider{$\l\l-1\r^{i+j}\Delta_{i,j}\l M\r\r_{\l k,l\r\in\llb1,n\rrb\setminus\left\{i\right\}\times\llb1,n\rrb\setminus\left\{j\right\}}$}{15/25}
\slideq{$\tmatrix|{\lambda_1\&\bullet\&\mdots\&\bullet\\0\&\ddots\&\ddots\&\vdots\\\vdots\&\ddots\&\ddots\&\bullet\\0\&\mdots\&0\&\lambda_n\\}|$}{16/25}
\slider{$\prod{i=1}{n}{\lambda_i}$}{16/25}
\slideq{$\phi$ est antisymétrique}{17/25}
\slider{$\phi\l x_1,\cdots,x_n\r=\eps\sigma\phi\l x_{\sigma\l1\r},\cdots,x_{\sigma\l n\r}\r$}{17/25}
\slideq{Ensemble des formes $n$-linéaires alternées}{18/25}
\slider{$\vect{\olddet_{\mathcal B}}$}{18/25}
\slideq{$\phi$ est alternée}{19/25}
\slider{$\phi\l x_1,\cdots,x_n\r=0$ s'il existe $i\neq j$ tel que $x_i=x_j$}{19/25}
\slideq{$\sl[n]{\mathbb K}$}{20/25}
\slider{$\ker\olddet=\left\{A\in\matgl n{\mathbb K}\;\middle|\;\det A=1\right\}$}{20/25}
\slideq{Valeurs des $x_k$ pour $A\tmatrix({x_1\\\vdots\\x_n\\})=B$\linebreak$A=\tmatrix[\mtxvline{thick}{1}\mtxvline{thick}{2}]({A_1\&\mdots\&A_n\\})$}{21/25}
\slider{$x_k=\frac{\det{\tmatrix[\mtxvline{thick}{1}\mtxvline{thick}{2}\mtxvline{thick}{3}\mtxvline{thick}{4}\mtxvline{thick}{5}\mtxvline{thick}{6}]({A_1\&\mdots\&A_{k-1}\&B\&A_{k+1}\&\mdots\&A_n\\})}}{\det A}$}{21/25}
\slideq{Cofacteur de position $\l i,j\r$ de $M$}{22/25}
\slider{$\l-1\r^{i+j}\Delta_{i,j}\l M\r$}{22/25}
\slideq{$\det A$}{23/25}
\slider{$\sum{\sigma\in\mathfrak S_n}{}{\eps\sigma a_{\sigma\l1\r,1}\cdots a_{\sigma\l n\r,n}}$\linebreak${}=\sum{\tau\in\mathfrak S_n}{}{\eps\tau a_{1,\tau\l1\r}\cdots a_{n,\tau\l n\r}}$}{23/25}
\slideq{Description du déterminant par les coordonnées\linebreak$\lc x_j\rc_{\mathcal B}=\tmatrix({a_{1,j}\\\vdots\\a_{n,j}\\})$}{24/25}
\slider{$\det[\mathcal B]{x_1,\cdots,x_n}=\sum{\sigma\in\mathfrak S_n}{}{\eps\sigma a_{\sigma\l1\r,1}\cdots a_{\sigma\l n\r,n}}$\linebreak${}=\sum{\tau\in\mathfrak S_n}{}{\eps\tau a_{1,\tau\l1\r}\cdots a_{n,\tau\l n\r}}$}{24/25}
\slideq{Expression de l'inverse par la comatrice}{25/25}
\slider{$M^{-1}=\frac{\com M^\top}{\det M}$}{25/25}
\end{document}
\documentclass[14pt,usepdftitle=false,aspectratio=169]{beamer}
\usepackage{preambule}
\setbeamercolor{structure}{fg=black}
\usepackage{matrices,al}\usepackage{polynomes,structures}\usepackage{bigoperators}\DeclareMathOperator{\oldid}{id}\newcommand{\id}{{\oldid}}
\hypersetup{pdftitle=Algèbre 2 -- Algèbre linéaire}
\title{Algèbre 2\\\emph{Algèbre linéaire}}
\author{}
\date{}
\begin{document}
\begin{frame}
    \titlepage
\end{frame}
\slideq{Famille génératrice de $E$}{1/31}
\slider{$\forall x\in E\;\exists\l \lambda_i\r_{i\in I},\;x=\sum{i\in I}{}{\lambda_ix_i}$\linebreak$\vect{\l x_i\r_{i\in I}}=E$}{1/31}
\slideq{Structure de $\al E{}$}{2/31}
\slider{$\l\al E{},+,\cdot,\circ\r$ est une $\mathbb K$-algèbre}{2/31}
\slideq{Caractérisation géométrique des projecteurs}{3/31}
\slider{$p$ est un projecteur si et seulement s'il existe deux sous-espaces $F$ et $G$ de $E$ tels que $F\oplus G=E$ avec $\forall\l f,g\r\in F\times G$, $p\l f+g\r=f$\linebreak$F=\im p$, $G=\ker p$\linebreak Un projecteur est une projection géométrique sur $\im p$ parallèlement à $\ker p$}{3/31}
\slideq{Polynome annulateur\linebreak$P\in\pol KX$ est annulateur de $u\in\al E{}$}{4/31}
\slider{$P\l u\r=0_{\al E{}}$}{4/31}
\slideq{Somme directe}{5/31}
\slider{$E\oplus F$ est directe si et seulement si $E\cap F=\left\{0\right\}$}{5/31}
\slideq{Structure des polynomes annulateurs}{6/31}
\slider{Idéal de $\pol KX$}{6/31}
\slideq{Endomorphisme nilpotent\linebreak$u\in\al E{}$}{7/31}
\slider{$\exists n\in\mathbb N,\;u^n=0_{\al E{}}$}{7/31}
\slideq{$\varphi\!:\!E\times F\to G$ est bilinéaire}{8/31}
\slider{$\forall\l x,x',y,y',\lambda\r\in E^2\times F^2\times\mathbb K$\linebreak$\varphi\l\lambda x+x',y\r=\lambda\varphi\l x,y\r+\varphi\l x',y\r$\linebreak$\varphi\l x,\lambda y+y'\r=\lambda\varphi\l x,y\r+\varphi\l x,y'\r$}{8/31}
\slideq{Si $E$ est un $\mathbb K$-ev\linebreak Un sous-ensemble $F$ de $E$ est un sous-espace vectoriel de $E$}{9/31}
\slider{$F$ est stable par les lois $+$ et $\cdot$ et les lois induites définissent sur $F$ une structure d'espace-vectoriel}{9/31}
\slideq{Endomorphisme d'espaces vectoriels}{10/31}
\slider{Application linéaire de $E$ dans lui-même\linebreak$\al E{}$}{10/31}
\slideq{Si $E$ est un $\mathbb K$-ev et $X\subset E$\linebreak$\vect X$}{11/31}
\slider{Plus petit sous-espace vectoriel de $E$ contenant $X$}{11/31}
\slideq{Automorphisme d'espaces vectoriels}{12/31}
\slider{Endomorphisme d'espaces vectoriels bijectif\linebreak$\gl E$}{12/31}
\slideq{Diagonalisation d'une symétrie}{13/31}
\slider{$s=\ker{s+\id}\oplus\ker{s-\id}$}{13/31}
\slideq{Symétrie}{14/31}
\slider{$s\circ s=\id$}{14/31}
\slideq{Endomorphisme diagonalisable\linebreak$\l b_i\r_{i\in I}$ une base de $E$}{15/31}
\slider{$\forall i\in I,\;\exists\lambda_i\in\mathbb K,\;f\l b_i\r=\lambda_ib_i$\linebreak Les $\lambda_i$ sont les valeurs propres\linebreak Si $x\neq0$, $f\l x\r=\lambda x$ est un vecteur propre associé à $\lambda$\linebreak$\ker{f-\lambda\id}$ est le sous-espace propre de $f$ associé à $\lambda$}{15/31}
\slideq{Caractérisation de l'image et diagonalisation d'un projecteur}{16/31}
\slider{$\im p=\ker{p-\id}$\linebreak$E=\ker p\oplus\ker{p-\id}$}{16/31}
\slideq{Projecteur}{17/31}
\slider{$p\circ p=p$}{17/31}
\slideq{Un ensemble $E$ est un espace vectoriel sur $\mathbb K$\linebreak$E$ est un $\mathbb K$-ev}{18/31}
\slider{$\l E,+\r$ est un groupe abélien\linebreak$E$ est muni d'une loi de composition externe $\cdot$ avec $\forall\l\lambda,\mu,x,y\r\in\mathbb K^2\times E^2$\linebreak$\l \lambda\mu\r x=\lambda\l\mu x\r$ (associativité externe ou pseudo-associativité)\linebreak$1_{\mathbb K} x=x$ (compatibilité du neutre de $\l \mathbb K,\times\r$)\linebreak$\lambda\l x+y\r=\lambda x+\lambda y$ (distributivité de $\cdot$ sur $+_{\scriptscriptstyle E}$)\linebreak$\l \lambda+\mu\r x=\lambda x+\mu x$ (distributivité de $\cdot$ sur $+_{\scriptscriptstyle\mathbb K}$)}{18/31}
\slideq{Image directe et réciproque de sous-espaces vectoriels par un homomorphisme}{19/31}
\slider{Si $E$ et $F$ sont deux groupes, et $f\in\al EF$ une application linéaire, $E'$ et $F'$ deux sous-espaces vectoriels de $E$ et $F$\linebreak $f\l E'\r$ est un sous-espace vectoriel de $F$\linebreak $f^{-1}\l F'\r$ est un sous-espace vectoriel de $F$}{19/31}
\slideq{$\vect X+\vect Y$}{20/31}
\slider{$\vect{X\cup Y}$}{20/31}
\slideq{Soit $E$ et $F$ deux $\mathbb K$-ev\linebreak Caractérisation des applications linéaires}{21/31}
\slider{$\forall\l\lambda,x,y\r\in\mathbb K\times E^2,\;f\l\lambda x+y\r=\lambda f\l x\r+f\l y\r$}{21/31}
\slideq{Si $\l A,+,\times\r$ est un anneau et $\mathbb K$ un corps\linebreak$A$ est une $\mathbb K$-algèbre}{22/31}
\slider{$\forall\l\lambda,x,y\r\in\mathbb K\times A^2$\linebreak$\lambda\cdot\l x\times y\r=\l\lambda\cdot x\r\times y=x\times\l\lambda\cdot y\r$}{22/31}
\slideq{Structure de $\al EF$}{23/31}
\slider{$\mathbb{K}$-ev}{23/31}
\slideq{Isomorphisme d'espaces vectoriels}{24/31}
\slider{Application linéaire bijective}{24/31}
\slideq{Base de $E$}{25/31}
\slider{Famille libre maximale de $E$\linebreak Famille génératrice minimale de $E$}{25/31}
\slideq{Caractérisation géométrique des symétries}{26/31}
\slider{$s$ est une symétrie si et seulement s'il existe deux sous-espaces $F$ et $G$ de $E$ tels que $F\oplus G=E$ avec $\forall\l f,g\r\in F\times G$, $s\l f+g\r=f-g$\linebreak$F=\ker{s-\id}$, $G=\ker{s+\id}$\linebreak Une symétrie est une symétrie géométrique par rapport à $\ker{s-\id}$ parallèlement à $\ker{s+\id}$}{26/31}
\slideq{Structure de $\ker{f}$\linebreak$f\!:\!E\to F$}{27/31}
\slider{Sous-espace vectoriel de $E$}{27/31}
\slideq{Si $E$ est un espace vectoriel et $F\subset E$\linebreak Caractérisation(s) des sous-espaces vectoriels}{28/31}
\slider{$0\in F$\linebreak$\forall\l x,y,\lambda\r\in F^2\times\mathbb K, \lambda x+y\in F$}{28/31}
\slideq{Structure de $\l\gl E,\circ\r$}{29/31}
\slider{Groupe}{29/31}
\slideq{Soit $E$ et $F$ deux $\mathbb K$-ev\linebreak$f\!:\!E\to F$ est une application linéaire}{30/31}
\slider{$\forall\l\lambda,x\r\in\mathbb K\times E,\;f\l\lambda x\r=\lambda f\l x\r$\linebreak$\forall\l x,y\r\in E^2,\; f\l x+y\r=f\l x\r+f\l y\r$}{30/31}
\slideq{Famille libre de $E$}{31/31}
\slider{$\forall\l\lambda_i\r_{i\in I},\;\sum{i\in I}{}{\lambda_ix_i}=0\Rightarrow\forall i\in I,\;\lambda_i=0$\linebreak$\forall x\in E\;\exists!\l \lambda_i\r_{i\in I},\;x=\sum{i\in I}{}{\lambda_ix_i}$}{31/31}
\end{document}
\documentclass[14pt,usepdftitle=false,aspectratio=169]{beamer}
\usepackage{preambule}
\setbeamercolor{structure}{fg=black}
\usepackage{polynomes}\usepackage{bigoperators}\let\oldmax\max\renewcommand\max[2][]{\oldmax_{#1}\l#2\r}\let\oldmin\min\renewcommand\min[2][]{\oldmin_{#1}\l#2\r}
\hypersetup{pdftitle=Algèbre 1 -- Polynômes et fractions rationnelles}
\title{Algèbre 1\\\emph{Polynômes et fractions rationnelles}}
\author{}
\date{}
\begin{document}
\begin{frame}
    \titlepage
\end{frame}
\slideq{Propriétés de $\varphi\!:\!\pol KX\to\pol Kx$}{1/16}
\slider{Homomorphisme d'anneaux surjectif}{1/16}
\slideq{Structure de $\pol AX$}{2/16}
\slider{Anneau commutatif}{2/16}
\slideq{$i$-ième polynôme interpolateur de Lagrange}{3/16}
\slider{$L_i=\prod{\substack{j=0\\j\neq i}}{n}{\frac{X-x_j}{x_i-x_j}}$}{3/16}
\slideq{Structure de $\pol AX$ si $\mathbb A$ est intègre}{4/16}
\slider{Anneau intègre commutatif}{4/16}
\slideq{$\val P$}{5/16}
\slider{$\min{\left\{n\in\mathbb N\;\middle|\;a_n\neq0\right\}}$}{5/16}
\slideq{Décomposition en éléments simples dans $\fr RX$\linebreak$F=\frac PQ$ avec $Q=Q_1^{\alpha_1}\cdots Q_k^{\alpha_k}$, $\deg{Q_i}\leqslant2$}{6/16}
\slider{$F=E+\sum{i=0}{k}{\sum{j=0}{\alpha_i}{\frac{A_{i,j}}{Q_i^j}}}$\linebreak$\deg{A_{i,j}}<\deg{Q_i}$}{6/16}
\slideq{Un anneau commutatif $\mathbb B$ est une algèbre sur un anneau commutatif $\mathbb A$\linebreak$\mathbb A$-algèbre $\mathbb B$}{7/16}
\slider{$\forall\l \lambda,\mu,x,y\r\in\mathbb A^2\times\mathbb B^2$\linebreak$\l \lambda\mu\r y=\lambda\l\mu y\r$\linebreak$\lambda\l xy\r=\l\lambda x\r y=x\l\lambda y\r$\linebreak$\l\lambda+\mu\r x=\lambda x+\mu x$\linebreak$\lambda\l x+y\r=\lambda x+\lambda y$\linebreak$1_{\mathbb A}x=x$}{7/16}
\slideq{Décomposition en éléments simples dans $\fr CX$\linebreak$F$ a des pôles $\l r_1,\cdots,r_k\r$ de multiplicité $\l\alpha_1,\cdots,\alpha_k\r$}{8/16}
\slider{$F=E+\sum{i=0}{k}{\sum{j=0}{\alpha_i}{\frac{\lambda_{i,j}}{\l X-r_i\r^j}}}$}{8/16}
\slideq{Interpolation de la fonction $f$ aux points $\l x_1,\cdots,x_n\r$}{9/16}
\slider{$P=\sum{i=1}{n}{f\l x_i\r L_i}$}{9/16}
\slideq{Décomposition en éléments simples de $\frac{P'}{P}$ de racines $\l r_1,\cdots,r_k\r$ de multiplicité $\l\alpha_1,\cdots,\alpha_k\r$}{10/16}
\slider{$\frac{P'}{P}=\sum{i=1}{n}{\frac{\alpha_i}{X-r_i}}$}{10/16}
\slideq{$\deg P$}{11/16}
\slider{$\max{\left\{n\in\mathbb N\;\middle|\;a_n\neq0\right\}}$}{11/16}
\slideq{Coefficient avec le terme $X-r$ de la décomposition en éléments simples de $F=\frac PQ$ dans $\fr CX$}{12/16}
\slider{$\lambda=\frac{P\l r\r}{\frac{Q}{X-r}\l r\r}=\frac{P\l r\r}{Q'\l r\r}$}{12/16}
\slideq{Partie polaire d'une décomposition en éléments simples dans $\fr CX$\linebreak$F$ a des pôles $\l r_1,\cdots,r_k\r$ de multiplicité $\l\alpha_1,\cdots,\alpha_k\r$}{13/16}
\slider{$\sum{j=0}{\alpha_i}{\frac{\lambda_{i,j}}{\l X-r_i\r^j}}$}{13/16}
\slideq{Relations de Viète\linebreak$P=\sum{k=0}{n}{a_kX^k}$ est scindé à racines $\l r_1,\cdots,r_n\r$}{14/16}
\slider{$\sum{K\in\mathcal P_k\l\llb1,n\rrb\r}{}{\prod{j\in K}{}{r_j}}=\l-1\r^k\frac{a_{n-k}}{a_n}$}{14/16}
\slideq{Propriétés de $\varphi\!:\!\pol KX\to\pol Kx$ si $\car{\mathbb K}=0$}{15/16}
\slider{Isomorphisme d'anneaux}{15/16}
\slideq{Structure de $\fr KX$}{16/16}
\slider{Corps}{16/16}
\end{document}
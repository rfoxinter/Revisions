\documentclass[14pt,usepdftitle=false,aspectratio=169]{beamer}
\usepackage{preambule}
\setbeamercolor{structure}{fg=black}
\usepackage{analyse,usuelles}\let\phi\varphi
\hypersetup{pdftitle=Analyse -- Intégration}
\title{Analyse\\\emph{Intégration}}
\author{}
\date{}
\begin{document}
\begin{frame}
    \titlepage
\end{frame}
\slideq{Structure de $\esc{\left[a,b\right]}$}{1/19}
\slider{Sous-espace vectoriel de $\mathbb R^{\left[a,b\right]}$}{1/19}
\slideq{Pas d'une subdivision $\sigma$}{2/19}
\slider{$p\l\sigma\r=\oldmax\limits_{i\in\llb0,n-1\rrb}\l\sigma_{i+1}-\sigma_i\r$}{2/19}
\slideq{Subdivision associée à $f\!:\!\left[a,b\right]\to\mathbb R$}{3/19}
\slider{Subdivision de $\left[a,b\right]$ telle que $f$ soit constante sur les $\left]\sigma_i,\sigma_{i+1}\right[$}{3/19}
\slideq{Fonction continue par morceaux sur un intervalle $I$}{4/19}
\slider{$f$ est continue sur tout segment inclus dans $I$}{4/19}
\slideq{Sommes de Riemann}{5/19}
\slider{Si $\sigma^n=\l \sigma_{n,k}\r_{k\in\llb1,l_n\rrb}$ une subdivision de $\left[a,b\right]$ tel que $\lim[n\to+\infty]{p\l\sigma^n\r}=0$ et $x_{n,k}\in\left]\sigma_{n,k},\sigma_{n,k+1}\right[$\linebreak$\int[x][a][b]{f\l x\r}=\lim[n\to+\infty]{\sum{k=0}{l_n}{p_{n,k}\times f\l x_{n,k}\r}}$\linebreak$p_{n,k}=\sigma_{n,k+1}-\sigma_{n,k}$}{5/19}
\slideq{Subdivision d'un intervalle $\left[a,b\right]$}{6/19}
\slider{$\sigma=\l a=\sigma_0<\cdots<\sigma_n=b\r$}{6/19}
\slideq{Relation de raffinement}{7/19}
\slider{$\sigma\leqslant\tau\Leftrightarrow\tau\subset\sigma$}{7/19}
\slideq{Sommes de Riemann sur $\left[0,1\right]$}{8/19}
\slider{$\int[x][0][1]{f\l x\r}=\lim[n\to+\infty]{\frac1n\sum{k=0}{n-1}{f\l\frac kn\r}}$\linebreak${}=\lim[n\to+\infty]{\frac1n\sum{k=1}{n}{f\l\frac kn\r}}$}{8/19}
\slideq{Intégrabilité au sens de Riemann de $f$ sur $\left[a,b\right]$}{9/19}
\slider{$\forall\varepsilon>0,\;\exists\l g,h\r\in\esc[-]f\times\esc[+]f$\linebreak$\int[x][a][b]{h\l x\r-g\l x\r}<\varepsilon$}{9/19}
\slideq{$\esc[+]f$}{10/19}
\slider{$\left\{g\in\esc{\left[a,b\right]}\;\middle|\;\forall x\in\left[a,b\right],\;g\l x\r\geqslant f\l x\r\right\}$}{10/19}
\slideq{Fonctions continues par morceaux sur un segment}{11/19}
\slider{Il existe une subdivision $a=\sigma_0<\cdots<\sigma_n=b$ de $\left[a,b\right]$ tel que $f$ soit continue sur les $\left]\sigma_i,\sigma_{i+1}\right[$ et admette une limite à droite et à gauche à chaque $\sigma_i$}{11/19}
\slideq{$\esc[-]f$}{12/19}
\slider{$\left\{g\in\esc{\left[a,b\right]}\;\middle|\;\forall x\in\left[a,b\right],\;g\l x\r\leqslant f\l x\r\right\}$}{12/19}
\slideq{Critère séquentiel d'intégrabilité}{13/19}
\slider{$\exists\l\l\phi_n\r,\l\theta_n\r\r\in\l\esc{\left[a,b\right]}^{\mathbb N}\r^2$\linebreak$\left|f\l x\r-\phi_n\l x\r\right|\leqslant\theta_n\l x\r$\linebreak$\lim[n\to+\infty]{\int[x][a][b]{\theta_n\l x\r}}=0$\linebreak$\lim[n\to+\infty]{\int[x][a][b]{\phi_n\l x\r}}=\int[x][a][b]{f\l x\r}$}{13/19}
\slideq{Critère d'intégrabilité de $f$ par encadrement}{14/19}
\slider{$\exists\l\phi,\theta\r\in\esc{[a,b]}^2$\linebreak$\left|f\l x\r-\phi\l x\r\right|\leqslant\theta\l x\r\wedge\int[x][a][b]{\theta\l x\r}\leqslant\varepsilon$}{14/19}
\slideq{Moyenne de $f$ sur $\left[a,b\right]$}{15/19}
\slider{$\frac{1}{b-a}\int[x][a][b]{f\l x\r}$}{15/19}
\slideq{Structure de $\fint{[a,b]}$}{16/19}
\slider{$\mathbb R$ espace vectoriel}{16/19}
\slideq{Intégrale de Riemann}{17/19}
\slider{$\int[x][a][b]{f\l x\r}$\linebreak${}=\sup{\int[x][a][b]{g\l x\r}\;\middle|\;g\in\esc[-]f}$\linebreak${}=\inf{\int[x][a][b]{g\l x\r}\;\middle|\;g\in\esc[+]f}$}{17/19}
\slideq{Intégrale d'une fonction en escalier}{18/19}
\slider{$\int[x][a][b]{f\l x\r}=\sum{i=0}{n-1}{\l \sigma_{i+1}-\sigma_i\r f_i}$\linebreak$f_i$ est la valeur constante de $f$ sur $\left]\sigma_i,\sigma_{i+1}\right[$}{18/19}
\slideq{Fonctions en escalier}{19/19}
\slider{$f\!:\!\left[a,b\right]\to\mathbb R$ est en escalier s'il existe $\sigma=\l a=\sigma_0<\cdots<\sigma_n=b\r$ telle que $f$ soit constante sur les $\left]\sigma_i,\sigma_{i+1}\right[$}{19/19}
\end{document}
\documentclass[14pt,usepdftitle=false,aspectratio=169]{beamer}
\usepackage{preambule}
\setbeamercolor{structure}{fg=black}
\let\oldmax\max\renewcommand\max[2][]{\oldmax_{#1}\l#2\r}\let\oldmin\min\renewcommand\min[2][]{\oldmin_{#1}\l#2\r}\let\oldsup\sup\renewcommand\sup[2][]{\oldsup_{#1}\l#2\r}\let\oldinf\inf\renewcommand\inf[2][]{\oldinf_{#1}\l#2\r}\usepackage{arithmetique}\usepackage{bigoperators}\usepackage{matrices}
\hypersetup{pdftitle=Algèbre 1 -- Arithmétique}
\title{Algèbre 1\\\emph{Arithmétique}}
\author{}
\date{}
\begin{document}
\begin{frame}
    \titlepage
\end{frame}
\slideq{Anneau euclidien}{1/13}
\slider{Si $\mathbb A$ est un anneau intègre, avec un stathme ($v\!:\!\mathbb A\setminus\left\{0\right\}\to\mathbb N$)\linebreak$A$ est euclidien si $\forall a\in\mathbb A,\;\forall b\in\mathbb A\setminus\left\{0\right\}\;\exists\l q,r\r\in\mathbb A^2,\;a=bq+r$\linebreak$r=0\vee v\l r\r<v\l b\r$}{1/13}
\slideq{Distributivité de $\times$ sur $\wedge$ et $\vee$}{2/13}
\slider{Si $a\neq0$ et $b\neq0$\linebreak$\l a\wedge b\r\times c=\l ac\r\wedge\l bc\r$\linebreak$\l a\vee b\r\times c=\l ac\r\vee\l bc\r$}{2/13}
\slideq{Théorème d'Euler}{3/13}
\slider{$n\in\mathbb N,\;x\in\mathbb N^*,\;\cgr{x^{\phi\l n\r}}{1}{n}$}{3/13}
\slideq{$a\wedge b$}{4/13}
\slider{$\max{\left\{n\in\mathbb N,n\div a\wedge n\div b\right\}}$\linebreak$\max[\l\mathbb N^*,\div\r]{\left\{n\in\mathbb N,n\div a\wedge n\div b\right\}}$\linebreak$\inf[\l\mathbb N^*,\div\r]{a,b}$\linebreak$a\mathbb Z+b\mathbb Z$\linebreak$\l a\r+\l b\r$ pour un anneau principal}{4/13}
\slideq{$\phi(n)$}{5/13}
\slider{$\left|\l\mathbb Z/n\mathbb Z\r^\times\right|$}{5/13}
\slideq{Divisibilité avec le produit}{6/13}
\slider{Si $a\wedge b=1$, $a\div c\wedge b\div c$, alors $ab\div c$}{6/13}
\slideq{Théorème des restes chinois\linebreak\linebreak\tmatrix\{{\cgr{x}{b_1}{a_1} \\ \vdots \\ \cgr{x}{b_n}{a_n} \\}.\linebreak\linebreak$\forall i\in\llb1,n\rrb,\;\forall j\in\llb1,n\rrb\setminus\left\{i\right\},\;a_i\wedge a_j=1$}{7/13}
\slider{$\widehat{a_i}=\prod{j\in\llb1,n\rrb\setminus\left\{i\right\}}{}{a_j}$\linebreak\linebreak$a_iu_i+\widehat{a_i}v_i=1$\linebreak\linebreak$\cgr{x}{\sum{i=1}{n}{b_iv_i\widehat{a_i}}}{\prod{i=1}{n}{a_i}}$}{7/13}
\slideq{Relation entre $\wedge$ et $\vee$}{8/13}
\slider{$\l a\wedge b\r\l a\vee b\r=ab$}{8/13}
\slideq{Lemme d'Euclide}{9/13}
\slider{Si $a\div bc$ et $a\in\mathbb P$, alors $a\div b\vee a\div c$\linebreak Si $a\wedge b=1$ et $a\wedge c=1$, alors $a\wedge bc=1$}{9/13}
\slideq{$a\vee b$}{10/13}
\slider{$\min{\left\{n\in\mathbb N,a\div n\wedge b\div n\right\}}$\linebreak$\max[\l\mathbb N^*,\div\r]{\left\{n\in\mathbb N,a\div n\wedge b\div n\right\}}$\linebreak$\sup[\l\mathbb N^*,\div\r]{a,b}$\linebreak$a\mathbb Z\cap b\mathbb Z$\linebreak$\l a\r\cap\l b\r$ pour un anneau principal}{10/13}
\slideq{Théorème de Fermat}{11/13}
\slider{$p\in\mathbb P,\;a\in\mathbb N,\;\cgr{a^p}{a}{p}$\linebreak Si $p$ ne divise pas $a$, $\cgr{a^{p-1}}{1}{p}$}{11/13}
\slideq{Lemme de Gauss}{12/13}
\slider{Si $a\div bc$ et $a\wedge b=1$, alors $a\div c$}{12/13}
\slideq{Formule de Legendre}{13/13}
\slider{$v_p\l n!\r=\sum{k=1}{+\infty}{\left\lfloor\frac n{p^k}\right\rfloor}$}{13/13}
\end{document}
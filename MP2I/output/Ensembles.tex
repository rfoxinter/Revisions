\documentclass[14pt,usepdftitle=false,aspectratio=169]{beamer}
\usepackage{preambule}
\setbeamercolor{structure}{fg=black}
\usepackage{dsft}
\hypersetup{pdftitle=Fondements -- Ensembles}
\title{Fondements\\\emph{Ensembles}}
\author{}
\date{}
\begin{document}
\begin{frame}
    \titlepage
\end{frame}
\slideq{Lemme de (Kuratowski-)Zorn}{1/7}
\slider{Tout ensemble inductif admet un élément maximal\linebreak Avec l'axiome du choix}{1/7}
\slideq{$\1{A\cap B}$}{2/7}
\slider{$\1{A}\1{B}$}{2/7}
\slideq{$\1{A\vartriangle B}$}{3/7}
\slider{$\1{A\vartriangle B}\equiv\1 A+\1 B\ \left[2\right]$}{3/7}
\slideq{$\1{\complement_EA}$}{4/7}
\slider{$1-\1{A}$}{4/7}
\slideq{$\1{A\cup B}$}{5/7}
\slider{$\1{A}+\1{B}-\1{A\cap B}=\1{A}+\1{B}-\1{A}\1{B}$}{5/7}
\slideq{$\1{A\uplus B}$}{6/7}
\slider{$\1{A}+\1{B}$}{6/7}
\slideq{Si $\l E,\leqslant\r$ est un ensemble ordonné\linebreak $E$ est un ensemble inductif}{7/7}
\slider{$\forall F\subset E$ avec $F$ totalement ordonné, $F$ admet un majorant dans $E$}{7/7}
\end{document}
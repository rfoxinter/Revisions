\documentclass[14pt,usepdftitle=false,aspectratio=169]{beamer}
\usepackage{preambule}
\setbeamercolor{structure}{fg=black}
\usepackage{equivalents}\let\oldlim\lim\renewcommand\lim[2]{\oldlim\limits_{#1}\l#2\r}\let\oldln\ln\renewcommand\ln[2][]{\oldln^{#1}\l#2\r}\usepackage{complexes,trigo}
\hypersetup{pdftitle=Analyse -- Calcul asymptotique}
\title{Analyse\\\emph{Calcul asymptotique}}
\author{}
\date{}
\begin{document}
\begin{frame}
    \titlepage
\end{frame}
\slideq{$u_n=\Om{v_n}$\linebreak Définition avec les suites si $\l v_n\r$ ne s'annule pas}{1/26}
\slider{$\exists\l\mu_n\r,\;\exists n_0\in\mathbb N,\;\forall n\geqslant n_0,\;u_n=\mu_nv_n$\linebreak Avec $\l\mu_n\r$ minorée}{1/26}
\slideq{$u_n=\O{1}$}{2/26}
\slider{$\l u_n\r$ est borné}{2/26}
\slideq{$\eq{u_n}{v_n}$}{3/26}
\slider{$u_n=v_n+\o{v_n}$}{3/26}
\slideq{$u_n=\Om{v_n}$\linebreak Définition avec $\operatorname{O}$}{4/26}
\slider{$v_n=\O{u_n}$}{4/26}
\slideq{Formule de Stirling}{5/26}
\slider{$\eq[+\infty]{n!}{\sqrt{2\pi n}\l\frac{n}{\e}\r^n}$}{5/26}
\slideq{$\eq{u_n}{u_n'}$\linebreak$a\in\mathbb R$}{6/26}
\slider{$\eq{u_n^a}{v_n^a}$}{6/26}
\slideq{$\eq{u_n}{u_n'}\wedge\eq{v_n}{v_n'}$}{7/26}
\slider{$\eq{u_nv_n}{u_n'v_n'}$}{7/26}
\slideq{$u_n=\Th{v_n}$\linebreak Définition avec un encadrement}{8/26}
\slider{$\exists \l M,M'\r\in\l\mathbb{R}_+\r^2,\;\exists n_0\in\mathbb N,\;\forall n\geqslant n_0$\linebreak$M\left|v_n\right|\leqslant\left|u_n\right|\leqslant M'\left|v_n\right|$}{8/26}
\slideq{Transitivité de $\operatorname{o}$ et $\operatorname{O}$}{9/26}
\slider{$u_n=\O{v_n}\wedge v_n=\O{w_n}\Rightarrow  u_n=\O{w_n}$\linebreak $u_n=\o{v_n}\wedge v_n=\o{w_n}\Rightarrow  u_n=\o{w_n}$\linebreak $u_n=\o{v_n}\wedge v_n=\O{w_n}\Rightarrow  u_n=\o{w_n}$\linebreak $u_n=\O{v_n}\wedge v_n=\o{w_n}\Rightarrow  u_n=\o{w_n}$}{9/26}
\slideq{Implication entre $\operatorname{o}$ et $\operatorname{O}$}{10/26}
\slider{$u_n=\o{v_n}\Rightarrow  u_n=\O{v_n}$}{10/26}
\slideq{$u_n=\o{1}$}{11/26}
\slider{$\l u_n\r$ tend vers $0$}{11/26}
\slideq{$u_n=\Th{v_n}$\linebreak Définition avec $\operatorname{O}$ et $\operatorname{\Omega}$}{12/26}
\slider{$u_n=\O{v_n}\wedge u_n=\Om{v_n}$}{12/26}
\slideq{$\eq{u_n}{u_n'}\wedge\eq{v_n}{v_n'}$\linebreak$u_n=\o{v_n}$}{13/26}
\slider{$u_n'=\o{v_n'}$}{13/26}
\slideq{$u_n=\o{v_n}$\linebreak Définition avec les suites}{14/26}
\slider{$\exists\l\varepsilon_n\r,\;\exists n_0\in\mathbb N,\;\forall n\geqslant n_0,\;u_n=\varepsilon_nv_n$\linebreak Avec $\lim{n\to\infty}{\varepsilon_n}=0$}{14/26}
\slideq{$u_n=\Th{v_n}$\linebreak Définition avec les suites}{15/26}
\slider{$\exists\l\mu_n\r,\;\exists n_0\in\mathbb N,\;\forall n\geqslant n_0,\;u_n=\mu_nv_n$\linebreak Avec $\forall n\in\mathbb N,\;0<\varepsilon\leqslant\mu_n\leqslant M$}{15/26}
\slideq{$u_n=\o{v_n}$\linebreak Définition avec un epsilon}{16/26}
\slider{$\forall\varepsilon\in\mathbb{R}_+,\;\exists n_0\in\mathbb N,\;\forall n\geqslant n_0,\;\left|u_n\right|\leqslant\varepsilon\left|v_n\right|$}{16/26}
\slideq{Équivalent d'un polynôme $P$ de degré $d=\deg\l P\r$ et de monôme dominant $a_dX^d$}{17/26}
\slider{$\eq{P\l n\r}{a_dn^d}$}{17/26}
\slideq{$u_n=\Om{v_n}$\linebreak Définition avec un minorant}{18/26}
\slider{$\exists M\in\mathbb{R}_+,\;\exists n_0\in\mathbb N,\;\forall n\geqslant n_0,\;\left|u_n\right|\geqslant M\left|v_n\right|$}{18/26}
\slideq{$\eq{u_n}{u_n'}\wedge\eq{v_n}{v_n'}$\linebreak Avec $\l v_n\r$ qui ne s'annule pas à partir d'un certain rang}{19/26}
\slider{$\eq{\frac{u_n}{v_n}}{\frac{u_n'}{v_n'}}$}{19/26}
\slideq{Sommes de $\operatorname{o}$ et $\operatorname{O}$}{20/26}
\slider{$u_n=\o{w_n}\wedge v_n=\o{w_n}\Rightarrow u_n+v_n=\o{w_n}$\linebreak $u_n=\O{w_n}\wedge v_n=\O{w_n}\Rightarrow u_n+v_n=\O{w_n}$\linebreak $u_n=\o{w_n}\wedge v_n=\O{w_n}\Rightarrow u_n+v_n=\O{w_n}$\linebreak $u_n=\O{w_n}\wedge v_n=\o{w_n}\Rightarrow u_n+v_n=\O{w_n}$}{20/26}
\slideq{$u_n=\O{v_n}$\linebreak Définition avec un majorant}{21/26}
\slider{$\exists M\in\mathbb{R}_+,\;\exists n_0\in\mathbb N,\;\forall n\geqslant n_0,\;\left|u_n\right|\leqslant M\left|v_n\right|$}{21/26}
\slideq{$u_n=\O{v_n}$\linebreak Définition avec les suites}{22/26}
\slider{$\exists\l\mu_n\r,\;\exists n_0\in\mathbb N,\;\forall n\geqslant n_0,\;u_n=\mu_nv_n$\linebreak Avec $\l\mu_n\r$ bornée}{22/26}
\slideq{Équivalents classiques}{23/26}
\slider{\begin{minipage}{0.5\textwidth}$\eq[0]{\ln{1+x}}{x}$\\$\eq[0]{\e^x-1}{x}$\\ Pour $a\neq0$\hfill\break$\eq[0]{\l1+x\r^a-1}{ax}$\\$\eq[0]{\sin{x}}{x}$\\$\eq[0]{\cos{x}-1}{-\frac{x^2}{2}}$\end{minipage}\begin{minipage}{0.5\textwidth}$\eq[0]{\tan{x}}{x}$\\$\eq[0]{\sh{x}}{x}$\\$\eq[0]{\ch{x}-1}{\frac{x^2}{2}}$\\$\eq[0]{\th{x}}{x}$\\$\eq[0]{\asin{x}}{x}$\\$\eq[0]{\atan{x}}{x}$\end{minipage}}{23/26}
\slideq{$u_n=l+\o{1}$}{24/26}
\slider{$\lim{n\to\infty}{u_n}=l$}{24/26}
\slideq{Produits de $\operatorname{o}$ et $\operatorname{O}$}{25/26}
\slider{$u_n=\o{w_n}\wedge v_n=\o{x_n}\Rightarrow u_nv_n=\o{w_nx_n}$\linebreak$u_n=\O{w_n}\wedge v_n=\o{x_n}\Rightarrow u_nv_n=\o{w_nx_n}$\linebreak$u_n=\o{w_n}\wedge v_n=\O{x_n}\Rightarrow u_nv_n=\o{w_nx_n}$\linebreak$u_n=\O{w_n}\wedge v_n=\O{x_n}\Rightarrow u_nv_n=\O{w_nx_n}$\linebreak$w_n\o{x_n}=\o{w_nx_n}$\linebreak$w_n\O{x_n}=\O{w_nx_n}$}{25/26}
\slideq{$\eq{u_n}{v_n}$}{26/26}
\slider{$\lim{n\to\infty}{\frac{u_n}{v_n}}=1$}{26/26}
\end{document}
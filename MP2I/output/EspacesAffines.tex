\documentclass[14pt,usepdftitle=false,aspectratio=169]{beamer}
\usepackage{preambule}
\setbeamercolor{structure}{fg=black}

\hypersetup{pdftitle=Algèbre 2 -- Espaces affines}
\title{Algèbre 2\\\emph{Espaces affines}}
\author{}
\date{}
\begin{document}
\begin{frame}
    \titlepage
\end{frame}
\slideq{Direction d'un sous-espace affine $F$ de $E$}{1/6}
\slider{L'unique sous-ensemble $V$ de $T$ tel que $F=A+V$}{1/6}
\slideq{Espace affine}{2/6}
\slider{$\forall\l x,t,u\r\in\mathbb E\times T^2,\;x+\l\vec t+\vec u\r=\l x+\vec t\r+\vec u$\linebreak$\forall x\in E,\;x+\vec0=x$\linebreak$\forall\l x,y\r\in E^2,\;\exists\vec t\in T,\;x=y+\vec t$\linebreak$\l\forall x\in E,\;x+\vec t=x\r\Rightarrow\vec t=\vec0$\linebreak Le dernier point peut être remplacé par $\forall x\in E,\;\l x+\vec t=x\Rightarrow\vec t=\vec0\r$}{2/6}
\slideq{Orbite d'un point $x\in E$ sous l'action d'un sous-ensemble $S$ de $T$}{3/6}
\slider{$\left\{x+\vec s,\;s\in S\right\}$}{3/6}
\slideq{Propriétés d'une fibre d'une application linéaire $u\in\mathcal L\l E,F\r$}{4/6}
\slider{$u^{-1}\l\left\{a\right\}\r$ est soit vide, soit dirigé par $\ker\l u\r$}{4/6}
\slideq{Sous-espace affine d'un espace affine $E$}{5/6}
\slider{$F\subset E$ est un sous-espace affine de $E$ s'il existe $A\in E$ et $V\subset T$ tel que $F$ est l'orbite de A sous l'action de V}{5/6}
\slideq{$\tau_{\vec t}\l X\r$}{6/6}
\slider{$\left\{x+\vec t,\;x\in X\right\}$}{6/6}
\end{document}